\subsection{Enhancements at version 6.0}

Features added at version 6.0 include:
\begin{itemize}
\item New lumping of death states.
\item Model pruning supported by the \verb'PRUNEIF' statement.
\item Improvements in efficiently generating large models.
\item Bit mapping of Boolean variables.
\item The logical operator \verb'NOT' for use in Boolean expressions.
\item The concatenation character for use in rate expressions.
\end{itemize}

These new features are discussed below.
 
\subsubsection{Lumping of Death States}
 
  Versions of ASSIST prior to 6.0 allowed the user to specify that all of the
death states could be lumped together as one death state.  This significantly
reduced the size of the model, but was unacceptable for users wishing to
discern the relative importance of various failure modes of the system.  
 
     ASSIST versions 6.0 and higher lump the death states of the model
according to the
\verb'DEATHIF' statement that defined that state to be a death state.  If
the ASSIST input file contains only one \verb'DEATHIF' statement, then the
model generated will also contain only one death state, state 1 of the
model.  However, if the input file contains 5 \verb'DEATHIF' statements,
then states 1 through 5 of the model will be death states.  State 1 will
contain all death states satisfying the conditions described by the first
\verb'DEATHIF' statement.  State 2 will contain all death states satisfying
\verb'DEATHIF' statement 2 and not satisfying \verb'DEATHIF' statement 1,
and so on.  Thus, the probability of system failure due to each condition
specified by a \verb'DEATHIF' statement will be given by the probability of
being in the corresponding death state.
 
     Lumping of death states in the manner described above is the default in
ASSIST version 6.0 and higher.
Users wishing to have all of the death states enumerated
must specify \verb'ONEDEATH = 0;' in the input file.
 
\subsubsection{Model Pruning}
 
     A model of a system with a large number of components tends to have
many long paths consisting of one or two failures of each type of component
before a condition of system failure is reached.  Because the occurrence of
so many failures is unlikely during a short mission, these long paths
typically contribute insignificant amounts to the probability of system
failure.  The dominant failure modes of the system are typically the short
paths to system failure consisting of failures of like components.  Model
pruning can be used to eliminate the long paths by conservatively assuming
that system failure occurs earlier on those paths.
 
     Model pruning is supported in ASSIST by the \verb'PRUNEIF' statement.
The \verb'PRUNEIF' statement has the same syntax as the \verb'DEATHIF'
statements:
\begin{flushleft} 
~~~~~\verb'PRUNEIF' {\em condition};
\end{flushleft}
where {\em condition} is a Boolean expression describing the states to be
pruned.  For those users that cannot remember the correct spelling of
\verb'PRUNEIF', the following syntax
\begin{flushleft}
     \verb'PRUNIF' {\em condition};
\end{flushleft}
is also acceptable.
 
     The pruned states generated are lumped together according to the
\verb'PRUNEIF' statement they satisfy, just as the death states are.  In a
model generated from an input file with 3 \verb'DEATHIF' statements and 2
\verb'PRUNEIF' statements, states 1 through 3 will be death states
corresponding to the 3 \verb'DEATHIF' statements and states 4 and 5 will be
pruned states corresponding to the 2 \verb'PRUNEIF' statements.
 
     ASSIST generates a statement in the model file identifying the pruned
states in the model.  For example, the model with 3 death states and 2
pruned states would contain the statement
\begin{verbatim}
     PRUNESTATES = (4,5);
\end{verbatim}
which will be used by SURE version 6 to separately list the prune-state
probabilities from the death-state probabilities.  Versions of SURE earlier
than 6.0 will simply list the pruned states as if they were death states.
 
\subsubsection{Efficiency Improvements}
 
     A number of improvements have been made in the efficiency with which
ASSIST generates and stores the model.  The change that can most be impacted
by the user is bit mapping of Boolean variables.  Up to 64 state space
variables specified as having ranges 0..1 (even in arrays) in the input file
will be stored efficiently as 1-bit Boolean variables rather than as 32-bit
integers.  When many Boolean variables are used, this greatly decreases the
memory needed to store each state.  That, in turn, significantly reduces
model generation time because less page swapping of virtual memory is
needed, and allows the generation of considerably larger models.  There is no
effect on the model generated.
 
     To take advantage of this feature, you may want to rethink your model
description strategies.  Using a different Boolean variable (i.e., range
0..1) to represent the state of each component of a type may now be more
efficient than using a single non-Boolean variable to represent the number of
working components of a type.

\subsubsection{Concatenation Character}
     The ASSIST program now allows the user to concatenate identifiers or 
values in the rate expression using the '\verb'^'' character.  For example, the
following ASSIST statement:
\begin{verbatim}
FOR I=1,3;
   IF xxx TRANTO xxx BY DELTA^I;
ENDFOR:
\end{verbatim}
will result in model transitions such as:
\begin{verbatim}
XXX,XXX = DELTA1;
XXX,XXX = DELTA2;
XXX,XXX = DELTA3;
\end{verbatim}
The concatenation character can be used anywhere in a rate statement.
When ASSIST encounters this character in a rate expression, it will 
recognize that the previous identifier name has terminated but will not leave
a space before printing the next term in the expression.

The following ASSIST input file shows how the concatenation character
can be used in a model of multiple triads with a pool of cold spares in
which the different triads have differing reconfiguration rates.
The expression \verb'BY FAST DELTA^J'
within the \verb'FOR' loop results in a reconfiguration rate of
``FAST DELTA1,'' for triad 1, ``FAST DELTA2,'' for triad 2, etc.

 \begin{verbatim}
(*  MULTIPLE TRIADS WITH POOL OF COLD SPARES *)
 
INPUT N_TRIADS;              (* Number of triads initially *)
INPUT N_SPARES;              (* Number of spares *)
LAMBDA = 1E-4;               (* Failure rate of active processors *)
 
SPACE = (NW: ARRAY[1..N_TRIADS] OF 0..3,  (* Number working procs per triad *)
         NS);                             (* Number of spare processors *)
START = (N_TRIADS OF 3, N_SPARES);
 
FOR J = 1, N_TRIADS;

      (* Active processor failure *)
   IF NW[J] > 0 TRANTO NW[J] = NW[J]-1 BY NW[J]*LAMBDA;

      (* Replace failed processor with working spare *)
   IF (NW[J] < 3) AND (NS > 0) TRANTO NW[J] = NW[J]+1, NS = NS-1 
      BY FAST DELTA^J;
 
   DEATHIF NW[J] < 2;      (* Two faults in a triad is system failure *)
 
ENDFOR;
\end{verbatim}

\subsubsection{Logical Operator NOT}

     The logical operator \verb'NOT' can now be used in Boolean expressions.
Examples of valid uses of this operator are:
\begin{verbatim}
if not(s1 <= s2) tranto s1=s1+1 by foo;
deathif not (s2 = 0 and s3 < 5);
deathif not s3 < 5 or not (s3 = 0);
deathif not not s4=4;
\end{verbatim}
 
\subsubsection{Trimming}

     A new method for reducing the size of semi-Markov reliability models
was recently developed by Dr. Allan White and Daniel Palumbo of NASA Langley.
The details of this trimming method and the theorem that this method is
conservative are given in their paper in RAMS 1990 entitled ``State
Reduction for Semi-Markov Reliability Models.''  The user is cautioned
to read this paper to determine the characteristics of systems for which
this trimming method is guaranteed to be conservative.  Not all systems can
be trimmed.

   The implementation in ASSIST 6.2 and higher is as follows.  The user has three
choices.  First, the default (``TRIM=0;'') is no trimming.  The second
choice is Allan's and Dan's trimming bound.  In this mode, the model is
altered such that each state in which the system is experiencing a recovery
has all failure transitions from it that don't go immediately to a death
state going to an intermediate trim state. 
There is a transition from that intermediate trim state to a final state
at rate TRIMOMEGA.  This is invoked by including the statement 
``TRIM = 1 WITH real'' where real is some real number.  For example,
\begin{verbatim}
TRIM = 1 WITH 6e-4;
\end{verbatim}
The expression ``WITH real'' is optional.  If this expression is not included,
the user will be prompted for the value.  In that case, the user is 
requested to type in the value followed by a semicolon (;).  A constant,
TRIMOMEGA, will then be set to whatever value was specified by the user.

     The third choice is Allan's and Dan's trimming bound plus an extra
transition is added from each prune state to another prune state at the
same rate TRIMOMEGA that was used for the trimming.

     The user is, again, referred to the RAMS paper for the justification 
for Allan's and Dan's trimming (TRIM=1) to determine if this
method is valid for his particular class of systems.  
In order to be conservative,
the TRIMOMEGA value should be set to be the sum of the failure rates of
all components in the system.  The justification for the TRIM=2 method
is as follows.  Each prune state of the model is not a death state of the
system.  Thus, another failure of some component must occur before system
failure can occur.  The TRIM=2 feature is simply including this extra
failure transition before pruning the model.  Again, to be conservative,
the TRIMOMEGA must be greater than or equal to the sum of the failure
rates for all components still in the system when any prune state is
reached.

\subsection{Enhancements at version 7.0}

The ASSIST program was
completely re-written in ``C'' at version 7.0.   Many new features were
added to the language.  Among these are:
\begin{itemize}
\item 28 character identifier names (First 12 must be unique for constants).
\item More complete error checking.
\item Inclusion of an ASSERT statement for invariant tests.
\item Two kinds of functions, namely the FUNCTION and IMPLICIT.
\item Set notation on the FOR statement allows greater flexibility.
\item Users can now specify their own wording for INPUT prompts.
\item Users can now specify options as ON or OFF as in ``COMMENT OFF;''.
\item Users can now have ``BOOLEAN'' state-space variables and constants.
      Strong type checking is performed.   The function ``COUNT'' will
      change a Boolean to an integer and the expression ``I $<>$ 0'' will
      change an integer to a Boolean.
\item Several command line options were added.
\item Command line options can be placed in the input file
      with the ``C\_OPTION'' command.
\item Identifier cross reference listings can be generated.
\item The state space may now contain nested parentheses.
\item The DIV and MOD operators perform integer quotient and remainder
      operations.
\item The XOR operator performs the Boolean exclusive-or operation.
\item State-space variables may now be incremented or decremented by one
      in the destination of a TRANTO with a brief syntax as in ``NWP$--$'' and
      ``NS$++$''.
\item Doubly subscripted arrays.
\item Additional built-in functions:
      \begin{itemize}
      \item SUM(list) will compute the sum of all listed parameters.  If arrays
                      are listed, then each element of each array is summed.
                      Rows and/or columns of doubly subscripted arrays can also
                      be summed.
      \item COUNT(list) will count the number of TRUE entries in all listed
                        Boolean parameters.   If arrays are listed, then
                        each TRUE element of each array is counted.
                      Rows and/or columns of doubly subscripted arrays can also
                      be counted.
      \item ANY(list) is the same as (COUNT(list) $>$ 0)
      \item ALL(list) is TRUE when all elements in the list are TRUE.
      \item ABS(X) will compute the absolute value of a number.
      \item GAM(X) will compute the gamma function.
      \item FACT(N) will compute the factorial of ``n''.
      \item COMB(N,K) will compute the combinations of ``n'' things taken ``k'' at a time.
      \item PERM(N,K) will compute the permutations of ``n'' things taken ``k'' at a time.
      \end{itemize}
\item Bit mapping of all state-space variables.
\item Array constants written to model file.   ARR[1,3] is the same as ARR\_1\_3.
\end{itemize}

\subsubsection{Enhancements at version 7.0.1}

This version of ASSIST changes the input file name prompt to:
\begin{codeexample}
Enter ASSIST input file:
\end{codeexample}
and increases the maximum number of state-space variables per nesting level
from 50 to 200.
\subsubsection{Enhancements at version 7.0.2}
This version fixes a bug which showed up when attempting to input the
TRIM option.
\subsubsection{Enhancements at version 7.0.3}
\begin{indenteditems}
\item The CYC operator was added to ASSIST.
This operator is like MOD but performs
the operation $1 + ((x-1)$ MOD $y)$.
\item The SIZE built-in function was added to take the size of an array.
The size of an array is defined to be the number of elements in the array.
For example, the size of an array ARRAY[25..27] is 3 and the size of an
array ARRAY[1..5,1..3] is 5$\times$3 $=$ 15.
\end{indenteditems}

\subsubsection{Enhancements at version 7.0.4}

This version of ASSIST looks up the new state in the hash table
before checking for DEATH, PRUNE, ASSERT conformance.   The idea
is that, if the state already exists, then it has already been
tested for such conformance and found to be non-lumped.   In the
special case of ONEDEATH OFF, the state is flagged as death with
an extra bit and the code will therefore still work correctly.

A model with 9 triads and 5 spares was used to test any speed improvements.
The model used was:
\begin{logfileexample}
(0001): (*  MULTIPLE TRIADS WITH POOL OF SPARES *)
(0002): 
(0003): COMMENT OFF;
(0004): INPUT N_TRIADS;         (* Number of triads initially *)
N_TRIADS? 9;

(0005): INPUT N_SPARES;         (* Number of spares *)
N_SPARES? 5;

(0006): N_PROCS = 3;            (* Number of active processors per triad *)
(0007): LAMBDA_P = 1E-4;        (* Failure rate of active processors *)
(0008): LAMBDA_S = 1E-5;        (* Failure rate of cold spare processors *)
(0009): DELTA1 = 3.6E3;         (* Reconfiguration rate to switch in spare *)
(0010): DELTA2 = 5.1E3;         (* Reconfiguration rate to break up a triad *)
(0011): 
(0012): SPACE = (NP: ARRAY[1..N_TRIADS] OF 0..N_PROCS,
(0013):                              (* Number of processors per triad *)
(0014):          NFP: ARRAY[1..N_TRIADS] OF 0..N_PROCS,
(0015):                              (* Number of failed active processors per triad *)
(0016):          NS,                 (* Number of spare processors *)
(0017):          NFS,                (* Number of failed spare processors *)
(0018):          NT: 0..N_TRIADS);   (* Number of non-failed triads *)
(0019): 
(0020): START = (N_TRIADS OF N_PROCS, N_TRIADS OF 0, N_SPARES, 0, N_TRIADS);
(0021): 
(0022): IF NS > NFS TRANTO NFS++ BY NS*LAMBDA_S;  (* Spare failure *)
(0023): 
(0024): FOR J = 1, N_TRIADS;
(0025):    IF NP[J] > NFP[J] TRANTO NFP[J]++
(0026):         BY (NP[J]-NFP[J])*LAMBDA_P;  (* Active processor failure *)
(0027): 
(0028):    IF NFP[J] > 0 THEN
(0029):       IF NS > 0 THEN
(0030):          IF NS > NFS TRANTO NFP[J]--,NS--
(0031):             BY (1-(NFS/NS))*NFP[J]*DELTA1;
(0032):             (* Replace failed processor with working spare *)
(0033): 
(0034):          IF NFS > 0 TRANTO NS--,NFS-- BY (NFS/NS)*NFP[J]*DELTA1;
(0035):             (* Replace failed processor with failed spare *)
(0036): 
(0037):       ELSE
(0038):          TRANTO NP[J]=0, NFP[J]=0, NS = NP[J]-NFP[J], NT-- BY DELTA2;
(0039):             (* Break up a failed triad when no spares available *)
(0040):       ENDIF;
(0041):    ENDIF;
(0042): 
(0043):    DEATHIF 2 * NFP[J] >= NP[J] AND NP[J] > 0;
(0044):      (* Two faults in an active triad is system failure *)
(0045): 
(0046): ENDFOR;
(0047): 
(0048): DEATHIF NT = 0;    (* System failure by exhaustion *)
\end{logfileexample}
It was run twice on the same CPU, once using ASSIST 7.0.3 and once
using ASSIST 7.0.4.

The results for ASSIST 7.0.3 were as follows:
\begin{center}\begin{tabular}{rl}
             0.74 & sec to parse the input file \\
        11,957.02 & sec to generate the model \\
          125,774 & states \\
        1,311,537 & transitions \\
\end{tabular}\end{center}

The results for ASSIST 7.0.4 were as follows:
\begin{center}\begin{tabular}{rl}
             0.86 & sec to parse the input file \\
         8,287.30 & sec to generate the model \\
          125,774 & states \\
        1,311,537 & transitions \\
\end{tabular}\end{center}

In conclusion, the new version with the hash lookup before the
test for DEATH, PRUNE, ASSERT is faster, even for large models with
a large number of states.

\subsubsection{Enhancements at version 7.0.5}

This version added the following:
\begin{indenteditems}
\item A fix was made to allow the last character in a file to be
      something other than a new-line character.
\item A new-line during a prompt resulting from an INPUT statement is
      now treated as a semicolon terminator unless preceded by a backwards
      slash character.
\item Support was added for state space ranges wider than 256.   For example:
      ``SPACE $=$ (NWP:0..32767);''
\item Some bugs resulting in extraneous/incorrect error messages appearing
      when parsing concatenation expressions were fixed.
\item Some new internals were added in anticipation of future enhancements.
\end{indenteditems}

\subsubsection{Enhancements at version 7.0.6}

This version introduced some internal changes required to support future
program development.   It is functionally equivalent to version 7.0.5.

\subsubsection{Enhancements at version 7.0.7}

This version added the following:
\begin{indenteditems}
\item A fix was made to allow the user to enter a value for the
      constant TRIMOMEGA before the TRIM statement.   The user 
      may now say ``TRIMOMEGA = 5.0*LAMBDA; TRIM ON;'' without
      an error.   At the previous release, the user had to say
      ``TRIM ON WITH 5.0*LAMBDA;''.   Both methods are now accepted.
\item A bug in the expression parser which caused ASSIST to generate error
      messages about integer overflows with complicated expressions was fixed.
\item The new version of ASSIST now differentiates between DEATH and PRUNE
      statements when printing the two warning messages:
      \begin{itemize}
          \item MODEL CONTAINS RECOVERY TRANSITIONS DIRECTLY TO DEATH
                STATE AND THEREFORE MAY NOT BE SUITED TO TRIMMING.
          \item MODEL CONTAINS RECOVERY TRANSITIONS DIRECTLY TO PRUNE
                STATE AND THEREFORE MAY NOT BE SUITED TO TRIMMING.
      \end{itemize}
      The previous version, when detecting a recovery directly to a PRUNE
      state printed the death state message.
\item This version introduced some internal changes required to support future
      program development.   These changes are functionally equivalent to
      the version 7.0.6.
\end{indenteditems}

\subsubsection{Enhancements at version 7.0.8}

This version added the following:
\begin{indenteditems}
\item Internal changes were made to support future products.
\item The counting scheme was modified when the number of transitions
      processed is written to the standard output/error file during
      rule generation.    The new algorithm prints a message every
      100 transitions until 1,000 transitions have been processed
      at which time a message is printed every
      1,000 transitions until 10,000 transitions have been processed
      at which time a message is printed only once every 10,000
      transitions until all transitions have been processed.
\end{indenteditems}

\subsubsection{Enhancements at version 7.0.9}

This version made several fixes.   These are:
\begin{indenteditems}
\item Internal changes were made to support future products.
\item The old version of ASSIST was allowing a trailing comma immediately
      before the closing parenthesis of a SPACE statement.   Although the
      extraneous comma was being ignored, the language description (BNF)
      specifically disallowed the syntax.   The new version prints an
      error message.
\item The VARIABLE statement was added to the language to define variables
      which depend upon the state-space values.   See
      section \ref{sec:VARIABLE} for more details.
\item Incorrect IMPLICIT and FUNCTION expansion was being printed to the
      log file when the ``DEBUG\$ EXPAND\$'' statement was present.   This
      problem was benign since it only effected the log file and not the
      model file.
\item The concatenation character was not working right with certain
      complicated expressions resulting in garbage being written to the
      model file.   The problem was that the language description (BNF)
      in the manual was not specific enough, therefore allowing for a more
      general usage of the concatenation character than was originally
      intended.   It was determined that it would not be very difficult
      to allow the extra syntax, so the ASSIST program was upgraded to
      correspond to the manual.   The old problem arose when the value
      of a built-in function invocation or the value of an array element
      reference followed the concatenation character without being enclosed
      in parentheses.   Concatenation still has high precedence so any
      expressions to be concatenated must still be enclosed in parentheses.
\item The default limits for the maximum number of scratch expressions and
      scratch operands have been tripled.   The user can still control
      these limits via the ASSIST command line.
\end{indenteditems}

\subsubsection{Enhancements at version 7.0.10}

This version made several fixes.   These are:
\begin{indenteditems}
\item Internal changes were made to support future products.
\item The new version will allow the user to use up to 16MB of space for
      parsing an input file.   The old version would allow use of only
      1MB of space.   This was of consequence only for input files with
      some very very long expressions or some very big storage requirements
      requested with the command line options.
\item The old version of the parser was allowing a VARIABLE definition
      to specify calling parameters even though calling parameters
      make no sense with variables and were being ignored during rule
      generation.   The new version prints an error message when an
      input file attempts this.
\item A bug in the expression parser which caused ASSIST to generate error
      messages about space variables out of range with complicated
      expressions was fixed.
\item During rule generation, when a state-space value goes out of bounds
      due to a transition to an invalid state,
      an error message is printed.   In addition, the invalid state was
      supposed to have been treated as a death state.   The new version
      correctly treats the destination state as a death state.   The
      model file has always been and still is being flagged as erroneous
      when an invalid state is encountered.
\item The MIN and MAX built-in list/array functions were added.   Like SUM,
      these two functions operate on lists and/or arrays of integers and/or
      reals and return a result of the same type as being passed.
\item The old version was complaining about real numbers being invalid
      in constant array definitions with the ``n OF'' syntax.   For example,
      ``ARR = [ 2.2, 5 OF 1.1 ];'', was generating an error message that
      the real number 1.1 was not allowed except in a rate expression.   The
      problem is now fixed and the syntax is allowed.
\end{indenteditems}

\subsubsection{Enhancements at version 7.0.11}

This version made several fixes.   These are:
\begin{indenteditems}
\item Internal changes were made to support future products.
\item A new command line option ``-auto'' is used to automatically insert
      both a ``RUN'' and an ``EXIT'' command at the end of the model file
      output.   It can be used together with ``-pipe'' when piping output
      to SURE.
\end{indenteditems}

\subsubsection{Enhancements at version 7.0.12}

This version made one minor fix:
\begin{indenteditems}
\item The old version was computing zero for 2**N where N $>$$=$ 31.   The
      new version still computes zero but prints an error message
      that an overflow has occurred.   The model file is flagged as
      erroneous so that it cannot be solved.
\end{indenteditems}

\subsubsection{Enhancements at version 7.0.13}

This version made the following minor fixes:
\begin{indenteditems}
\item The new version will compile without warnings with
      version 2.1 of the ``gcc'' compiler.
\item The new version uses release number 7.0-13 under VMS.
\item The new version adds a test for command line memory option
      values which must be multiples of four.   This test is made
      for all architectures with either 16 or 32 bit integers.
\end{indenteditems}

\subsubsection{Enhancements at version 7.0.14}

This version made the following minor fixes:
\begin{indenteditems}
\item Certain Boolean expressions with embedded comments were being reported
      as being in error with a message such
      as ``[ERROR] RELATIONAL OPERATOR MUST FOLLOW NUMERIC
      QUANTITY IN BOOLEAN EXPRESSION''.   The new version of \acronym{ASSIST}
      correctly handles all comments including those embedded in Boolean
      expressions.   Some such expressions
      are being generated with \acronym{TOTAL}.
\item The command line option
      \option{WRAP\_LONG\_CONSTANT\_EXPRESSIONS}
      (\option{WRAP\_LONG} for short)
      was added (also with the ``C\_OPTION WRAP\_LONG\_CONSTANT\_EXPRESSIONS;''
      command) in order to tell \acronym{ASSIST} to write each term of constant
      expressions on a separate line so that \acronym{SURE}
      will be able to read in the model.   Such long expressions for TRIMOMEGA
      are being generated by \acronym{TOTAL}.
\item Longer and more descriptive warning messages are allowed in order to
      support \acronym{TOTAL}.
\item The line number of the transition used to get to the current state
      is now printed when assertions fail.
\item Other minor internal changes were made to support \acronym{TOTAL}.
\end{indenteditems}

\subsubsection{VARIABLE Statement}
\label{sec:VARIABLE}

The \rw{VARIABLE} statement is used to define a variable which is not part
of but is dependent upon the state space.   The variable can be either a
scalar or an array.

Variables oftentimes produce the same results as do implicits.   The difference
is when the expression evaluation is performed.   With an \rw{IMPLICIT}, the
expression to the right of the equal sign is not evaluated until it is
actually used.   With a \rw{VARIABLE} the expression is evaluated once for
each new state regardless of whether or not the variable is ever referenced.
There is therefore a performance trade-off.   If the expression is going
to be referenced inside of a loop or many different times, a \rw{VARIABLE}
will probably be more efficient than an \rw{IMPLICIT}.   If, however, the
references to the expression are all protected by \rw{IF} tests which
hold true for very few states, then an \rw{IMPLICIT} may be more efficient
than a \rw{VARIABLE}.

Another difference is that a \rw{VARIABLE} can define an array whereas
an \rw{IMPLICIT} cannot.

For example, if \word{NWP} is a state-space variable representing the number
of working processors and \word{NI} is a constant denoting the number of
processors initially, then the declaration:
\begin{codeexample}
VARIABLE NFP[NWP] = NI - NWP;
\end{codeexample}
defines \word{NFP}, denoting the number of failed processors, to
be the difference between the number initially and the current
number of working processors.  The above variable can be
referenced as illustrated in the following \rw{DEATHIF} statement:
\begin{codeexample}
DEATHIF NWP <= NFP;
\end{codeexample}

The syntax is:
\begin{syntaxexample}
\literal{VARIABLE} \bnf{identifier} \literal{[} \bnf{state-space-variable-list} \literal{]} \literal{=} \maybeonce{\literal{BOOLEAN}} \bnf{definition-clause} \literal{;}
\end{syntaxexample}
where a definition clause defines either a scalar or an array.  Its syntax is:
\begin{syntaxexample}
\bnf{expr}
\orline{3mm}
\literal{[} \maybeonce{\bnf{\#} \literal{OF}} \bnf{expr} \manytimes{\literal{,} \maybeonce{\bnf{\#} \literal{OF}} \bnf{expr}} \literal{]}
\orline{3mm}
\literal{ARRAY (} \maybeonce{\bnf{\#} \literal{OF}} \bnf{expr} \manytimes{\literal{,} \maybeonce{\bnf{\#} \literal{OF}} \bnf{expr}} \literal{)}
\end{syntaxexample}


The
\bnf{state-space-variable-list}
is made up of one or more state-space
variable name identifiers separated by commas.
The identifiers must already have been defined in a \rw{SPACE}
statement.
All state-space variables that are referenced in the
definition clause of an \rw{VARIABLE} must be listed in the state-space variable list.

The \bnf{definition-clause} of the \rw{VARIABLE} definition is the expression defining
the \rw{VARIABLE} as either a variable or an array which is dependent upon the state space.
Named constants and reserved words may be freely referenced to the right of the equal sign
and must not be specified with the state-space variables.

The \rw{VARIABLE} may be referenced
in \rw{TRANTO}, \rw{DEATHIF},
\rw{PRUNEIF}, \rw{ASSERT}, \rw{IF}, or \rw{FOR}
statements or in later \rw{VARIABLE}, \rw{IMPLICIT}, or \rw{FUNCTION} definitions in
exactly the same manner as a constant would be referenced.
The \rw{VARIABLE} is referenced in the same way as a constant would be.  Square brackets
are used when the \rw{VARIABLE} is an array.

The following example
illustrates the definition and use of \rw{VARIABLE}s:
\begin{codeexample}
SPACE = (NWP:ARRAY[1..5] OF 0..3); (* Number working in each of five triads *)

VARIABLE NFP[NWP] = [
                       3 - NWP[1],
                       3 - NWP[2],
                       3 - NWP[3]
                    ];    (* Number failed in each triad *)
VARIABLE TOTALW[NWP] = SUM(NWP);
VARIABLE TOTALF[NWP] = 3*5-TOTALW;

FOR III IN [1..5]
   IF (NFP[III]>0) TRANTO NWP[III]=0 BY 1.E-5;
ENDFOR;
DEATHIF TOTALF > 6;
\end{codeexample}
In the above example, the variable named \word{NFP} is declared
to be dependent upon the state-space variable \word{NWP} and to be an array
with three elements.   The value for each element is
computed as three minus the current number of
working processors in the triad corresponding to the array index.
The variable \word{TOTALW} is also dependent upon the state-space
variable \word{NWP}.   Its value is
computed as the sum of the number of working processors in each triad
over all five triads.   The variable \word{TOTALF} is also
a dependent upon the state-space variable \word{NWP}.
The definition of \word{TOTALF} references the value of variable
\word{TOTALW}.   Since \word{TOTALW} is computed based upon the state-space
variable \word{NWP}, then \word{NWP} must be listed in the state-space
variable list for \word{TOTALF} even though it is only referenced
indirectly.   When \word{NFP} is referenced in the \rw{IF}, the value of
\word{III} is passed as a subscript to satisfy the array requirements
as declared in the \word{VARIABLE NFP} line.


\subsection{General Comments}
 
Hopefully, you will find that \acronym{ASSIST} versions 7.0 and above 
generate models considerably faster, use much less physical memory, 
and thus allow the generation of larger models than previous versions.
Several model reduction techniques have been
developed for use with the \acronym{ASSIST} program.  Since the user can reduce the
model before it is generated by \acronym{ASSIST}, considerably larger and more complex
systems can now be handled.  The use of these techniques, including the new
\verb'PRUNEIF' statement, are described in the paper ``Reliability
Analysis of Large, Complex Systems Using ASSIST,'' Proceedings $8^{th}$ Digital
Avionics Systems Conference, October 1988.

\section{Warning on associativity of exponentiation}

Please note that the associativity of the exponentiation operator was
inconsistant between versions of SURE/STEM/PAWS prior to version 7.9.8 and
FTC version 2.8.2 and the new version of ASSIST.   Users of ASSIST 7.0
and higher should use SURE/STEM/PAWS 7.9.8 or higher and FTC 2.8.2 or
higher.

The prior releases of SURE/STEM/PAWS/FTC used left-to-right
associativity whereas the new versions of these programs and ASSIST 7.0 use
the more common right-to-left associativity.   For example, consider:
\begin{seta}
    VAL = 1.1 ** 1.883 ** 1.448;
\end{seta}
In the prior releases of SURE/STEM/PAWS/FTC, this was equivalent to:
\begin{seta}
    VAL = (1.1 ** 1.883) ** 1.448;
\end{seta}
Whereas in ASSIST 7.0, and the version 7.9.8 of SURE/STEM/PAWS/FTC it
is equivalent to:
\begin{seta}
    VAL = 1.1 ** (1.883 ** 1.448);
\end{seta}

