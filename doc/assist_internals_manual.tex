\documentstyle [12pt,alltt]{article}
\setlength{\textwidth}{6.75in}
\setlength{\textheight}{9.0in}
\setlength{\topmargin}{0.0in}
\setlength{\oddsidemargin}{0.25in}
\setlength{\evensidemargin}{0.25in}
\setlength{\parindent}{0.0in}
\setlength{\parskip}{6 pt}
\newcounter{boxcenx}        \newcounter{boxceny}
\newcounter{arrowheadx}     \newcounter{arrowheady}
\newcounter{savAx}          \newcounter{savAy}
\newcounter{savBx}          \newcounter{savBy}
\newcounter{savCx}          \newcounter{savCy}
\newcounter{savDx}          \newcounter{savDy}
\newcounter{savEx}          \newcounter{savEy}
\newcounter{savFx}          \newcounter{savFy}
\newcounter{savGx}          \newcounter{savGy}
\newcounter{savHx}          \newcounter{savHy}
\newcounter{tempx}          \newcounter{tempy}
\newcounter{tempw}          \newcounter{temph}
\newcounter{fallw}          \newcounter{fallh}
\newcounter{nextboxx}       \newcounter{nextboxy}
\newcounter{dflthalfbox}    \newcounter{dfltfulltag}
\newcounter{dfltheight}     \newcounter{negdfltheight}
\newcounter{dflthheight}    \newcounter{negdflthheight}
\newcounter{dfltradius}     \newcounter{negdfltradius}
\newcounter{runx}           \newcounter{runy}
%=========================================================================
\newcommand{\atsign}{\char64}
\newcommand{\dollarsign}{\char36}
\newcommand{\itemtext}[1]{{\small $\Longrightarrow$ \vspace{4pt}} \\ {\sf {#1}}}
\newcommand{\anywhat}[1]{\underline{\em {#1}}~}
\newcommand{\anylinnum}{\anywhat{linenumber}}
\newcommand{\anyid}{\anywhat{id}}
\newcommand{\anytok}{\anywhat{token}}
\newcommand{\anystatenum}{\anywhat{statenumber}}
\newcommand{\anyident}{\anywhat{id}}
\newcommand{\anyopt}{\anywhat{opt}}
\newcommand{\anymsg}{\anywhat{message}}
\newcommand{\anyarray}{\anywhat{arrayname}}
\newcommand{\anyfunc}{\anywhat{func}}
\newcommand{\anybin}{\anywhat{func}}
\newcommand{\anyimpl}{\anywhat{impl}}
\newcommand{\anymin}{\anywhat{min}}
\newcommand{\anymax}{\anywhat{max}}
\newcommand{\anyint}{\anywhat{int}}
\newcommand{\anyreal}{\anywhat{real}}
\newcommand{\anynum}{\anywhat{num}}
\newcommand{\grammar}[1]{{\apdixfootsiz\sf $<${#1}$>$}}
\newcommand{\startfig}{\begin{figure}[htbp]\begin{center}}
\newcommand{\finishfig}[2]{\end{center}\caption{#1}\label{#2}\end{figure}}
\newcommand{\starttab}{\begin{table}[htbp]\begin{center}}
\newcommand{\finishtab}[2]{\end{center}\caption{#1}\label{#2}\end{table}}
%.......\newenvironment{safefigure}[2]
%.......   {\begin{figure}[htbp]\begin{center}}{
%.......    \end{center}\caption{#1}\label{#1}\end{figure}}
%.......\newenvironment{safetable}[2]
%.......   {\begin{table}[htbp]\begin{center}}{
%.......    \end{center}\caption{#1}\label{#1}\end{table}}
\newcommand{\subsubsubsection}[1]
  {\vspace{-2mm}\begin{bf}\begin{tabbing}
   {\addcontentsline{toc}{subsubsection}{~~~~~~~~~~#1}}{#1}\vspace{-8mm}
   \end{tabbing}\end{bf}}
\newcommand{\subsubsubsubsection}[1]
  {\vspace{-2mm}\begin{bf}\begin{tabbing}
   {#1}\vspace{-8mm}
   \end{tabbing}\end{bf}}
   
\newcommand{\qovalerror}
 {\begin{picture}(40,20)\thicklines
  \put(20,10){\oval(40,20)}
  \put(20,10){\makebox(0,0)[lb]{\elvrm \large{error}}}
  \end{picture}}
\newcommand{\qboxerror}{\fbox{\large{error}}}
%=========================================================================
%\newenvironment
%   {typedef_tabbing}
%   {\begin{footnotesize}\begin{tt}\begin{tabbing}
%    wwwww \= www \= wwww \= \kill}{\end{tabbing}\end{tt}\end{footnotesize}}
\newcommand{\orline}[1]{\vspace{2mm}\hspace{#1}{\scriptsize or}\vspace{2mm}}
\newenvironment{codeexample}
{\par \small \setlength{\baselineskip}{.14in} 
 \begin{list}{}{\setlength{\leftmargin}{3em}} \item
 \begin{alltt} \index{examples}}{
 \end{alltt} \end{list} \par  \normalsize \setlength{\baselineskip}{.19in}}
\newenvironment{logfileexample}
{\par \footnotesize \setlength{\baselineskip}{.14in} 
 \begin{list}{}{\setlength{\leftmargin}{3em}} \item
 \begin{alltt} \index{examples}}{
 \end{alltt} \end{list} \par  \normalsize \setlength{\baselineskip}{.19in}}
\newenvironment{SmallPrintLogfileexample}
{\par \scriptsize \setlength{\baselineskip}{.14in} 
 \begin{list}{}{\setlength{\leftmargin}{3em}} \item
 \begin{alltt} \index{examples}}{
 \end{alltt} \end{list} \par  \normalsize \setlength{\baselineskip}{.19in}}
\newenvironment{syntaxexample}
{\par \normalsize \setlength{\baselineskip}{.14in} 
 \begin{list}{}{\setlength{\leftmargin}{3em}} \item
 \begin{alltt} \index{syntax}}{
 \end{alltt} \end{list} \par  \normalsize \setlength{\baselineskip}{.19in}}
\newenvironment{indenteditems}
{\par \small \setlength{\baselineskip}{.14in} 
 \begin{list}{}{\setlength{\leftmargin}{3em}}}{
 \end{list} \par  \normalsize \setlength{\baselineskip}{.19in}}
\newenvironment
   {typedef_tabbing}
   {\begin{small}\begin{tt}\begin{tabbing}
    wwwww \= www \= wwww \= \kill}{\end{tabbing}\end{tt}\end{small}}
\newenvironment
   {fast_picture}[2]
   {\begin{center}\begin{picture}(#1,#2)
    \fastpicdefaults}{\end{picture}\end{center}}
\newenvironment{bnf_tabbing}
   {\begin{tabbing}\bnftabbing}{\end{tabbing}}
%=========================================================================
%
%   For referencing reserved words within the text of a paragraph:
%
%       \inerror    is for emphasizing text before an errorneous example.
%       \rw         is for reserved words such as "DEATHIF", "TRANTO", etc.
%       \word       is for inline words such as "LAMBDA = 2", etc.
%       \acronym    is for acronyms such as "SURE", "ASSIST", "TOTAL", etc.
%       \extent     is for filename extents such as ".alog", ".ast", etc.
%       \unixoption is for options such as -XREF, etc.  Omit the minus.
%       \vmsoption  is for options such as /XREF, etc.  Omit the slash.
%
\newcommand{\inerror}[1]{\begin{bf}\begin{em}{#1}\end{em}\end{bf}}
\newcommand{\rw}[1]{\begin{tt}{#1}\end{tt}}
\newcommand{\Xrw}[1]{\begin{tt}{#1}~~\end{tt}}
\newcommand{\word}[1]{\begin{tt}{#1}\end{tt}}
\newcommand{\Xword}[1]{\begin{tt}{#1}~~\end{tt}}
\newcommand{\acronym}[1]{\begin{sc}{#1}\end{sc}}
\newcommand{\Xacronym}[1]{\begin{sc}{#1}~~\end{sc}}
\newcommand{\extent}[1]{\begin{em}{``#1''}\end{em}}
\newcommand{\unixoption}[1]
    {\begin{bf}\begin{boldmath} $-$#1 \end{boldmath}\end{bf}}
\newcommand{\vmsoption}[1]
    {\begin{sc}\begin{bf}\begin{boldmath} $/$#1 \end{boldmath}\end{bf}\end{sc}}
\newcommand{\unixverboption}[1]{\(-\)#1}
\newcommand{\vmsverboption}[1]{\(/\)#1}
\newcommand{\manytimes}[1]
    {{\raisebox{-.5ex}{\LARGE \{}}
     {#1}
     {\raisebox{-.5ex}{\LARGE \}}}}
\newcommand{\maybeonce}[1]
    {{\raisebox{-.5ex}{\LARGE [}}
     {#1}
     {\raisebox{-.5ex}{\LARGE ]}}}
%    {{\raisebox{-.8ex}{\Huge \(\sqsubset\)}}
%     {#1}
%     {\raisebox{-.8ex}{\Huge \(\sqsupset\)}}}
\newcommand{\smilingface}
{\raisebox{-0.8ex}{$\stackrel{\raisebox{-0.8ex}{$\stackrel{.~.}{.}$}}{\smile}$}}
%     {{\tiny \( \begin{array}{c} .. \\ \cdot \\ \smile \\ \end{array} \) }}
%
\newcommand{\bnf}[1]{\begin{em}\begin{footnotesize}{\(<\)#1\(>\)}\end{footnotesize}\end{em}}
\newcommand{\unixliteral}[1]
    {\begin{bf}\begin{boldmath}{#1}\end{boldmath}\end{bf}}
\newcommand{\vmsliteral}[1]
    {\begin{sc}\begin{bf}\begin{boldmath}{#1}\end{boldmath}\end{bf}\end{sc}}
%
%   For quoting single line statements within the text of a paragraph:
%
\newcommand{\ASSIST}[1]
   {\begin{typedef_tabbing} \2 {#1} \\ \end{typedef_tabbing}}
%=========================================================================
\newcommand{\qqon}{}
\newcommand{\qqoff}{}
\newcommand{\qqtiny}[1]{\mbox{\tiny\rm #1}}
\newcommand{\qqsmall}[1]{\mbox{\scriptsize\sf #1}}
\newcommand{\bnftabbing}{WWWWWWWWWWWW \= WW \= WWWW \= \kill}
\newcommand{\blockiftabbing}{WWWWWWWWWWWW \= WW \= IF \= \kill}
\newcommand{\implicittabbing}{WWWWWWWWWWWW \= WW \= IMPLICIT \= \kill}
\newcommand{\forlooptabbing}{WWWWWWWWWWWW \= WW \= FORW \= \kill}
\newcommand{\builtintabbing}
    {WWWWWWWWWWWW \= WW \= WWWWWW \= WW \= WWWWWW \= WW \= WWWWWW \kill}
\newcommand{\digittabbing}
    {WWWWWWWWWWWW \= WW \= {\bf W} \= $|$~{\bf W} \= $|$~{\bf W}
                    \= $|$~{\bf W} \= $|$~{\bf W} \kill}
\newcommand{\foottabbing}{WWW \= \kill}
\newcommand{\1}{\>}
\newcommand{\2}{\> \>}
\newcommand{\3}{\> \> \>}
%=========================================================================
%
%  Eight registers are defined for use with these new commands:
%
%         A,B,C,D,E,F,G,H
%
%  Registers must be specified with a capital letter.   A capital "R"
%  in the description of the name of a command denotes a register.   For
%  example:
%
%        \savRboxpos
%
%  stands for any of the following commands:
%
%        \savAboxpos
%        \savBboxpos
%        \savCboxpos
%        ...
%        \savHboxpos
%
%  Description of new commands which are valid for the "fast_picture"
%  environment:
%
%      \begin{fast_picture} .... \end{fast_picture}
%
%             new command syntax                      description
%      --------------------------------    ---------------------------------
%      \apdixsection{text}                 Defines a new section within an
%                                          appendix.   The specified text
%                                          is printed in the document.
%      \setboxpos{x}{y}                    Sets the lower left hand position
%                                          of the current box.
%      \savRboxpos                         Saves the lower left hand corner
%                                          of the current box in a register.
%      \setboxsizes{half-wid}{full-tag}    Sets the default widths for the
%                                          size of a new box.  Note that
%                                          only half the width of the box
%                                          is specified whereas the full
%                                          width of the tag field must be
%                                          specified.
%      \fastpicdefaults                    Resets all defaults to their
%                                          original value.
%      \calcboxcen{left}{bot}{half-wid}    Calculates the center of the
%                                          box and stores the result in the
%                                          box center register.
%      --------------------------------    ---------------------------------
%      \myowndot                           Not intended for general use.
%                                          Specifies the size of the dot
%                                          at the beginning of all arrows.
%      --------------------------------    ---------------------------------
%      \tempdoublewide                     Not intended for general use.
%      \rtarrow{x}{y}{leng}                Not intended for general use.
%      \rtarrowdown{x}{y}{leng}{fall}      Not intended for general use.
%      \rtarrowxdown{x}{y}{l}{f}{l}{f}     Not intended for general use.
%      \qlblnobox{x}{y}{msg}{half-wid}{h}  Not intended for general use.
%      \qvalbox{x}{y}{msg}{half-wid}{h}    Not intended for general use.
%      \qleftlblbox{x}{y}{tag-msg}         Not intended for general use.
%       {box-msg}{tag-wid}{half-boxwid}{h}
%      \qrightlblbox{x}{y}{box-msg}        Not intended for general use.
%       {tag-msg}{half-boxwid}{tag-wid}{h}
%      \qground{x}{y}{hw}                  Not intended for general use.
%      --------------------------------    ---------------------------------
%      \putboxafterarrow                   Positions the next box to the
%                                          right of the arrowhead.
%      \wideboxtoright{half-wid}           Positions boxes left-to-right
%      \boxtoright                         Positions boxes left-to-right
%      \groundunderarrow                   Puts the ground (null pointer)
%                                          symbol under the arrowhead.
%      \putwideboxunderarrow{half-wid}     Puts the next box centered under
%                                          the most recently drawn arrowhead.
%                                          Called by stackwideRbox and
%                                          intended for use with the wide
%                                          commands.
%      \putboxunderarrow                   Puts the next box centered under
%                                          the most recently drawn arrowhead.
%                                          Called by stackRbox and intended
%                                          for use with the commands that
%                                          refer to the default box size.
%      \putuponebox                        Not intended for general use.
%      \putdownonebox                      Not intended for general use.
%      \stackwideRbox{half-wid}            Begins a stack of boxes under
%                                          the most recently drawn arrowhead.
%                                          The box stack position is computed
%                                          and saved in the specified
%                                          register so that \nextRbox can
%                                          be used in series.  Half the
%                                          width of the box must be specified.
%      \stackRbox                          Like \stackwideRbox except that
%                                          half the width of the default
%                                          box size is used.
%      \nextRbox                           Computes the position of the next
%                                          box based upon the previous box
%                                          drawn using the given register.
%      \leftwidetagbox{tag-msg}{box-msg}   Draws a wide box with a tag message
%           {full-tag-wid}{half-box-wid}   off to the left.   The box is
%                                          drawn with the lower left hand
%                                          corner at the current box position
%                                          as set with \nextRbox.
%      \rightwidetagbox{box-msg}{tag-msg}  Like \leftwidetagbox except that
%            {half-box-wid}{full-tag-wid}  the tag message if off to the right.
%      \lefttagbox{tag-msg}{box-msg}       Like \leftwidetagbox except that
%                                          the default values for the box and
%                                          tag sizes are used.
%      \righttagbox{box-msg}{tag-msg}      Like \lefttagbox except that the
%                                          tag message if off to the right.
%      \widevalbox{msg}{half-wid}          Draws a wide box and places the
%                                          message text centered inside it.
%                                          The message text and half the width
%                                          of the box must be specified.
%      \valbox{msg}                        Like \widevalbox except that the
%                                          default values for the box sizes
%                                          are used.
%      --------------------------------    ---------------------------------
%      \outarrow{leng}                     Draws an arrow pointed right out
%                                          of the center of the current
%                                          (most recently rendered) box.
%                                          The length is specified in
%                                          the default units (usually pts).
%      \outarrowmsg{leng}{msg}             Like \outarrow but specifies
%                                          message text to place to the
%                                          right of the arrowhead.
%      \outarrowdown{leng}{fall}           Draws an arrow to the right and
%                                          then pointed down beginning at
%                                          the center of the current box.
%                                          The length of the horizontal (leng)
%                                          and vertical (fall) line segments
%                                          must be specified.
%      \outarrowdownrightdown              Draws an arrow which points to
%          {leng}{fall}{leng}{fall}        the right out of the center of the
%                                          current box and then falls and
%                                          then continues to the right and
%                                          then falls again.
%      \outarrowdownleftdown               Draws an arrow which points to
%          {leng}{fall}{leng}{fall}        the right out of the center of the
%                                          current box and then falls and
%                                          then continues to the left and
%                                          then falls again.
%      \outarrowdownleftdownright          Draws an arrow which points to
%          {leng}{fall}{leng}{fall}{leng}  the right out of the center of the
%                                          current box and then falls and
%                                          then continues to the left and
%                                          then falls again then points right.
%      --------------------------------    ---------------------------------
%
%
%
%=========================================================================
\newcommand{\setrun}[2]{\setcounter{runx}{#1}\setcounter{runy}{#2}}
\newcommand{\runleft}[1]{\addtocounter{runx}{-#1}}
\newcommand{\runright}[1]{\addtocounter{runx}{#1}}
\newcommand{\runup}[1]{\addtocounter{runy}{#1}}
\newcommand{\rundown}[1]{\addtocounter{runy}{-#1}}
\newcommand{\runradleft}{\runleft{\value{dfltradius}}}
\newcommand{\runradright}{\runright{\value{dfltradius}}}
\newcommand{\runradup}{\runup{\value{dfltradius}}}
\newcommand{\runraddown}{\rundown{\value{dfltradius}}}
\newcommand{\runtoarrow}
   {\setcounter{arrowheadx}{\value{runx}}
    \setcounter{arrowheady}{\value{runy}}}
\newcommand{\putatarrowhead}[1]
   {\put(\value{arrowheadx},\value{arrowheady}){#1}}
\newcommand{\newpara}{\newline \vspace{-2mm} \newline}
\newcommand{\startappendix}{\appendix \footnotesize}
\newcommand{\apdixsection}[1]{\subsection{#1}}
\newcommand{\apdixfootsiz}{\scriptsize}
\newcommand{\oldapdixsection}[1]{\begin{Large}\begin{sf}
    \vspace{7mm} \noindent \underline{#1} \newline \vspace{3mm} \newline
    \end{sf}\end{Large}}
\newcommand{\setboxpos}[2]{\setcounter{nextboxx}{#1}\setcounter{nextboxy}{#2}}
\newcommand{\setarrow}[2]{\setcounter{arrowheadx}{#1}\setcounter{arrowheady}{#2}}
%
%  Graphics commands to save/restore arrowhead position
%
%\newcommand{\savCarrowhead}{\setcounter{savCx}{\value{arrowheadx}}
%                            \setcounter{savCy}{\value{arrowheady}}}
%\newcommand{\unsavCarrowhead}{\setcounter{arrowheadx}{\value{savCx}}
%                              \setcounter{arrowheady}{\value{savCy}}}
%\newcommand{\savAarrowhead}{\setcounter{savAx}{\value{arrowheadx}}
%                            \setcounter{savAy}{\value{arrowheady}}}
%\newcommand{\unsavAarrowhead}{\setcounter{arrowheadx}{\value{savAx}}
%                              \setcounter{arrowheady}{\value{savAy}}}
%\newcommand{\savBarrowhead}{\setcounter{savBx}{\value{arrowheadx}}
%                            \setcounter{savBy}{\value{arrowheady}}}
%\newcommand{\unsavBarrowhead}{\setcounter{arrowheadx}{\value{savBx}}
%                              \setcounter{arrowheady}{\value{savBy}}}
%
%  Graphics commands to save/restore next-box position
%
\newcommand{\savAboxpos}{\setcounter{savAx}{\value{nextboxx}}
                         \setcounter{savAy}{\value{nextboxy}}}
\newcommand{\unsavAboxpos}{\setcounter{nextboxx}{\value{savAx}}
                           \setcounter{nextboxy}{\value{savAy}}}
\newcommand{\savBboxpos}{\setcounter{savBx}{\value{nextboxx}}
                         \setcounter{savBy}{\value{nextboxy}}}
\newcommand{\unsavBboxpos}{\setcounter{nextboxx}{\value{savBx}}
                           \setcounter{nextboxy}{\value{savBy}}}
\newcommand{\savCboxpos}{\setcounter{savCx}{\value{nextboxx}}
                         \setcounter{savCy}{\value{nextboxy}}}
\newcommand{\unsavCboxpos}{\setcounter{nextboxx}{\value{savCx}}
                           \setcounter{nextboxy}{\value{savCy}}}
\newcommand{\savDboxpos}{\setcounter{savDx}{\value{nextboxx}}
                         \setcounter{savDy}{\value{nextboxy}}}
\newcommand{\unsavDboxpos}{\setcounter{nextboxx}{\value{savDx}}
                           \setcounter{nextboxy}{\value{savDy}}}
\newcommand{\savEboxpos}{\setcounter{savEx}{\value{nextboxx}}
                         \setcounter{savEy}{\value{nextboxy}}}
\newcommand{\unsavEboxpos}{\setcounter{nextboxx}{\value{savEx}}
                           \setcounter{nextboxy}{\value{savEy}}}
\newcommand{\savFboxpos}{\setcounter{savFx}{\value{nextboxx}}
                         \setcounter{savFy}{\value{nextboxy}}}
\newcommand{\unsavFboxpos}{\setcounter{nextboxx}{\value{savFx}}
                           \setcounter{nextboxy}{\value{savFy}}}
\newcommand{\savGboxpos}{\setcounter{savGx}{\value{nextboxx}}
                         \setcounter{savGy}{\value{nextboxy}}}
\newcommand{\unsavGboxpos}{\setcounter{nextboxx}{\value{savGx}}
                           \setcounter{nextboxy}{\value{savGy}}}
\newcommand{\savHboxpos}{\setcounter{savHx}{\value{nextboxx}}
                         \setcounter{savHy}{\value{nextboxy}}}
\newcommand{\unsavHboxpos}{\setcounter{nextboxx}{\value{savHx}}
                           \setcounter{nextboxy}{\value{savHy}}}
%
%  Graphics commands to save/restore arrow position
%
\newcommand{\savAarwpos}{\setcounter{savAx}{\value{arrowheadx}}
                         \setcounter{savAy}{\value{arrowheady}}}
\newcommand{\unsavAarwpos}{\setcounter{arrowheadx}{\value{savAx}}
                           \setcounter{arrowheady}{\value{savAy}}}
\newcommand{\savBarwpos}{\setcounter{savBx}{\value{arrowheadx}}
                         \setcounter{savBy}{\value{arrowheady}}}
\newcommand{\unsavBarwpos}{\setcounter{arrowheadx}{\value{savBx}}
                           \setcounter{arrowheady}{\value{savBy}}}
\newcommand{\savCarwpos}{\setcounter{savCx}{\value{arrowheadx}}
                         \setcounter{savCy}{\value{arrowheady}}}
\newcommand{\unsavCarwpos}{\setcounter{arrowheadx}{\value{savCx}}
                           \setcounter{arrowheady}{\value{savCy}}}
\newcommand{\savDarwpos}{\setcounter{savDx}{\value{arrowheadx}}
                         \setcounter{savDy}{\value{arrowheady}}}
\newcommand{\unsavDarwpos}{\setcounter{arrowheadx}{\value{savDx}}
                           \setcounter{arrowheady}{\value{savDy}}}
\newcommand{\savEarwpos}{\setcounter{savEx}{\value{arrowheadx}}
                         \setcounter{savEy}{\value{arrowheady}}}
\newcommand{\unsavEarwpos}{\setcounter{arrowheadx}{\value{savEx}}
                           \setcounter{arrowheady}{\value{savEy}}}
\newcommand{\savFarwpos}{\setcounter{savFx}{\value{arrowheadx}}
                         \setcounter{savFy}{\value{arrowheady}}}
\newcommand{\unsavFarwpos}{\setcounter{arrowheadx}{\value{savFx}}
                           \setcounter{arrowheady}{\value{savFy}}}
\newcommand{\savGarwpos}{\setcounter{savGx}{\value{arrowheadx}}
                         \setcounter{savGy}{\value{arrowheady}}}
\newcommand{\unsavGarwpos}{\setcounter{arrowheadx}{\value{savGx}}
                           \setcounter{arrowheady}{\value{savGy}}}
\newcommand{\savHarwpos}{\setcounter{savHx}{\value{arrowheadx}}
                         \setcounter{savHy}{\value{arrowheady}}}
\newcommand{\unsavHarwpos}{\setcounter{arrowheadx}{\value{savHx}}
                           \setcounter{arrowheady}{\value{savHy}}}
%
%  Miscellaneous graphics calculations
%
\newcommand{\setradius}[1]
   {\setcounter{dfltradius}{#1}\setcounter{negdfltradius}{-#1}}
\newcommand{\setboxheight}[1]
   {\setcounter{dflthheight}{#1}\setcounter{negdflthheight}{-#1}
    \setcounter{dfltheight}{#1}\setcounter{negdfltheight}{-#1}
    \addtocounter{dfltheight}{#1}\addtocounter{negdfltheight}{-#1}}
\newcommand{\setboxsizes}[2]
   {\setcounter{dflthalfbox}{#1}\setcounter{dfltfulltag}{#2}}
\newcommand{\fastpicdefaults}
   {\setboxsizes{9}{72}\setboxheight{10}\setradius{12}}
\newcommand{\calcboxcen}[3]
   {\setcounter{boxcenx}{#1}\setcounter{boxceny}{#2}
    \addtocounter{boxcenx}{#3}\addtocounter{boxceny}{\value{dflthheight}}}
\newcommand{\recalcboxcen}[1]
  {\setcounter{boxcenx}{\value{nextboxx}}\setcounter{boxceny}{\value{nextboxy}}
   \addtocounter{boxcenx}{#1}\addtocounter{boxceny}{\value{dflthheight}}}
\newcommand{\tempdoublewide}[1]{\setcounter{tempw}{#1}\addtocounter{tempw}{#1}}
\newcommand{\tempdoublehigh}[1]{\setcounter{temph}{#1}\addtocounter{temph}{#1}}
%
%  Low level graphics commands
%
\newcommand{\myowndot}{\circle*{4}}
\newcommand{\boxcendot}{\put(\value{boxcenx},\value{boxceny}){\myowndot}}
\newcommand{\doton}{\renewcommand{\myowndot}{\circle*{4}}}
\newcommand{\dotoff}{\renewcommand{\myowndot}{}}
\newcommand{\dparrow}[3]
   {\put(#1,#2){\vector(0,-1){#3}}\put(#1,#2){\myowndot}
    \setcounter{arrowheadx}{#1}\setcounter{arrowheady}{#2}
    \addtocounter{arrowheady}{-#3}}
\newcommand{\dparrowrt}[4]
   {\put(#1,#2){\line(0,-1){#3}}\put(#1,#2){\myowndot}
    \setcounter{arrowheadx}{#1}\setcounter{arrowheady}{#2}
    \addtocounter{arrowheady}{-#3}
    \put(\value{arrowheadx},\value{arrowheady}){\vector(1,0){#4}}
    \addtocounter{arrowheadx}{#4}}
\newcommand{\rtarrow}[3]
   {\put(#1,#2){\vector(1,0){#3}}\put(#1,#2){\myowndot}
    \setcounter{arrowheadx}{#1}\setcounter{arrowheady}{#2}
    \addtocounter{arrowheadx}{#3}}
\newcommand{\rtarrowdown}[4]
   {\put(#1,#2){\line(1,0){#3}}\put(#1,#2){\myowndot}
    \setcounter{arrowheadx}{#1}\setcounter{arrowheady}{#2}
    \addtocounter{arrowheadx}{#3}
    \put(\value{arrowheadx},\value{arrowheady}){\vector(0,-1){#4}}
    \addtocounter{arrowheady}{-#4}}
\newcommand{\rtarrowx}[5]
   {\put(#1,#2){\line(1,0){#3}}\put(#1,#2){\myowndot}
    \setcounter{arrowheadx}{#1}\setcounter{arrowheady}{#2}
    \addtocounter{arrowheadx}{#3}
    \put(\value{arrowheadx},\value{arrowheady}){\line(0,-1){#4}}
    \addtocounter{arrowheady}{-#4}
    \put(\value{arrowheadx},\value{arrowheady}){\vector(1,0){#5}}
    \addtocounter{arrowheadx}{#5}}
\newcommand{\rtarrowux}[5]
   {\put(#1,#2){\line(1,0){#3}}\put(#1,#2){\myowndot}
    \setcounter{arrowheadx}{#1}\setcounter{arrowheady}{#2}
    \addtocounter{arrowheadx}{#3}
    \put(\value{arrowheadx},\value{arrowheady}){\line(0,1){#4}}
    \addtocounter{arrowheady}{#4}
    \put(\value{arrowheadx},\value{arrowheady}){\vector(1,0){#5}}
    \addtocounter{arrowheadx}{#5}}
\newcommand{\rtarrowxdown}[8]
   {\put(#1,#2){\line(1,0){#3}}\put(#1,#2){\myowndot}
    \setcounter{arrowheadx}{#1}\setcounter{arrowheady}{#2}
    \addtocounter{arrowheadx}{#3}
    \put(\value{arrowheadx},\value{arrowheady}){\line(0,-1){#4}}
    \addtocounter{arrowheady}{-#4}
    \put(\value{arrowheadx},\value{arrowheady}){\line(#7,0){#8}}
    \addtocounter{arrowheadx}{#5}
    \put(\value{arrowheadx},\value{arrowheady}){\vector(0,-1){#6}}
    \addtocounter{arrowheady}{-#6}}
\newcommand{\rtarwxdnrt}[9]
   {\put(#1,#2){\line(1,0){#3}}\put(#1,#2){\myowndot}
    \setcounter{arrowheadx}{#1}\setcounter{arrowheady}{#2}
    \addtocounter{arrowheadx}{#3}
    \put(\value{arrowheadx},\value{arrowheady}){\line(0,-1){#4}}
    \addtocounter{arrowheady}{-#4}
    \put(\value{arrowheadx},\value{arrowheady}){\line(#7,0){#8}}
    \addtocounter{arrowheadx}{#5}
    \put(\value{arrowheadx},\value{arrowheady}){\line(0,-1){#6}}
    \addtocounter{arrowheady}{-#6}
    \put(\value{arrowheadx},\value{arrowheady}){\vector(1,0){#9}}
    \addtocounter{arrowheadx}{#9}}
\newcommand{\rtarrowuxdown}[8]
   {\put(#1,#2){\line(1,0){#3}}\put(#1,#2){\myowndot}
    \setcounter{arrowheadx}{#1}\setcounter{arrowheady}{#2}
    \addtocounter{arrowheadx}{#3}
    \put(\value{arrowheadx},\value{arrowheady}){\line(0,1){#4}}
    \addtocounter{arrowheady}{#4}
    \put(\value{arrowheadx},\value{arrowheady}){\line(#7,0){#8}}
    \addtocounter{arrowheadx}{#5}
    \put(\value{arrowheadx},\value{arrowheady}){\vector(0,-1){#6}}
    \addtocounter{arrowheady}{-#6}}
%
%  Medium level graphics commands
%
\newcommand{\shiftboxcen}[1]{\addtocounter{boxcenx}{#1}}
\newcommand{\shiftboxcenup}[1]{\addtocounter{boxceny}{#1}}
\newcommand{\droparrow}[1]{\dparrow{\value{boxcenx}}{\value{boxceny}}{#1}}
\newcommand{\droparrowright}[2]
   {\dparrow{\value{boxcenx}}{\value{boxceny}}{#1}{#2}}
\newcommand{\fallarrow}[1]
   {\setcounter{fallh}{\value{boxceny}}
    \addtocounter{fallh}{\value{negdflthheight}}
    \dotoff\dparrow{\value{boxcenx}}{\value{fallh}}{#1}\doton}
\newcommand{\fallarrowright}[2]
   {\setcounter{fallh}{\value{boxceny}}
    \addtocounter{fallh}{\value{negdflthheight}}
    \dotoff\dparrowrt{\value{boxcenx}}{\value{fallh}}{#1}{#2}\doton}
\newcommand{\contarrow}[1]
   {\dotoff\rtarrow{\value{arrowheadx}}{\value{arrowheady}}{#1}\doton}
\newcommand{\contarrowlabeled}[2]
   {\put(\value{arrowheadx},\value{arrowheady}){\makebox(#1,\value{dfltradius}){#2}}
    \contarrow{#1}}
\newcommand{\contdownarrow}[1]
   {\dotoff\dparrow{\value{arrowheadx}}{\value{arrowheady}}{#1}\doton}
\newcommand{\contdownarrowlabeled}[3]
   {\setcounter{tempy}{\value{arrowheady}}\addtocounter{tempy}{-#2}
    \setcounter{tempx}{\value{arrowheadx}}\addtocounter{tempy}{2}
    \put(\value{tempx},\value{tempy}){\raisebox{-0.8ex}{#3}}
    \contdownarrow{#1}}
\newcommand{\outarrow}[1]{\rtarrow{\value{boxcenx}}{\value{boxceny}}{#1}}
\newcommand{\outarrowmsg}[2]
   {\outarrow{#1}
    \setcounter{tempx}{\value{arrowheadx}}\addtocounter{tempx}{2}
    \put(\value{tempx},\value{arrowheady}){\raisebox{-0.8ex}{#2}}}
\newcommand{\outarrowdown}[2]
   {\rtarrowdown{\value{boxcenx}}{\value{boxceny}}{#1}{#2}}
\newcommand{\outarrowdownright}[3]
   {\rtarrowx{\value{boxcenx}}{\value{boxceny}}{#1}{#2}{#3}}
\newcommand{\outarrowupright}[3]
   {\rtarrowux{\value{boxcenx}}{\value{boxceny}}{#1}{#2}{#3}}
\newcommand{\outarrowdownrightdown}[4]
   {\rtarrowxdown{\value{boxcenx}}{\value{boxceny}}{#1}{#2}{#3}{#4}{1}{#3}}
\newcommand{\outarrowdownleftdown}[4]
   {\rtarrowxdown{\value{boxcenx}}{\value{boxceny}}{#1}{#2}{-#3}{#4}{-1}{#3}}
\newcommand{\outarrowdownrightdownright}[5]
   {\rtarwxdnrt{\value{boxcenx}}{\value{boxceny}}{#1}{#2}{#3}{#4}{1}{#3}{#5}}
\newcommand{\outarrowdownleftdownright}[5]
   {\rtarwxdnrt{\value{boxcenx}}{\value{boxceny}}{#1}{#2}{-#3}{#4}{-1}{#3}{#5}}
\newcommand{\outarrowuprightdown}[4]
   {\rtarrowuxdown{\value{boxcenx}}{\value{boxceny}}{#1}{#2}{#3}{#4}{1}{#3}}
\newcommand{\outarrowupleftdown}[4]
   {\rtarrowuxdown{\value{boxcenx}}{\value{boxceny}}{#1}{#2}{-#3}{#4}{-1}{#3}}
\newcommand{\msgabove}[4]
   {\put(#1,#2){\makebox(#3,10){\raisebox{10 pt}{#4}}}}
\newcommand{\msgtoright}[4]
   {\put(#1,#2){\makebox(#3,10)[lb]{\raisebox{-5 pt}{#4}}}}
\newcommand{\messageabovearrow}[2]
   {\msgabove{\value{boxcenx}}{\value{boxceny}}{#1}{#2}}
\newcommand{\boxcenlineup}[1]
   {\put(\value{boxcenx},\value{boxceny}){\line(0,1){#1}}}
\newcommand{\boxcenlinedown}[1]
   {\put(\value{boxcenx},\value{boxceny}){\line(0,-1){#1}}}
\newcommand{\boxcenlineright}[1]
   {\put(\value{boxcenx},\value{boxceny}){\line(1,0){#1}}}
\newcommand{\boxcenlineleft}[1]
   {\put(\value{boxcenx},\value{boxceny}){\line(-1,0){#1}}}
%
%  Higher level graphics commands
%
\newcommand{\qrtcircled}[4]
   {\tempdoublewide{#4}
    \setcounter{tempx}{#1}\addtocounter{tempx}{#4}
    \put(\value{tempx},#2){\oval(\value{tempw},\value{tempw})}
    \setcounter{tempy}{#2}\addtocounter{tempy}{-#4}
    \put(#1,\value{tempy}){\makebox(\value{tempw},\value{tempw}){#3}}
    \addtocounter{tempw}{#1}
    \setcounter{arrowheadx}{\value{tempw}}\setcounter{arrowheady}{#2}}
\newcommand{\qdncircled}[4]
   {\tempdoublewide{#4}
    \setcounter{tempy}{#2}\addtocounter{tempy}{-#4}
    \put(#1,\value{tempy}){\circle{\value{tempw}}}
    \setcounter{tempx}{#1}\addtocounter{tempx}{-#4}
    \setcounter{temph}{\value{tempy}}
    \addtocounter{tempy}{-#4}
    \put(\value{tempx},\value{tempy}){\makebox(\value{tempw},\value{tempw}){#3}}
    \addtocounter{tempw}{\value{tempx}}
    \setcounter{arrowheadx}{\value{tempw}}\setcounter{arrowheady}{\value{temph}}}
\newcommand{\qlblnobox}[5]
   {\tempdoublewide{#4}\put(#1,#2){\makebox(\value{tempw},#5){#3}}
    \calcboxcen{#1}{#2}{#4}}
\newcommand{\qvalbox}[5]
   {\tempdoublewide{#4}\put(#1,#2){\framebox(\value{tempw},#5){#3}}
    \calcboxcen{#1}{#2}{#4}}
\newcommand{\qvalfolder}[5]
   {\qvalbox{#1}{#2}{#3}{#4}{#5}
    \setcounter{tempx}{#1}\addtocounter{tempx}{10}
    \setcounter{tempy}{#2}\addtocounter{tempy}{#5}
    \put(\value{tempx},\value{tempy}){\oval(18,8)[t]}}
\newcommand{\qleftlblbox}[7]
   {\setcounter{tempx}{#1}\addtocounter{tempx}{-#5}\addtocounter{tempx}{-2}
    \tempdoublewide{#6}
    \put(\value{tempx},#2){\makebox(#5,#7)[r]{#3}}
    \put(#1,#2){\framebox(\value{tempw},#7){#4}}
    \calcboxcen{#1}{#2}{#6}}
\newcommand{\qrightlblbox}[7]
   {\tempdoublewide{#5}\setcounter{tempx}{#1}
    \addtocounter{tempx}{\value{tempw}}\addtocounter{tempx}{2}
    \put(#1,#2){\framebox(\value{tempw},#7){#3}}
    \put(\value{tempx},#2){\makebox(#6,#7)[l]{#4}}
    \calcboxcen{#1}{#2}{#5}}
\newcommand{\qground}[3]
   {\tempdoublewide{#3}\setcounter{tempx}{#1}\setcounter{tempy}{#2}
    \addtocounter{tempy}{-18}\addtocounter{tempx}{-2}
    \put(\value{tempx},\value{tempy}){\line(1,0){4}}
    \addtocounter{tempy}{6}\addtocounter{tempx}{2}\addtocounter{tempx}{-#3}
    \put(\value{tempx},\value{tempy}){\line(1,0){\value{tempw}}}
    \addtocounter{tempy}{6}\addtocounter{tempx}{-#3}
    \addtocounter{tempw}{#3}\addtocounter{tempw}{#3}
    \put(\value{tempx},\value{tempy}){\line(1,0){\value{tempw}}}
    \addtocounter{tempy}{6}\addtocounter{tempx}{-#3}
    \addtocounter{tempw}{#3}\addtocounter{tempw}{#3}
    \put(\value{tempx},\value{tempy}){\line(1,0){\value{tempw}}}}
%
\newcommand{\messageunderarrow}[1]
   {\addtocounter{arrowheady}{\value{negdflthheight}}
    \qlblnobox{\value{arrowheadx}}{\value{arrowheady}}{#1}{0}{0}}
\newcommand{\groundunderarrow}
    {\qground{\value{arrowheadx}}{\value{arrowheady}}{6}}
\newcommand{\putwideboxunderarrow}[1]
   {\setcounter{nextboxx}{\value{arrowheadx}}
    \setcounter{nextboxy}{\value{arrowheady}}
    \addtocounter{nextboxx}{-#1}\addtocounter{nextboxy}{\value{negdfltheight}}}
\newcommand{\putboxafterarrow}
   {\setcounter{nextboxx}{\value{arrowheadx}}
    \setcounter{nextboxy}{\value{arrowheady}}
    \addtocounter{nextboxy}{\value{negdflthheight}}}
\newcommand{\wideboxtoright}[1]
   {\addtocounter{nextboxx}{#1}\addtocounter{nextboxx}{#1}}
\newcommand{\boxtoright}{\wideboxtoright{\value{dflthalfbox}}}
\newcommand{\putboxunderarrow}{\putwideboxunderarrow{\value{dflthalfbox}}}
\newcommand{\putuponebox}{\addtocounter{nextboxy}{\value{dfltheight}}}
\newcommand{\putuponetallbox}[1]
   {\tempdoublehigh{#1}\addtocounter{nextboxy}{\value{temph}}}
\newcommand{\putdownonebox}{\addtocounter{nextboxy}{\value{negdfltheight}}}
%
\newcommand{\stackwideAbox}[1]{\putwideboxunderarrow{#1}\putuponebox\savAboxpos}
\newcommand{\stackwideBbox}[1]{\putwideboxunderarrow{#1}\putuponebox\savBboxpos}
\newcommand{\stackwideCbox}[1]{\putwideboxunderarrow{#1}\putuponebox\savCboxpos}
\newcommand{\stackwideDbox}[1]{\putwideboxunderarrow{#1}\putuponebox\savDboxpos}
\newcommand{\stackwideEbox}[1]{\putwideboxunderarrow{#1}\putuponebox\savEboxpos}
\newcommand{\stackwideFbox}[1]{\putwideboxunderarrow{#1}\putuponebox\savFboxpos}
\newcommand{\stackwideGbox}[1]{\putwideboxunderarrow{#1}\putuponebox\savGboxpos}
\newcommand{\stackwideHbox}[1]{\putwideboxunderarrow{#1}\putuponebox\savHboxpos}
%
\newcommand{\stackAbox}{\putboxunderarrow\putuponebox\savAboxpos}
\newcommand{\stackBbox}{\putboxunderarrow\putuponebox\savBboxpos}
\newcommand{\stackCbox}{\putboxunderarrow\putuponebox\savCboxpos}
\newcommand{\stackDbox}{\putboxunderarrow\putuponebox\savDboxpos}
\newcommand{\stackEbox}{\putboxunderarrow\putuponebox\savEboxpos}
\newcommand{\stackFbox}{\putboxunderarrow\putuponebox\savFboxpos}
\newcommand{\stackGbox}{\putboxunderarrow\putuponebox\savGboxpos}
\newcommand{\stackHbox}{\putboxunderarrow\putuponebox\savHboxpos}
%
\newcommand{\stackAboxtoright}{\putboxafterarrow\putuponebox\savAboxpos}
\newcommand{\stackBboxtoright}{\putboxafterarrow\putuponebox\savBboxpos}
\newcommand{\stackCboxtoright}{\putboxafterarrow\putuponebox\savCboxpos}
\newcommand{\stackDboxtoright}{\putboxafterarrow\putuponebox\savDboxpos}
\newcommand{\stackEboxtoright}{\putboxafterarrow\putuponebox\savEboxpos}
\newcommand{\stackFboxtoright}{\putboxafterarrow\putuponebox\savFboxpos}
\newcommand{\stackGboxtoright}{\putboxafterarrow\putuponebox\savGboxpos}
\newcommand{\stackHboxtoright}{\putboxafterarrow\putuponebox\savHboxpos}
%
\newcommand{\nextAbox}{\unsavAboxpos\putdownonebox\savAboxpos}
\newcommand{\nextBbox}{\unsavBboxpos\putdownonebox\savBboxpos}
\newcommand{\nextCbox}{\unsavCboxpos\putdownonebox\savCboxpos}
\newcommand{\nextDbox}{\unsavDboxpos\putdownonebox\savDboxpos}
\newcommand{\nextEbox}{\unsavEboxpos\putdownonebox\savEboxpos}
\newcommand{\nextFbox}{\unsavFboxpos\putdownonebox\savFboxpos}
\newcommand{\nextGbox}{\unsavGboxpos\putdownonebox\savGboxpos}
\newcommand{\nextHbox}{\unsavHboxpos\putdownonebox\savHboxpos}
%
%
\newcommand{\dropradius}{\addtocounter{arrowheady}{\value{negdfltradius}}}
\newcommand{\fwdradius}{\addtocounter{arrowheadx}{\value{dfltradius}}}
\newcommand{\raddprt}{\dropradius\fwdradius}
\newcommand{\circlemsg}[3]
  {\qrtcircled{#1}{#2}{#3}{\value{dfltradius}}}
\newcommand{\circleafterarrow}[1]
  {\qrtcircled{\value{arrowheadx}}{\value{arrowheady}}{#1}{\value{dfltradius}}}
\newcommand{\circleunderarrow}[1]
  {\qdncircled{\value{arrowheadx}}{\value{arrowheady}}{#1}{\value{dfltradius}}}
\newcommand{\leftwidetalltagbox}[5]
  {\qleftlblbox{\value{nextboxx}}{\value{nextboxy}}{#1}{#2}{#3}{#4}{#5}}
\newcommand{\rightwidetalltagbox}[5]
  {\qrightlblbox{\value{nextboxx}}{\value{nextboxy}}{#1}{#2}{#3}{#4}{#5}}
\newcommand{\leftwidetagbox}[4]
  {\qleftlblbox{\value{nextboxx}}{\value{nextboxy}}{#1}{#2}{#3}{#4}{\value{dfltheight}}}
\newcommand{\rightwidetagbox}[4]
  {\qrightlblbox{\value{nextboxx}}{\value{nextboxy}}{#1}{#2}{#3}{#4}{\value{dfltheight}}}
\newcommand{\lefttagbox}[2]
  {\leftwidetagbox{#1}{#2}{\value{dfltfulltag}}{\value{dflthalfbox}}}
\newcommand{\righttagbox}[2]
  {\rightwidetagbox{#1}{#2}{\value{dflthalfbox}}{\value{dfltfulltag}}}
\newcommand{\valbox}[1]
  {\qvalbox{\value{nextboxx}}{\value{nextboxy}}{#1}{\value{dflthalfbox}}{\value{dfltheight}}}
\newcommand{\valfolder}[1]
  {\qvalfolder{\value{nextboxx}}{\value{nextboxy}}{#1}{\value{dflthalfbox}}{\value{dfltheight}}}
\newcommand{\widevalbox}[2]
  {\qvalbox{\value{nextboxx}}{\value{nextboxy}}{#1}{#2}{\value{dfltheight}}}
\newcommand{\widevalfolder}[2]
  {\qvalfolder{\value{nextboxx}}{\value{nextboxy}}{#1}{#2}{\value{dfltheight}}}
\newcommand{\nobox}[1]
  {\qlblnobox{\value{nextboxx}}{\value{nextboxy}}{#1}{\value{dflthalfbox}}{\value{dfltheight}}}
\newcommand{\widenobox}[2]
  {\qlblnobox{\value{nextboxx}}{\value{nextboxy}}{#1}{#2}{\value{dfltheight}}}
\newcommand{\widenoboxtall}[3]
  {\qlblnobox{\value{nextboxx}}{\value{nextboxy}}{#1}{#2}{#3}}
\newcommand{\nextbitinbyte}[1]{\valbox{#1}\boxtoright}
\newcommand{\bitsinbyte}[8]
  {\nextbitinbyte{#1}\nextbitinbyte{#2}\nextbitinbyte{#3}\nextbitinbyte{#4}
   \nextbitinbyte{#5}\nextbitinbyte{#6}\nextbitinbyte{#7}\nextbitinbyte{#8}
   \boxtoright}
\newcommand{\overbracemsg}[2]{$\overbrace{\makebox(#1,20){}}^{#2}$}
\newcommand{\underbracemsg}[2]{$\underbrace{\makebox(#1,20){}}_{#2}$}
\newcommand{\twoliner}[2]{{\begin{array}{c} {#1}\\{#2}\\ \end{array}}}
\newcommand{\twoovermsg}[3]
   {\overbracemsg{#1}{\twoliner{{\qqtiny{#2}}}{{\qqsmall{#3}}}}}
\newcommand{\twoundermsg}[3]
   {\underbracemsg{#1}{\twoliner{{\qqsmall{#2}}}{{\qqtiny{#3}}}}}
\newcommand{\putatbox}[1]{\put(\value{nextboxx},\value{nextboxy}){#1}}
\newcommand{\bitdummylabel}[6]
   {\unsavAboxpos\putatbox{\twoundermsg{42}{NP}{}}
    \wideboxtoright{21}\putatbox{\twoovermsg{28}{#1}{NWP}}
    \wideboxtoright{14}\putatbox{\twoundermsg{14}{Q}{}}
    \wideboxtoright{7}\putatbox{\twoovermsg{70}{#2}{ELE[1]}}
    \wideboxtoright{35}\putatbox{\twoundermsg{56}{ELE[2]}{#3}}
    \wideboxtoright{28}\putatbox{\twoovermsg{70}{#4}{ELE[3]}}
    \wideboxtoright{35}\putatbox{\twoundermsg{56}{ELE[4]}{#5}}
    \wideboxtoright{28}\putatbox{\twoovermsg{70}{#6}{ELE[5]}}
    \wideboxtoright{35}\putatbox{\twoundermsg{84}{unused}{}}}

\newcommand{\verboption}[1]{\unixverboption{#1}}
\newcommand{\option}[1]{\unixoption{#1}}
\newcommand{\literal}[1]{\unixliteral{#1}}
\makeindex
\begin{document}
\pagenumbering{roman}
\title{The ASSIST Internals Reference Manual}
\markboth{ASSIST Internals Reference Manual}{ASSIST Internals Reference Manual}
\author{Sally C. Johnson
        and \\
        David P. Boerschlein \\
         ~\\
         NASA Langley Research Center \\
          Hampton, VA 23681--5225 \\
       }
\maketitle
\begin{abstract}
The Abstract Semi-Markov Specification Interface to the
SURE Tool (\acronym{ASSIST}) program was developed at NASA Langley Research
Center in order to analyze the reliability of virtually any
fault-tolerant system.   A user manual \cite{assist7man} was developed to
detail its use.   Certain technical specifics are of no concern to the end
user, yet are of importance to those who must maintain and/or verify the
correctness of the tool.   This document takes a detailed look into these
technical issues.
\end{abstract}
 

\tableofcontents
\listoftables
\listoffigures
\newpage
\pagenumbering{arabic}
\section{Introduction}

This manual is designed to be used by system administrators and programmers
in order to be able to understand the internals of the \acronym{ASSIST}
program.   Parts of the manual may apply only to system administrators and
other parts may apply to only programmers.

This manual is quite technical and is not intended for the typical
end user.   Users are referred instead to the
user manual \cite{assist7man}.

It is assumed that the reader is already somewhat familiar with the syntax
and semantics of an \acronym{ASSIST} input file and can program in the
language.   This familiarity can be gained by reading the
user manual \cite{assist7man}.

It is also assumed that the reader has a basic knowledge of the
\acronym{ANSI} ``C''
programming language.   Good texts on
the ``C''\index{compilers required!ANSI C} language
\cite{Kernighan,Microsoft,quick} are readily
available.   Although it talks a lot about programming on the IBM,
Robert Lafore's book \cite{Microsoft} gives one of the more excelant
discussions of \acronym{ANSI} ``C''.

After covering this manual, the reader should:
\begin{enumerate}
\item know where to find the source code and how to compile and install
      it.
\item know which source code files contain functions for given purposes.
\item have an advanced understanding of the internals of
      and be able to write programs using features that the typical
      end users are unaware of.
\item have a detailed understanding of all data structures used by
      the \rw{ASSIST} source code.
\item be able to read an \rw{ASSIST} loadmap (\unixoption{loadmap} or
      \vmsoption{loadmap} option).
\item understand about an \rw{ASSIST} object file and how an \rw{ASSIST}
      program is parsed and stored in memory in preparation for model
      generation.
\item understand the pseudo-code language that \rw{ASSIST} uses to generate
      the model file.
\item be confident about the correctness of the algorithms and data
      structures used to implement the language.
\item be able to maintain the source code and understand about
      any subtle implications of any changes to it.
\end{enumerate}

\section{Source Code Files}
Because part of the source code for the \acronym{ASSIST} was of a general
token and expression parsing nature and a smaller part was specific to
the \acronym{ASSIST} program itself, not all of the
source\index{source code!location of} code resides in
the same directory.   This allows only one copy of common functions to be
kept for linking with other programs such
as \acronym{TOTAL} and \acronym{SURE}.   The directory\index{file structure}
tree structure showing the organization of the source code on disk is given
in Figure \ref{fig:filetree}.

\startfig
\begin{fast_picture}{300}{190}
\setradius{30}
\setrun{120}{160}\runtoarrow\circleafterarrow{s\_reliab}
\put(150,130){\line(0,-1){40}}
\put(30,90){\line(1,0){240}}
\put(30,90){\line(0,-1){30}}
\put(150,90){\line(0,-1){30}}
\put(270,90){\line(0,-1){30}}
\setrun{0}{30}\runtoarrow\circleafterarrow{lib}
\setrun{120}{30}\runtoarrow\circleafterarrow{common}
\setrun{240}{30}\runtoarrow\circleafterarrow{s\_assist}
\end{fast_picture}
\finishfig{Source Code File Tree Structure}{fig:filetree}\index{file structure}

The ``s\_reliab'' sub-directory can be located anywhere
on the file system by the system administrator.
Under it come the ``lib'', ``common'', and ``s\_assist'' sub-directories.
Under it also come other subdirectories, such as ``s\_sure\_stem\_paws'',
which are not of interest in this paper.

In the ``s\_reliab'' sub-directory there is a ``README'' file.   The file
is a shell script under \acronym{UNIX} and a command file under
\acronym{VMS}.   To run the respective script or command file, enter the
respective command as shown for each system:
\begin{codeexample}
% cat README        {\scriptsize UNIX command}
$ TYPE README.VMS   {\scriptsize VMS command}
\end{codeexample}

\subsection{Building all of the programs from scratch}

There is a command file that builds all of the programs from scratch.   To
build all of the programs, enter the respective command as shown for each
system:\index{installation procedure}
\begin{codeexample}
% make           {\scriptsize UNIX command, requires GNU-GCC compiler}
$ @make compile  {\scriptsize VMS command, requires VAX/VMS compiler}
\end{codeexample}\index{compilers required!GNU-GCC or VMS-CC}

Because some of the code for related utilities (such as \acronym{SURE})
is still written in Pascal, the standard operating system
Pascal\index{compilers required!Pascal} compiler is also required.

Before releasing the new versions, the following should already
have been done in order to properly test the versions:
\begin{itemize}
      \item Source code changes must have been made consistent
                 in all three areas, namely:
                 \begin{itemize}
                 \item \char91{dpb.s\_reliab}\char93
                 \item \~dpb/s\_reliab
                 \item \~dpb/sun3\_s\_reliab
                 \end{itemize}
      \item Help files on vax should have been edited and re-built in:
            [dpb.help]
      \item man files on sun should have been edited and tested in:
            \~dpb/s\_reliab/manl
      \item ASSIST manual changes should have been made as necessary
            in: \~dpb/assman
\end{itemize}

When a new version of assist/sure/stem/paws/ftc is tested and ready, the
following steps must be performed in sequence to ensure that the new version
will be given out in its entierty:

\begin{enumerate}
   \item Rebuild the sun-sparc version:
      \begin{itemize}
      \item login on a sparc
      \item cd \~dpb/s\_reliab
      \item make
      \end{itemize}
   \item Rebuild the sun-sun3 version:
      \begin{itemize}
      \item login on a sun-3
      \item cd \~dpb/sun3\_s\_reliab
      \item make
      \end{itemize}
   \item Rebuild the vax vms version:
      \begin{itemize}
      \item login on a vax
      \item set def disk\$wahoo:[dpb.s\_reliab]
      \item {\atsign}make compile
      \end{itemize}
   \item From a vax system, fetch the executables and
          put them into their proper bin directories:
      \begin{itemize}
      \item set def disk\$wahoo:[dpb.s\_reliab]
      \item {\atsign}fetch
      \end{itemize}
   \item From the sun-3 or sun-4, fetch all the executables and
          put them into their proper bin directories:
      \begin{itemize}
      \item cd \~dpb/s\_reliab
      \item fetch
      \end{itemize}
   \item From the sun-4, re-build all of the manuals:
      \begin{itemize}
      \item cd \~dpb/assman
      \item make
      \item grep "sun4" *
          (Note that the new versions of assist/sure/stem/paws will be used
          from \~dpb/s\_reliab/sun/bin4/ in order to re-process all examples
          with the new version numbers and features.)
!  a copy of the built
!  manuals will automatically be placed into \~dpb/s\_reliab/doc/.
      \item cd \~dpb/s\_reliab/doc
      \item all.lat
      \item all.pr
      \item cd \~dpb/s\_reliab/cosmic
      \item build\_release\_notes
      \item ftp\_release\_notes
      \end{itemize}
   \item Transfer the built documentation across the network:
      \begin{itemize}
      \item set def disk\$wahoo:[dpb.s\_reliab.doc]
      \item show def
      \item del *.*;*
      \item ftp air54
      \item login dpb
      \item cd ./s\_reliab/doc
      \item pwd
      \item mget CUR*E
      \item mget *.tex
      \item mget all.*
      \item type binary
      \item mget *.dvi
      \item quit
      \item purge
      \end{itemize}
   \item Transfer the built executables across the network:
      \begin{itemize}
      \item set def disk\$wahoo:[dpb.s\_reliab.bin]
      \item show def
      \item del [.sun3]*.*;*
      \item del [.sun4].*;*
      \item set def disk\$wahoo:[dpb.s\_reliab.bin.vms]
      \item show def
      \item ftp air54
      \item login dpb
      \item cd ./s\_reliab/bin/sun3
      \item type binary
      \item cd ../sun3
      \item lcd [-.sun3]
      \item pwd
      \item lpwd
      \item mget *
      \item cd ../sun4
      \item lcd [-.sun4]
      \item pwd
      \item lpwd
      \item mget *
      \item quit
      \end{itemize}
   \item Transfer the help files across the network:
      \begin{itemize}
      \item set def disk\$wahoo:[dpb.s\_reliab.manl]
      \item show def
      \item del *.*;*
      \item ftp air54
      \item login dpb
      \item cd ./s\_reliab/manl
      \item pwd
      \item mget *
      \item cd ../help
      \item lcd [-.help]
      \item pwd
      \item lpwd
      \item mput *.hlp
      \item mput *.com
      \item quit
      \end{itemize}
   \item Generate the VMS save set and copy to public\$tape 
      \begin{itemize}
      \item set def disk\$wahoo:[dpb.s\_reliab]
      \item {\atsign}gen\_saveset
      \item {\atsign}give\_saveset
      \item set def public\$tape
      \item ftp air54
      \item login dpb
      \end{itemize}
   \item Generate the tar file and copy to anonymous ftp
      \begin{itemize}
      \item cd \~dpb/sun3\_s\_reliab
      \item gen\_tarfile
      \item cd \~dpb/s\_reliab
      \item gen\_tarfile
      \item give\_both\_tarfiles
      \end{itemize}
   \item Copy the postscript ASSIST user manual to anonymous ftp
      \begin{itemize}
      \item cd \~dpb/s\_reliba/cosmic/s\_bundle/doc
      \item cp assist\_user\_manual.ps /home/ftp/pub/sure
      \item cd \~dpb/s\_reliab
      \item gen\_tarfile
      \item give\_both\_tarfiles
      \end{itemize}
   \item Compress the files in anonymous ftp to save transfer time
      \begin{itemize}
      \item rlogin air15
      \item cd /home/ftp/pub/sure
      \item pwd
      \item compress -v *.tar *.ps
      \end{itemize}
\end{enumerate}


The following can be done simultaneously:

\begin{tabular}{rl}
~~~~~~~~~~~~~~~~Phase I: & (1) (2) (3) \\
~~~~~~~~~~~~~~~~Phase II: & (4) (5) \\
~~~~~~~~~~~~~~~~Phase III: & (6) \\
~~~~~~~~~~~~~~~~Phase IV: & (7) (8) (9) \\
~~~~~~~~~~~~~~~~Phase V: & (10) (11) (12) \\
~~~~~~~~~~~~~~~~Phase VI: & (13) \\
\end{tabular}


\subsection{The ``lib'' subdirectory}

\index{source code!location of}
The ``lib'' sub-directory is used for source code which is of completely
general use and not just for reliability analysis utilities such as
\acronym{ASSIST}, \acronym{SURE}, \acronym{STEM}, \acronym{PAWS},
and \acronym{TOTAL}.   It contains the following source code files:
\begin{codeexample}
commonio.c
commonio.h
\end{codeexample}
as well as the following build files:
\begin{codeexample}
Makefile          {\scriptsize UNIX command, requires GNU-GCC compiler}
buildlib_vax.com  {\scriptsize VMS command, requires VAX/VMS compiler}
\end{codeexample}
The build files are automatically invoked by the make files in the
parent directory, so the system administrator will usually never need to
run them separately.

\subsubsection{The ``commonio'' library}
\label{sec:commonio}
The ``commonio'' library contains functions of a general input/output nature.
Among these are:
\begin{codeexample}
siglen                 {\scriptsize to compute the significant length of string less trailing whitespace}
safe_strcat_truncated  {\scriptsize to concatenate strings but only up to buffer length}
\end{codeexample}


\subsection{The ``common'' subdirectory}

\index{source code!location of}
The ``common'' sub-directory
is used for source code that is common to more than
one of the reliability analysis utilities which include
\acronym{ASSIST}, \acronym{SURE}, \acronym{STEM}, \acronym{PAWS},
and \acronym{TOTAL}.   It contains the following source code files:
\begin{codeexample}
calclim.c
doopval.c
evalexpr.c
factchek.c
factlib.c
facttest.c
gamma.c
lexlib.c
lowiolib.c
prseval.c
prsiolib.c
prslib.c
strlib.c
syslib.c
test_gam.c

cm_defs.h
cm_prsvars.h
cm_sys.h
cm_types.h
cm_vars.h
common.h
errdefs.h
eval_ext.h
fact_ext.h
gamma.h
gammalim.h
lex_ext.h
lextabsP.h
lib_def.h
lio_ext.h
main_def.h
pass.h
perrdefs.h
pio_ext.h
prs_ext.h
prs_math.h
prsdefs.h
prstabsP.h
prstypes.h
prsvars.h
rwdefs.h
safeeval.h
str_ext.h
strtabsP.h
sys_ext.h
tokdefs.h
user_ext.h
x_stdio.h
x_time.h
\end{codeexample}

as well as the following build files:
\begin{codeexample}
Makefile             {\scriptsize UNIX command, requires GNU-GCC compiler}
buildcommon_vax.com  {\scriptsize VMS command, requires VAX/VMS compiler}
\end{codeexample}
The build files are automatically invoked by the make files in the
parent directory, so the system administrator will usually never need to
run them separately.

\subsubsection{The ``calclim'' program}
\label{sec:calclim}
The ``calclim'' program is a program which attempts to compute
limits for required include file parameters which are system dependent.
It is of use for system administrators who wish to port to a new
system which is not yet supported.   The program writes several ``\#define''
constants to the standard output which can be redirected to a scratch file.
The contents of the scratch file can be edited into the ``gammalim.h'' file.

\subsubsection{The ``test\_gam'' program}
\label{sec:testgam}

The ``test\_gam'' program is used to test the output of the gamma functions
``cgamma'' and ``lngamma'' which give the complete gamma function and the
natural logarithm of the complete gamma function, respectively.

\subsubsection{The ``doopval.c'' include file}
\label{sec:doopval}
The ``doopval.c'' file
is a portion of source code which is included
twice in the ``prseval'' library.   It is included twice because there
are two versions of certain macros.   The first version is for the parse
stage and includes extra type checking.   The second version is for the
generation stage and has less type checking because it is assumed that
expressions have parsed without syntax errors.   The second version is
therefore significantly faster, which is important during rule generation.

\subsubsection{The ``evalexpr.c'' include file}
\label{sec:evalexpr}
The ``evalexpr.c'' file
is a portion of source code which is included
twice in the ``prseval'' library.   It is included twice because there
are two versions of certain macros.   The first version is for the parse
stage and includes extra type checking.   The second version is for the
generation stage and has less type checking because it is assumed that
expressions have parsed without syntax errors.   The second version is
therefore significantly faster, which is important during rule generation.

\subsubsection{The ``factlib'' library}
\label{sec:factlib}
The ``factlib'' library contains combinatorial functions.   Among these are
functions to compute factorials, combinations, and permutations.
% \begin{codeexample}
% ifact          {\scriptsize to compute the factorial of n}
% icomb          {\scriptsize to compute combinations of n taken k at a time}
% iperm          {\scriptsize to compute permutations of n taken k at a time}
% \end{codeexample}

\subsubsection{The ``gamma'' library}
\label{sec:gamma}
The ``gamma'' library contains the gamma function $\Gamma(x)$.   It has three
functions for the complete gamma function, the incomplete gamma function,
and the natural logarithm of the gamma function.

\subsubsection{The ``lexlib'' library}
\label{sec:lexlib}
The ``lexlib'' library contains functions used during lexical scanning of
an input file.   Among these are functions to decode numbers, read tokens,
search and add to the identifier table, allocate memory for lexical functions,
handle the identifier activation level scope for FOR indices and formal
macro parameters, sum numeric arrays, count Boolean arrays, collect cross
reference information for the cross reference map, and convert between
relative identifier number and identifier address in memory.
% \begin{codeexample}
% init_lexlib                  {\scriptsize initialization routine}
% re_init_lexlib               {\scriptsize initialization routine}
% set_idn_num_counts           {\scriptsize initialization routine}
% lookup_as_symbolic_token     {\scriptsize see if 1 or 2 char symbol token}
% lookup_operator_token        {\scriptsize lookup operator in table}
% give_rw_operation_token
% type decode_expected_integer
% type decode_expected_real
% decode_number
% which_reserved_word
% load_char_string_token
% interpret_alpha_token
% gettoken_only
% dumtoken
% gettoken_work
% showtoken
% allocate_identifier_table
% free_identifier_table
% next_id_ptr
% move_top_identifiers_down
% preset_searchid_abort_if_exists
% set_searchid_abort_if_exists
% reset_searchid_abort_if_exists
% qqsearchid
% declare_computational_ident
% abort_on_table_overflow
% save_value_in_number_table
% qqsearchid_new
% searchid_old
% searchid_dummy
% searchid_oldssv
% searchid_dummy_or_oldssv
% searchid_oldssvconst
% searchid_dcl
% searchid_dcl_at_top
% searchid_silent
% deactivate_identifier_level
% id_unique_first_n
% push_ident_scope
% pop_ident_scope
% type array_builtin_err
% arraysum
% arrayrowsum
% arraycolsum
% arraycount
% arrayrowcount
% arraycolcount
% count_the_sub_array
% sum_the_sub_array
% relative_identifier
% absolute_identifier
% already_declared_err
% statement_name_reserved_word
% compute_offset
% xref_declaration
% xref_setvalue
% xref_utilize
% XREF_FIX_UNSET_ETC
% xref_label
% \end{codeexample}

\subsubsection{The ``lowiolib'' library}
\label{sec:lowiolib}
The ``lowiolib'' library contains functions for low level input/output.
Included are system independent routines to open/close files,
check file permissions,
skip comments,
and convert between relative and abolute identifier addresses.

\subsubsection{The ``prseval'' library}
\label{sec:prseval}
The ``prseval'' library contains functions for evaluation of expressions
which have been successfully parsed with no error messages.  Among them
are functions to control the postfix operation and value stacks, perform
an operation between two numeric/Boolean quantities, and evaluate the
value of an expression.

\subsubsection{The ``prsiolib'' library}
\label{sec:prsiolib}
The ``prsiolib'' library contains low-level i/o functions for general
purpose parsing of reliability analysis programs.
Included are system independent routines to open/close files with
various buffer sizes,
manage temporary file names,
print error/warning messages, and
traverse the
input file a line at a time, and traverse the line
a character at a time.

\subsubsection{The ``prslib'' library}
\label{sec:prslib}
The ``prslib'' library contains functions for general purpose parsing of
reliability analysis programs.
Included are functions to determine the
precedence of expression operators,
test for certain kinds of reserved words,
convert from an infix operator to its
corresponding postfix operator, parse an expression,
and make use of the
``prseval'' library to evaluate expressions for an expected type of answer.
Also included are routines to parse ranges and option flag definitions
as described in the \rw{ASSIST} user manual \cite{assist7man}.

\subsubsection{The ``strlib'' library}
\label{sec:strlib}
The ``strlib'' library contains functions for general purpose string
manipulation.   Included are functions to compare strings,
capitalize strings, encode numbers, test for blank lines, compute
addresses in memory, construct filenames by appending file extents,
and strip off trailing whitespace.

\subsubsection{The ``syslib'' library}
\label{sec:syslib}
The ``syslib'' library contains functions of a system dependent nature.
Included are functions to compute and print the elapsed processing times
and routines to allocate zero filled memory.   Some systems automatically
initialize allocated memory to zeros and some
do not so system independent functions
were required which guarantee the memory to be initialized to binary zeros.

\subsubsection{User supplied library routines}
\label{sec:usersupplied}
Certain functions are referenced which are not defined.   These
are ``user supplied'' routines which are specific to the application.
Each utility (e.g., \acronym{ASSIST}, \acronym{TOTAL}) must supply its own
version.   The functions which must be supplied are listed in the file
``user\_ext.h''.
\subsection{The ``s\_assist'' subdirectory}

\index{source code!location of}
The ``s\_assist'' sub-directory
is used for source code that is specific to the \acronym{ASSIST} program.
It contains the following source code files:

\begin{codeexample}
assist.c
astgen.c
astlex.c
astprs.c
aststr.c
fixlib.c
iolib.c
macprs.c
objlib.c

agen_ext.h
alex_ext.h
alextabs.h
aprs_ext.h
ast_lexio_defs.h
ast_lexio_vars.h
astdefs.h
astr_ext.h
astrw.h
asttabsP.h
asttypes.h
astvars.h
booldefs.h
errtabsP.h
errvarsP.h
fix_ext.h
io_ext.h
mac_ext.h
macvars.h
obj_ext.h
objdefs.h
objtypes.h
objvars.h
objvarsP.h
uprs_ext.h
\end{codeexample}

as well as the following build files:
\begin{codeexample}
Makefile             {\scriptsize UNIX command, requires GNU-GCC compiler}
buildassist_vax.com  {\scriptsize VMS command, requires VAX/VMS compiler}
\end{codeexample}
The build files are automatically invoked by the make files in the
parent directory, so the system administrator will usually never need to
run them separately.

\subsubsection{The ``assist'' program}
\label{sec:assistmain}
The ``assist'' program contains the main function for the \acronym{ASSIST}
utility.  It also contains
the revision number\index{revision number!which source file}\index{version number!which source file}.

\subsubsection{The ``astgen'' library}
\label{sec:astgen}
The ``astgen'' library contains functions for generation of a model file
from the data structures constructed during the parse state.   Included are
routines to hash a state node into a \acronym{SURE} state number, maintain
the hash table, extend the hash table by grabbing more memory if necessary,
print transitions to the model file, and interpret the rule section code.
Details are given in Chapter \ref{chap:process}.

\subsubsection{The ``astlex'' library}
\label{sec:astlex}
The ``astlex'' library contains functions specific to
\acronym{ASSIST} and used during lexical scanning
of an input file.   Most lexical functions are of a more general nature
and are included in the ``lexlib'' library.
See Section \ref{sec:lexlib} for more details.

Among the functions in the ``astlex'' library are functions to produce
the cross reference map, get a token, and allocate parse memory.
Some of the routines are required by the common libraries.
See Section \ref{sec:usersupplied}.

\subsubsection{The ``astprs'' library}
\label{sec:astprs}
The ``astprs'' library contains functions which parse the grammar of the
\acronym{ASSIST} language.   A few expression parsing routines are of a
more general nature and are included in the ``prslib'' and ``prseval''
libraries.  See Sections \ref{sec:prslib} and \ref{sec:prseval} for
more details.

Among the functions in the ``astprs'' library are functions to
parse space pictures, starting states,
reserved word statements (such as \rw{INPUT}),
rules, macros (such as \rw{FUNCTION} and \rw{IMPLICIT}),
option definitions, and constant definitions.
Also included are functions to allocate parse memory, save data structures,
and load data structures into the minimum amount of memory necessary.

\subsubsection{The ``aststr'' library}
\label{sec:aststr}
The ``aststr'' library contains functions which manipulate strings in the
\acronym{ASSIST} language.   Most string routines are of a
more general nature and are included in the ``strlib'' library.
See Section \ref{sec:strlib} for more details.

Among the functions in the ``aststr'' library are functions to
pack and unpack state-space values and nodes.

\subsubsection{The ``fixlib'' library}
\label{sec:fixlib}
The ``fixlib'' library contains functions necessary to convert relative
offsets to absolute ``C'' pointers after reading in a ``.aobj'' file.

\subsubsection{The ``iolib'' library}
\label{sec:iolib}
The ``iolib'' library contains input/output functions specific to the
\acronym{ASSIST} language.   Most input/output routines are of a
more general nature and are included in the ``lowiolib'' library.
See Section \ref{sec:lowiolib} for more details.

Among the functions in the ``iolib'' library are functions to
strip the \acronym{ASSIST} command line options, handle error messages,
and print memory maps to the log file.

\subsubsection{The ``macprs'' library}
\label{sec:macprs}
The ``macprs'' library contains functions necessary to parse
the \acronym{ASSIST} macros (\rw{IMPLICIT} and \rw{FUNCTION} definitions
and references).  Included are functions to advance to the next token
wherever it may be (on the expansion stack or in the input file), parse
macros, and expand macros.

\subsubsection{The ``objlib'' library}
\label{sec:objlib}
The ``objlib'' library contains functions necessary to save and retrieve
the internal data structures into and from the object file.


\section{The format of the ``.aobj'' file}
\label{sec:aobj}

During parsing of an an \acronym{ASSIST} input file, the data and code
necessary to generate the model file is written to an object file.   This
file is re-read and loaded into memory before generating the transitions
between the states in the model.

The object (``.aobj'') file is defined to be a sequence of table entries.
Each table entry has four fields, namely:
\begin{itemize}
\item section index
\item count of ``n'' data elements
\item size of each data element
\item actual data ($ n \times size $ bytes)
\end{itemize}
The first table entry in the object file is always the header entry and
the last is always the end-of-file entry.
Figures \ref{fig:ObjFormat} and \ref{fig:TableEntry} give this format.

The format of the header is an array of MAX\_OBJ\_COUNTER\_DIM elements of
size memsize\_t.   For the \acronym{VAX} and \acronym{SUN} systems, the
elements are longs.   Each element is the number of bytes of memory required
for each of the corresponding sections of data and/or code.   These sections
are listed in table \ref{tab:ObjSec}.
Some of the constants in the table are defined in the file ``objdefs.h'' and
others are defined in the file ``astdefs.h''.

\starttab
\begin{small}\begin{tabular}{|c|c|c|}
\hline Section & \#define & description \\
\hline
1 & OBJ\_CHAR\_DATA & Character constants \\
2 & OBJ\_BOOL\_DATA & Boolean constants \\
3 & OBJ\_INT\_DATA & Integer (Long) constants \\
4 & OBJ\_REAL\_DATA & Real (Double) constants \\
5 & OBJ\_SOFF\_DATA & State offset structure constants \\
6 & future usage & future usage \\
7 & future usage & future usage \\
8 & future usage & future usage \\
9 & OBJ\_CHAR\_VARDATA & Character variables \\
10 & OBJ\_BOOL\_VARDATA & Boolean variables \\
11 & OBJ\_INT\_VARDATA & Integer (Long) variables \\
12 & OBJ\_REAL\_VARDATA & Real (Double) variables \\
13 & OBJ\_SOFF\_VARDATA & State offset structure variables \\
14 & future usage & future usage \\
15 & future usage & future usage \\
16 & OBJ\_EXPR & Expression structures \\
17 & OBJ\_OPERANDS & Expression operand pointers \\
18 & OBJ\_OPS & Expression operation constants \\
19 & OBJ\_VARINF & Variable pointer unions \\
20 & OBJ\_SETRNGE & Set range bound structures \\
21 & OBJ\_PIX & State space picture data \\
22 & OBJ\_BOOLTEST & Boolean test expression structures \\
23 & OBJ\_TRANTO & TRANTO clause structures \\
24 & OBJ\_IF & Block if structure \\
25 & OBJ\_FOR & for construct (loop) structures \\
26 & OBJ\_CALC & for variable calculations \\
27 & future usage & future usage \\
$\vdots$ & $\vdots$ & $\vdots$  \\
39 & future usage & future usage \\
40 & OBJ\_CODE\_0 & Beginning of code (PREAMBLE) \\
41 & OBJ\_CODE\_0 + 1 + OPCODE\_ASSERT & Code (ASSERT section) \\
42 & OBJ\_CODE\_0 + 1 + OPCODE\_DEATHIF & Code (DEATHIF section) \\
43 & OBJ\_CODE\_0 + 1 + OPCODE\_PRUNEIF & Code (PRUNEIF section) \\
44 & OBJ\_CODE\_0 + 1 + OPCODE\_TRANTO & Code (TRANTO section) \\
45 & OBJ\_CODE\_0 + 1 + OPCODE\_CALC & Code (CALC-booltest section) \\
46 & OBJ\_CODE\_0 + 1 + OPCODE\_CALC\_T & Code (CALC-transition section) \\
47 & future usage & Code (future section) \\
0xf0 & OBJ\_HEADER & header record \\
0xf1 & OBJ\_IDTABLE & Identifier table \\
0xf2 & future & future use \\
0xf3 & OBJ\_OPREC & Option record \\
0xf4 & OBJ\_VERBATIM\_HEAD & declarations sent to model file verbatim \\
0xf5 & OBJ\_VERBATIM\_TAIL & trailing text to write to model file \\
     & & (such as ``$\backslash$nRUN;$\backslash$nEXIT$\backslash$n;'' with \option{pipe} option) \\
0xff & OBJ\_EOF & End-of-file entry \\
\hline
\end{tabular}\end{small}
\finishtab{Object Code Table Sections}{tab:ObjSec}

\startfig
\begin{fast_picture}{240}{130}
\setboxpos{0}{120}\savBboxpos
      \nextBbox\widevalbox{HEADER table entry}{120}
      \nextBbox\widevalbox{table entry}{120}
      \nextBbox\widevalbox{table entry}{120}
      \nextBbox\widevalbox{$\cdots$}{120}
      \nextBbox\widevalbox{EOF table entry}{120}
\end{fast_picture}
\finishfig{Format of ``.aobj'' file}{fig:ObjFormat}

\startfig
\begin{fast_picture}{360}{60}
\setboxpos{0}{45}\savBboxpos
\nextBbox
      \widevalbox{long}{30}\wideboxtoright{30}
      \widevalbox{long}{30}\wideboxtoright{30}
      \widevalbox{long}{30}\wideboxtoright{30}
      \widevalbox{byte array}{90}
\nextBbox
      \widevalbox{Section}{30}\wideboxtoright{30}
      \widevalbox{count}{30}\wideboxtoright{30}
      \widevalbox{size each}{30}\wideboxtoright{30}
      \widevalbox{optional data}{90}
\end{fast_picture}
\finishfig{Format of each table entry in ``.aobj'' file}{fig:TableEntry}


When an object file is read and loaded into memory, the following steps
take place in the order listed:

\begin{enumerate}
\item The object file is read and loaded in its entirety.  Memory is
      allocated after reading the header, identifer table, and number
      table records (i.e, up to three blocks of memory are allocated).
      Memory allocated while reading the header is divided as required
      for the separate sections.
      The OBJ\_OPREC data is read directly into the static storage for
      the ``option\_rec'' structure.   Other data is loaded into memory
      The identifier table pointers are adjusted with a call to
      ``fixup\_identifier\_table'' upon loading the table.
\item All expression pointers are adjusted with a call to ``fixup\_expressions''.
\item All other pointers are adjusted with a call to ``fixup\_user''.
\item The optional memory map (\option{loadmap}) is generated.
\end{enumerate}


\section{Processing of an Input File}
\label{chap:process}

The processing of an \acronym{ASSIST} input file takes place in two phases:
parsing of the input file followed by model generation.

The diagram in Figure \ref{fig:dataflow2}
illustrates the data information flow between
files during the processing of an \acronym{ASSIST} input file.

\startfig
\begin{fast_picture}{500}{170}
\put(70,20){\dashbox{4}(370,140){}}\put(70,140){\makebox(45,20){ASSIST}}
\setboxpos{0}{100}\putuponebox\savFboxpos
    \nextFbox\widevalfolder{foo.ast}{30}\shiftboxcen{30}\dotoff
    \outarrowdownleftdown{160}{30}{110}{20}\savAarwpos
    \messageabovearrow{160}{{\scriptsize parse phase}}
             \savBboxpos\shiftboxcen{160}
    \shiftboxcenup{-12}
             \messageabovearrow{120}{{\scriptsize input lines}}\outarrow{120}
             \putboxafterarrow\addtocounter{nextboxy}{-5}
             \widevalfolder{foo.alog}{30}
    \unsavBboxpos\recalcboxcen{30}\shiftboxcen{190}
             \boxcenlineup{35}\shiftboxcenup{35}
             \messageabovearrow{200}{{\scriptsize named constants, etc.}}
             \outarrowdownleftdownright{200}{70}{100}{30}{20}
             \putboxafterarrow\widevalfolder{foo.mod}{30}\shiftboxcen{30}
                              \outarrowmsg{60}{SURE}
    \unsavAarwpos\putwideboxunderarrow{30}
             \widevalfolder{foo.aobj}{30}\shiftboxcen{30}
             \outarrowupright{150}{42}{50}
             \messageabovearrow{150}{{\scriptsize model generation phase}}
             \shiftboxcen{150}\shiftboxcenup{26}
                              \messageabovearrow{50}{{\scriptsize map}}
             \shiftboxcenup{-26}\boxcenlinedown{10}\shiftboxcenup{-10}
             \outarrow{50}\shiftboxcenup{-15}
                          \messageabovearrow{50}{{\scriptsize transitions}}
\end{fast_picture}

\finishfig{Data File Flow in ASSIST}{fig:dataflow2}

%...\subsection{Input File Parsing}
%...
%...The following is a pseudo-code version of the algorithm used to parse
%...the input file:
%...
%...\begin{codeexample}
%...(* ========================= main algorithm ========================= *)
%...
%...MAIN:
%...   begin in setup section.
%...   PARSE SETUP SECTION:
%...   WHILE (more statements) AND (still in setup section) DO:
%...         call PARSE A STATEMENT.
%...   ENDWHILE.
%...   PARSE START SECTION:
%...   WHILE (more statements) AND (still in start section) DO:
%...         call PARSE A STATEMENT.
%...   ENDWHILE.
%...   PARSE RULE SECTION:
%...   WHILE (more statements) DO:
%...         call PARSE A STATEMENT.
%...   ENDWHILE.
%...   STOP PARSING.
%...END.
%...
%...(* ====================== subroutines/functions ===================== *)
%...
%...SUBR  PARSE A STATEMENT:
%...      check first token for reserved word.
%...      IF (wrong section) THEN print error message.
%...      IF (new statement is a rule) AND (not in rule section) THEN:
%...         IF (space statement is missing) THEN print error message.
%...         IF (start statement is missing) THEN print error message.
%...         do not parse the statement yet.
%...         now in rule section.
%...      ELSE:
%...         parse the statement.
%...         IF (statement was space statement) THEN now in start section.
%...         IF (statement was start statement) THEN now in start section.
%...      ENDIF.
%...ENDSUBR.
%...\end{codeexample}
%...
%...\subsection{Model Generation}
%...
%...The model is generated using an iterative algorithm.   The algorithm begins
%...with the start state and recursively
%...applies all transitions to it.   All states out
%...of a given state are generated and added to the ready list before the next
%...state in the list is expanded.
%...
%...\subsection{Model Generation Algorithm}

\label{sec:modelgen}
The model generation algorithm builds the model
from the start state by recursively
applying the \rw{TRANTO} rules.  A list of states to be processed, called
the ``Ready Set'', begins with only the start state.  Before application
of a rule, \acronym{ASSIST} checks all of the \rw{ASSERT} conditions and
prints any warning messages.  It then checks
all death conditions to determine if the current state is a death state.
Since a death state denotes system failure, no transitions can leave a death
state.
If the state is not a death state, \acronym{ASSIST}
then checks all prune conditions
to determine if the current state is a prune state.
If \acronym{ASSIST} finds a state-space variable that is out of range
or detects some other error in the state, the state is treated as a death
state.
Each of the \rw{TRANTO} rules is then evaluated for the nondeath state.
If the condition expression of the \rw{TRANTO} rule evaluates to true for the 
current state, then the destination expression is used to determine
the state-space
variable values of the destination state.  If the destination state has not
already been defined in the model, then the new state is added to the Ready Set
of states to be processed.  The rate of the transition is determined from the
rate expression, and the transition description is printed to the model file.
When all of the \rw{TRANTO} rules have been applied to it, the state is
removed from the Ready Set.  When the Ready Set is empty, then all possible
paths terminate in death states, and model building is complete.

By default, all death states are aggregated or lumped (\rw{ONDEATH ON})
according to the first \rw{DEATHIF} statement to which the state conformed.
If the user turns \rw{ONDEATH OFF}, then all death states are physically stored
in memory.   Death states, when physically stored, are flagged as being
death states to distinguish them from operational states.   If the death
states are lumped, then death states are not physically stored in memory.
Only operational states are stored and none of them are flagged.

Note that the ready list is a subset of the set of all state nodes that have
been processed up to any given point in time.   All state nodes which have
been processed must remain in memory so as to be able to look up each new
destination state.   States which have already been processed are not
processed again (in the case of loops).

The following is a pseudo-code version of the algorithm used to generate
the model:
\begin{logfileexample}
(* ====================== subroutines/functions ===================== *)
FUNC  PROCESS(state,trim,fast,in\_error)
      (* note that ``fast'' is ignored when trimming is off *)):
      state number \(\longleftarrow\) search existing states.
      IF (state already present) THEN:
         is a death state if flagged as such.
         is not a prune state.
      ELSE:
         save current state and dependent variable values.
            recompute dependent variables which were referenced in
                 ASSERT, DEATHIF, PRUNEIF sections for state.
            test all ASSERT's printing WARNING message when a test fails.
            test all DEATHIF's.
            IF (not a death) test all PRUNEIF's.
         restore current state and dependent variable values.
         IF (in error) THEN:
            consider as death state.
         ENDIF.
         IF (death or prune state) AND (fast) AND (trimming is on) THEN:
            warning message \(\longrightarrow\)
                 Model contains recovery transitions directly to death state
                 and therefore may not be suited to trimming.
         ENDIF.
      ENDIF.
      IF (death state) AND (lumping) THEN:
         state number \(\longleftarrow\) death state number.
      ELSE IF (prune state) THEN:
         state number \(\longleftarrow\) prune state number.
      ELSE IF ((trim) AND (trimming is on)) THEN:
         state number \(\longleftarrow\) trim state number.
      ELSE (* not being lumped *):
         IF (death state) THEN:
            flag the state.
         ENDIF.
         IF (state does not yet exist) THEN:
            state number \(\longleftarrow\) add the state to the ready list.
         ENDIF.
      ENDIF.
      RETURN state number.
ENDFUNC.

(* ========================= main algorithm ========================= *)

MAIN:
    (* process the start state *)
    error \(\longleftarrow\) compute start state.
%    compute dependent variables referenced in TRANTO section.
    start state number \(\longleftarrow\) call PROCESS(start state,NORMAL,N/A,error).

    (* generate the model *)
    ready list \(\longleftarrow\) pointer to first state.
    FOR current-state IN [all states on ready list] LOOP:
        IF (state is not flagged) THEN:
           set fast transition counter to zero.
           FOR (all recovery (fast) TRANTO's) DO:
               error \(\longleftarrow\) compute new state.
               new state number \(\longleftarrow\) call PROCESS(new state,NORMAL,FAST,error).
               print the transition to the model file.
               increment fast transition counter.
           ENDFOR.
           IF (trimming is on) THEN:
               set slow transition counter to zero.
               FOR (all non-recovery (slow) TRANTO's) DO:
                   error \(\longleftarrow\) compute new state.
                   IF (fast transition counter > 0) THEN:
                      new state number \(\longleftarrow\) call PROCESS(new state,TRIM,SLOW,error).
                   ELSE:
                      new state number \(\longleftarrow\) call PROCESS(new state,NORMAL,SLOW,error).
                   ENDIF.
                   print the transition to the model file.
               ENDFOR.
           ELSE:
               set slow transition counter to zero.
               FOR (all non-recovery (slow) TRANTO's) DO:
                   compute new state.
                   new state number \(\longleftarrow\) call PROCESS(new state,NORMAL,N/A).
                   print the transition to the model file.
               ENDFOR.
           ENDIF.
        ENDIF.
        warning if no transitions out of a non-death state.
        ready list \(\longleftarrow\) increment pointer to next ready state.
    ENDFOR.

    (* print extra trim transitions *)
    IF (trimming is on) THEN:
       FOR current-state IN [trim state only] DO:
           print transition from current-state to trim death
                 state BY TRIMOMEGA.
       ENDFOR.
       IF  (pruning with TRIMOMEGA (i.e., trim=2)) THEN
           FOR current-state IN [prune states] DO:
               print transition from current-state to current prune
                     death state BY TRIMOMEGA.
           ENDFOR.
       ENDIF.
    ENDIF.
    FOR (all TRANTO's) DO:
       IF (never referenced) THEN:
           print a warning message that TRANTO was never used.
       ENDIF.
    ENDFOR.
    IF (fatal error occurred) THEN:
       flag model file as erroneous.
    ENDIF.
STOP.
\end{logfileexample}



\section{The Pseudo Code Language used by ASSIST}
\label{chap:pseudo}

The pseudo code language used by \rw{ASSIST} is similar to a stripped down
assembly language with instructions which very closely resemble statements
and expression operations which very closely
resemble expression syntax in the \rw{ASSIST} language.

\subsection{Instructions}
\label{sec:instructions}

Every instruction has two parts, namely an operation code and a pointer to
the either a data structure or another instruction.  The instruction pointer
union occurs first in the structure because some
system architectures require pointer alignment on full or half word boundaries.
The opcode comes first in the pseudo language, so it will be described first.

An instruction is stored in data of the following types:
\begin{codeexample}
typedef struct t__instruction_pointer_union
   \{
       void *vvv;                      /* to cast to block_if_type, etc. */
       relative_address_type reladdr;  /* relative address of code */
   \} instruction_pointer_union_type;
\end{codeexample}

\begin{codeexample}
typedef struct t__instruction
   \{
       instruction_pointer_union_type ptr;
       opcode_type opcode;      /* instruction operation code */
   \} instruction_type;          /*    for_loop_type, assert_type, ... */
\end{codeexample}


The following operation codes are defined:
\begin{itemize}
\item The {\bf ASSERT} instruction makes an assertion and prints a warning
      message when the current state does not pass the assertion.   It has
      one required parameter which is a non-null pointer to a Boolean test
      data structure for the condition to be tested for conformance.
      See Section \ref{sec:assert} on page \pageref{sec:assert}
      for details on this data structure.
\item The {\bf DEATHIF} instruction tests the current state and signals
      system failure when a condition is met.  It has
      one required parameter which is a non-null pointer to a Boolean test
      data structure for the condition to be tested for system failure.
      See Section \ref{sec:deathif} on page \pageref{sec:deathif}
      for details on this data structure.
\item The {\bf PRUNEIF} instruction tests the current state and signals
      system failure due to model pruning when a condition is met.  It has
      one required parameter which is a non-null pointer to a Boolean test
      data structure for the condition to be tested for system failure due
      to model pruning.
      See Section \ref{sec:pruneif} on page \pageref{sec:pruneif}
      for details on this data structure.
\item The {\bf TRANTO} instruction defines a transition from one state to
      another state.  It has
      one required parameter which is a non-null pointer to a transition
      data structure.
      See Section \ref{sec:tranto} on page \pageref{sec:tranto}
      for details on this data structure.
\item The {\bf CALC} instruction defines the computation of a dependent
      variable.   Such variables differ from named constants in that they
      are dependent upon the state-space variables.
      See Section \ref{sec:calc} on page \pageref{sec:calc}
      for details on this data structure.
\item The {\bf START} instruction defines a transition to the start state.
      It has
      one required parameter which is a non-null pointer to a transition
      data structure.
      The starting transition occurs in the preamble section code and
      points to a transion containing all constant expressions with no
      references to any variables whatsoever.
      See Sections \ref{sec:start} and  \ref{sec:tranto}
      on pages \pageref{sec:start} and \pageref{sec:tranto}
      for details on this data structure.
\item The {\bf SPACE} instruction occurs in the preamble and points to
      the state-space picture.   It causes an address to be loaded into
      the state-space picture register.   The state-space picture is
      used when a comment is printed to the model file.  The SPACE instruction
      has one required parameter which is a non-null pointer to the state
      space picture data.
      See Section \ref{sec:space} on page
      \pageref{sec:space}
      for details on this data structure.
\item The {\bf BLOCK\_IF} instruction can occur in any section except the
      preamble section.   It has
      one required parameter which is a non-null pointer to a block IF
      data structure as defined later in Section \ref{sec:blockif}.
      The data structure must contain test condition data and pointers
      to THEN and ELSE subroutines.  The IF pseudo-instruction
      will set aside space for a data structure and load it with the
      addresses for the GOSUB instructions.  For example,
      ``IF \bnf{expr} THEN GOSUB \bnf{code}'' or 
      ``IF \bnf{expr} THEN GOSUB \bnf{code} ELSE GOSUB \bnf{code}''.
      See Section \ref{sec:blockif} on page \pageref{sec:blockif}
      for details on this data structure.
\item The {\bf FOR\_LOOP} instruction can occur in any section except the
      preamble section.   It has
      one required parameter which is a non-null pointer to a FOR
      data structure as defined later in Section \ref{sec:forloop}.
      See Section \ref{sec:forloop} on page \pageref{sec:forloop}
      for details on this data structure.
\item The {\bf BEGIN} instruction denotes the beginning of a rule section
      of code.   There must appear, in the preamble section, exactly one
      BEGIN instruction for each of the rule sections.   These must
      appear in the correct sequence.   Each BEGIN instruction has
      one required parameter which is a non-null pointer to the beginning
      of the code for the corresponding rule section.   The first begin
      corresponds to the first rule section (ASSERT section), the second
      to the second section (DEATHIF section), etc.   When an program is
      executed, the BEGIN addresses are stored in the BEGIN registers
      and an implicit GOSUB is executed to each of these addresses
      for each state in the model during processing (see
      Section \ref{sec:modelgen}).
\item The {\bf END} instruction is used to terminate the preamble section
      of the code.   It has no parameters.
\item The {\bf GOTO} instruction is used to jump to an instruction.  It has
      one required parameter which is a non-null pointer to the instruction
      where processing is to continue.   Control will never continue with
      the next instruction following a GOTO unless there is another GOTO
      which points to it.
\item The {\bf GOSUB} instruction denotes transfer of control to a subroutine.
      It causes a recursive call to the subroutine evaluator unit which
      processes instructions until either an END or RETURN instruction is
      encountered.
\item The {\bf RETURN} instruction denotes the end of a subroutine and
      instructs the subroutine evaluator to return to its invoking activation
      level causing processing to resume with the instruction immediately
      following the most recently encountered GOSUB.   If the activation
      level was invoked due to a BEGIN instruction, then processing of the
      current state will continue on as detailed in Section \ref{sec:modelgen}.
\end{itemize}

The instruction pointer union contains a pointer to another instruction
or one of thee following data types:
\begin{codeexample}
typedef struct t__booltest
   \{
       expression_type *expr;         /* boolean expr to ASSERT,DEATHIF,etc. */
       short source_code_line_number; /* line number in listing file */
       short lumping_sequence;        /* sequence index (0..n-1) in source */
   \} booltest_type;                   /* e.g., first DEATHIF, second DEATHIF */
\end{codeexample}

\begin{codeexample}
typedef struct t__block_if
   \{
       expression_type *then_test;    /* boolean expression for THEN */
       instruction_type *then_clause; /* code for THEN clause */
       instruction_type *else_clause; /* code for ELSE clause */
   \} block_if_type;
\end{codeexample}

\begin{codeexample}
typedef struct t__for_loop
   \{
       identifier_info_type *ident;  /* index variable */
       set_range_type *set_ranges;   /* pointer to array of IN ranges */
       short set_range_count;        /* count of number of IN ranges */
       instruction_type *body;       /* pointer to BODY of loop */
   \} for_loop_type;
\end{codeexample}

\begin{codeexample}
typedef struct t__tranto_clause
   \{
       space_expression_type sex;   /* list of space transition expressions */
       expression_type *rate_exprs; /* ptr to array of rate expressions */
       short n_rate_exprs;          /* count of rate expressions */
       short source_code_line_number;
   \} tranto_clause_type;
\end{codeexample}

\begin{codeexample}
typedef struct t__state_space_picture
   \{
       vars_union_type *varu;
       Boolean *is_nested;
       short nvaru;
   \} state_space_picture_type;
\end{codeexample}

The kind of data structure pointed to is inherently implied by the specific
operation code in the instruction.

The language is designed so as to guarantee that the first ``n''
opcodes correspond to the rule sections, come first, and
are numbered $0$ through $n-1$.   All opcodes are ordinally contiguous.
The value of ``n'' is given by the RULE\_OPCODE\_INDEX\_COUNT constant.

The section in which an opcode appears is masked into the high bits of
the opcode field.

Examples of pseudo code programming are not given here because pseudo
code programs are not directly written by programmers.   If the user
asks for a load map (\option{loadmap} option), the format of the map
will use these pseudo operation codes.   See section \ref{sec:codeexample}
on page \pageref{sec:codeexample} for a sample memory layout which includes
the pseudo code.

\subsection{Expression operators}
\label{sec:ops}

Expressions are made up of operands, which are pointers into the identifier
table, combined with operations for the arithmetic/logic unit (ALU).

The same set of operations are used in both infix and postfix expressions.
A few of the operations in the set, however, apply only to one or the other.

There are two operations which are used to instruct the unit to look
for an operand.  These are listed in Table \ref{tab:ALU-opnd}.
Both of these operations are of the highest precedence.

\starttab
\begin{tabular}{|c|c|}
\hline OP\_VAL    & Operand specified by user \\
\hline OP\_INSVAL & Operand inserted during parsing \\
\hline
\end{tabular}
\finishtab{Operand referencing operations in ALU}{tab:ALU-opnd}

The remaining operations will be covered from lowest precedence to highest
precedence.

The binary arithmetic operations act upon two numerical values and yield
either a numeric or a logical (Boolean) result.   These operations are
given in Table \ref{tab:ALU-binarith}

\starttab
\begin{tabular}{|c|c|}
\hline OP\_OR            & $x$ OR $y$ \\
\hline OP\_XOR           & $x$ XOR $y$ \\
\hline OP\_AND           & $x$ AND $y$ \\
\hline OP\_BOOL\_EQ      & $x$ $==$ $y$ \\
\hline OP\_BOOL\_NE      & $x$ $\sim\sim$ $y$ \\
\hline OP\_LT            & $x$ $<$ $y$ \\
\hline OP\_GT            & $x$ $>$ $y$ \\
\hline OP\_LE            & $x$ $<=$ $y$ \\
\hline OP\_GE            & $x$ $>=$ $y$ \\
\hline OP\_EQ            & $x$ $=$ $y$ \\
\hline OP\_NE            & $x$ $<>$ $y$ \\
\hline OP\_ADD           & $x$ $+$ $y$ \\
\hline OP\_SUB           & $x$ $-$ $y$ \\
\hline OP\_MUL           & $x$ $*$ $y$ \\
\hline OP\_DVD           & $x$ $/$ $y$ \\
\hline OP\_MOD           & $x$ MOD $y$ \\
\hline OP\_CYC           & $x$ CYC $y$ \\
\hline OP\_QUO           & $x$ DIV $y$ \\
\hline OP\_POW           & $x$ $**$ $y$ ~~~ (infix only) \\
\hline OP\_RPOWR         & $x$ $**$ $y$ ~~~ (postfix only) \\
\hline OP\_IPOWI         & $i$ $**$ $n$ ~~~ (postfix only) \\
\hline OP\_RPOWI         & $x$ $**$ $n$ ~~~ (postfix only) \\
\hline
\end{tabular}
\finishtab{Binary arithmetic operations in ALU}{tab:ALU-binarith}

Certain arithmetic operations act upon a pointer to the beginning of an
array and the integer value of an index.   The pointer actually points
into the identifier table so that type and range bound information is
available.    These operations are given in
Table \ref{tab:ALU-array}.   Note that \rw{OP\_IXDBY2} is a tertiary
operator since it acts upon a pointer into the identifier table followed
by a primary index value followed by a secondary index value.

\starttab
\begin{tabular}{|c|c|}
\hline OP\_IXDBY         & $v$ \char91 ~$i$~ \char93 (postfix only) \\
\hline OP\_IXDBY2        & $v$ \char91 ~$i,j$~ \char93 (postfix only) \\
\hline OP\_I\_LB         & \char91  (infix only) \\
\hline OP\_I\_WILD       & $*$ (infix only) \\
\hline OP\_I\_RB         & \char93  (infix only) \\
\hline
\end{tabular}
\finishtab{Array element referencing operations in ALU}{tab:ALU-array}

Certain additional operations are used in infix expressions in order to
group operands and/or expressions together.   These are listed in
Table \ref{tab:ALU-group}.

\starttab
\begin{tabular}{|c|c|}
\hline OP\_PARENS   & $(~~)$ ~~~ (future use)\\
\hline OP\_I\_LP    & $($ ~~~ (infix only) \\
\hline OP\_I\_RP    & $)$ ~~~ (infix only) \\
\hline OP\_I\_CMMA  & $,$ ~~~ (infix only) \\
\hline
\end{tabular}
\finishtab{Grouping operations in ALU}{tab:ALU-group}

Unary arithmetic operators are given in Table \ref{tab:ALU-unarith}.

\starttab
\begin{tabular}{|c|c|}
\hline OP\_INC     & $n$ $++$ \\
\hline OP\_DEC     & $n$ $-~-$ \\
\hline OP\_NOT     & NOT $p$ \\
\hline OP\_NEG     & $-$ $x$ \\
\hline OP\_STNCHR  & $x$ $\longrightarrow$ $blank$ \\
\hline OP\_STNBOO  & $x$ $\longrightarrow$ FALSE \\
\hline OP\_STNINT  & $n$ $\longrightarrow$ 0 \\
\hline OP\_STNRE   & $x$ $\longrightarrow$ 0.0 \\
\hline OP\_ItoR    & $n$ $\longrightarrow$ $n$.000000 \\
\hline OP\_BtoI    & $p$ $\longrightarrow$
                         $ \left\{ \begin{tabular}{c}
                                     0 if $\not p$ \\
                                     1 if $p$ \\
                                 \end{tabular}
                           \right. $ \\
\hline
\end{tabular}
\finishtab{Unary arithmetic operations in ALU}{tab:ALU-unarith}

Unary built-in function operators are additional unary operators which
are included in the ALU in order to more efficiently process built-in
functions in \acronym{ASSIST}.   These are
listed in Table \ref{tab:ALU-unbltn}.

\starttab
\begin{tabular}{|c|c|}
\hline OP\_SQRT     & SQRT($x$) \\
\hline OP\_EXP      & EXP($x$) \\
\hline OP\_LN       & LN($x$) \\
\hline OP\_ABS      & ABS($x$) \\
\hline OP\_SIN      & SIN($x$) \\
\hline OP\_COS      & COS($x$) \\
\hline OP\_TAN      & TAN($x$) \\
\hline OP\_ARCSIN   & ARCSIN($x$) \\
\hline OP\_ARCCOS   & ARCCOS($x$) \\
\hline OP\_ARCTAN   & ARCTAN($x$) \\
\hline OP\_FACT     & FACT($x$) \\
\hline OP\_GAM      & GAM($x$) \\
\hline OP\_SIZE     & SIZE($arr$) \\
\hline OP\_COUNT1   & COUNT1($x$) (identity function) \\
\hline OP\_IMIN1    & IMIN1($x$) (identity function) \\
\hline OP\_RMIN1    & RMIN1($x$) (identity function) \\
\hline OP\_IMAX1    & IMAX1($x$) (identity function) \\
\hline OP\_RMAX1    & RMAX1($x$) (identity function) \\
\hline OP\_ISUM1    & ISUM1($x$) (identity function) \\
\hline OP\_RSUM1    & RSUM1($x$) (identity function) \\
\hline OP\_ANY1     & ANY1($x$) (identity function) \\
\hline OP\_ALL1     & ALL1($x$) (identity function) \\
\hline OP\_COUNT    & COUNT($arr$) \\
\hline OP\_IMIN     & IMIN($arr$) \\
\hline OP\_RMIN     & RMIN($arr$) \\
\hline OP\_IMAX     & IMAX($arr$) \\
\hline OP\_RMAX     & RMAX($arr$) \\
\hline OP\_ISUM     & ISUM($arr$) \\
\hline OP\_RSUM     & RSUM($arr$) \\
\hline OP\_ANY      & ANY($arr$) \\
\hline OP\_ALL      & ALL($arr$) \\
\hline
\end{tabular}
\finishtab{Unary arithmetic operations in ALU}{tab:ALU-unbltn}

Binary built-in function operations are additional binary operators which
are included in the ALU in order to more efficiently process built-in
functions in \acronym{ASSIST}.   These are
listed in Table \ref{tab:ALU-binbltn}.   Note that the wildcard row/column
operations are binary because, in postfix, they act upon a pointer into
the identifier table followed by the value of the column or row subscript
which will remain fixed during the summation.

\starttab
\begin{tabular}{|c|c|}
\hline OP\_COMB       & COMB($n$,$k$) \\
\hline OP\_PERM       & PERM($n$,$k$) \\
\hline OP\_ROWCOUNT   & COUNT($arr \char91 i , * \char93$) \\
\hline OP\_COLCOUNT   & COUNT($arr \char91 * , i \char93$) \\
\hline OP\_IROWMIN    & MIN($intarr \char91 i , * \char93$) \\
\hline OP\_ICOLMIN    & MIN($intarr \char91 * , i \char93$) \\
\hline OP\_RROWMIN    & MIN($realarr \char91 i , * \char93$) \\
\hline OP\_RCOLMIN    & MIN($realarr \char91 * , i \char93$) \\
\hline OP\_IROWMAX    & MAX($intarr \char91 i , * \char93$) \\
\hline OP\_ICOLMAX    & MAX($intarr \char91 * , i \char93$) \\
\hline OP\_RROWMAX    & MAX($realarr \char91 i , * \char93$) \\
\hline OP\_RCOLMAX    & MAX($realarr \char91 * , i \char93$) \\
\hline OP\_IROWSUM    & SUM($intarr \char91 i , * \char93$) \\
\hline OP\_ICOLSUM    & SUM($intarr \char91 * , i \char93$) \\
\hline OP\_RROWSUM    & SUM($realarr \char91 i , * \char93$) \\
\hline OP\_RCOLSUM    & SUM($realarr \char91 * , i \char93$) \\
\hline OP\_ROWANY     & ANY($boolarr \char91 i , * \char93$) \\
\hline OP\_COLANY     & ANY($boolarr \char91 * , i \char93$) \\
\hline OP\_ROWALL     & ALL($boolarr \char91 i , * \char93$) \\
\hline OP\_COLALL     & ALL($boolarr \char91 * , i \char93$) \\
\hline
\end{tabular}
\finishtab{Binary arithmetic operations in ALU}{tab:ALU-binbltn}

There are some list functions which take an undetermined number of parameters.
The postfix notation for these functions is a list of parameters followed by
a count of the number of parameters followed by the list operator itself.
For example:
\begin{codeexample}
SUM(6,ARRAY,18)
\end{codeexample}
would be represented as the postfix:
\begin{codeexample}
Operands:  6,ARRAY,18,3
Operators: OP\_VAL,OP\_VAL,OP\_VAL,OP\_INSVAL,OP\_ILISSUM
\end{codeexample}
The list function operators are listed in Table \ref{tab:ALU-list}.

\starttab
\begin{tabular}{|c|c|}
\hline OP\_LISCOUNT & COUNT($p_{1}$,$p_{2}$,...,$p_{n}$) \\
\hline OP\_ILISMIN  & MIN($i_{1}$,$i_{2}$,...,$i_{n}$) \\
\hline OP\_RLISMIN  & MIN($x_{1}$,$x_{2}$,...,$x_{n}$) \\
\hline OP\_ILISMAX  & MAX($i_{1}$,$i_{2}$,...,$i_{n}$) \\
\hline OP\_RLISMAX  & MAX($x_{1}$,$x_{2}$,...,$x_{n}$) \\
\hline OP\_ILISSUM  & SUM($i_{1}$,$i_{2}$,...,$i_{n}$) \\
\hline OP\_RLISSUM  & SUM($x_{1}$,$x_{2}$,...,$x_{n}$) \\
\hline OP\_LISANY   & ANY($p_{1}$,$p_{2}$,...,$p_{n}$) \\
\hline OP\_LISALL   & ALL($p_{1}$,$p_{2}$,...,$p_{n}$) \\
\hline
\end{tabular}
\finishtab{List operations in ALU}{tab:ALU-list}

The binary variable concatenation construction operation is listed in
Table \ref{tab:ALU-concat}.

\starttab
\begin{tabular}{|c|c|}
\hline OP\_CONCAT   & $x$ $\wedge$ $n$ \\
\hline
\end{tabular}
\finishtab{Concatenation operations in ALU}{tab:ALU-concat}

There are also some operations which push standard values such as zero
and one onto the evaluation stack.  These operations are faster than having
to look up the value in memory.   These are listed in Table \ref{tab:ALU-push}.

\starttab
\begin{tabular}{|c|c|}
\hline OP\_PZ       & Push binary zeros (0,FALSE) \\
\hline OP\_PRZ      & Push real zero (0.0000000000) \\
\hline OP\_PU       & Push integer unity (1) \\
\hline OP\_PBU      & Push Boolean unity (TRUE) \\
\hline OP\_PRU      & Push real unity (1.0000000000) \\
\hline OP\_NIX      & No operation \\
\hline
\end{tabular}
\finishtab{Standard value push operations in ALU}{tab:ALU-push}

\section{Data Structures used to parse ASSIST Source Code}
\label{chap:parsedata}

In order to understand the internals of \acronym{ASSIST}, it is necessary
to understand the data structures used by the program and some rationale
about why they are set up the way they are.   It is the purpose of this
chapter to provide a basis for this understanding.

The sections in this chapter are organized topically instead of alphabetically.
Some sections therefore reference more than one data structure and others
may not reference any data structures.   All data structures used by
\acronym{ASSIST} are referenced at least once.  Some data structures
are referenced more than once.

\subsection{The source code input line data structure}
\label{sec:srcline}

Each source code input line is stored in a structure of the following type:
\begin{codeexample}
typedef struct t__source_line_info
   \{
        short last_line_in_error;
        short last_line_in_warning;
        short error_count_this_line;
        short warning_count_this_line;
        short old_line_number;
        short line_number;
        Boolean line_shown_on_screen;
        scanning_character_info_type char_pair;
        char old_line_buffer[LINE_MAXNCH_P];
        char line_buffer[LINE_MAXNCH_P];
        short line_buffer_ix;
        Boolean must_fudge_it;
   \} source_line_info_type;
\end{codeexample}


\subsection{The identifier table data structure}
\label{sec:idinfo}

The identifier table is stored as an array of elements of the following type:
\begin{codeexample}
typedef struct t__identifier_info
  \{
     pointer_union_type ptr;    /* address in memory / function-parm-count */
     union
        \{
           struct qqbothidinfqq
              \{
                 dim_pair_type first;
                 dim_pair_type second;
              \} dims;
           dim_pair_type body;
        \} index;
     short scope_level;         /* scope level (negative iff. inactive) */
     char name[IDENT_MAXNCH_P]; /* identifier to search for */
     type_flagword_type flags;  /* type information */
  \} identifier_info_type;
\end{codeexample}
where the data structure for a dimension pair is of the following type:
\begin{codeexample}
typedef struct t__dim_pair
  \{
    Subscript lower;/* lower dimension (ARRAY) body-index (FUNCTION/IMPLICIT)*/
    Subscript upper;/* upper dimension (ARRAY) body-index (FUNCTION/IMPLICIT)*/
  \} dim_pair_type;
\end{codeexample}

and where the data structure for a pointer union is of the following type:
\begin{codeexample}
typedef union t__pointer_union
   \{
     relative_address_type relative_address;
     short parameter_count;
     void *vvv;               /* included for completeness */
     Boolean *bbb;            /* used when BOOL_TYPE */
     state_offset_type *sss;  /* used when SSVAR_TYPE */
     char *ccc;               /* used when CHAR_TYPE */
     int_type *iii;           /* used when INT_TYPE */
     real_type *rrr;          /* used when REAL_TYPE */
   \} pointer_union_type;
\end{codeexample}



The ``ptr'' field points to the data in memory for the identifier in question.
If the identifier corresponds to a state-space variable, then the data is 
an offset into and a bit-string length within the current state string.

The ``scope\_level'' field is used when parsing \rw{FOR} and block
\rw{IF} structures to
ascertain that the same variable name is not used for nested
structures and to ascertain that the scope of a variable is still active.  It
is also used to keep track of formal parameters in macro definition bodies.

For example, the following is \inerror{illegal}
because the scope of \word{III} would still be active when it was
being re-defined in the nested FOR loop:
\begin{codeexample}
FOR III IN [1..10]
    FOR III IN [1..2]
    ENDFOR
ENDFOR
\end{codeexample}

Also, the following
would be \inerror{illegal}
because the scope of \word{III} is no longer active after the \rw{ENDFOR}
has terminated the loop.
\begin{codeexample}
FOR III IN [1..10]
    ...
ENDFOR
IF (III>7) ...
\end{codeexample}


The ``index'' field is used for the subscript bound(s)
if the identifier
is for an array constant or an array state-space variable.
It points to
the beginning of the body of
a macro (\rw{IMPLICIT} variables and \rw{FUNCTION}) definition when the
identifier is for the name of a function.
For Arrays, both the ``dims.first'' and ``dims.second'' sub-fields are used for
the first and second doubly subscripted array bounds, respectively.
If the values stored in a bound are equal to
the special ``\word{SIMPLE\_IDENTIFIER}'' value, then that bound does not
apply.   For macros, only the ``body'' sub-field applies.   Note that
the ``dims.first'' and ``dims.second'' sub-fields are sequential
in memory whereas the ``body'' subfield overlays the ``dims'' subfield.
This is done to save memory because an identifier will either be an array
or a macro but never both.

Note that an \rw{IMPLICIT} array
is implemented as a macro, not as an array.  Bounds do not apply since
substituted parameters are not checked for syntactical/semantic correctness
until actually parsed.   The count of the number of parameters expected for a
macro is stored in the ``ptr.parameter\_count'' field.

The ``lower'' sub-field of either the ``dims.first'' or ``dims.second''
field is used for the lower subscript
bound if the identifier
is for an array constant or an array state-space variable.

The ``lower'' sub-field of the ``body'' field points to the first
token for the body of a \rw{FUNCTION} or \rw{IMPLICIT} definition when the
identifier is for the name of a macro.

The ``upper'' sub-field of either the ``dims.first'' or ``dims.second''
field is used for the upper subscript
bound if the identifier
is for an array constant or an array state-space variable.

The ``upper'' sub-field of the ``body'' field points one
token past the end of the body of a \rw{FUNCTION} or \rw{IMPLICIT} definition when the
identifier is for the name of a macro.   This may or may not point to a valid
token.  The test ``token-pointer < upper'' on a while loop will guarantee
that the correct tokens for a body will be accessed.

The ``name'' field is used to store the name of the identifier.   This field
is 32 characters long with 4 characters reserved for the implementation and
28 character are reserved for the user.   Identifier names can therefore
be at most 28 characters long.    Literal number strings are also stored
in the identifier name field so that they can be easily printed when printing
the rate expressions to the model file.   If an identifier is a literal number
string, it will begin with a pound sign (\#).   A digit string can be at most
28 characters long.   The number ``-6.023000000000000000000E+23'' is therefore
legal but ``-6.0230000000000000000000E+23'' is illegal because it is one digit
too long.   Since ``real\_type'' is defined to be ``double'' when floats are
only 32 bits long and is defined to be ``float'' when floats are 60 or 64
bits long, the precision of the machine allows for only about 12-18 significant
digits at most, so this restriction of 20 decimal places in
scientific notation should not pose any serious limitations.

The ``flags'' field is a string of 8 bits which are packed as follows:
\starttab
\begin{tabular}{|c|c|}
\hline
bit(s) & interpretation \\
\hline
\hline
0-2    & computational type (char,bool,int,real, or state-offset) \\
3      & expression result variable or IMPLICIT variable \\
4      & FOR loop index variable \\
5      & state-space variable \\
6      & FUNCTION (IMPLICIT variable if bit 3 is also set) \\
7      & array constant or array variable \\
\hline
\end{tabular}
\finishtab{How bits are packed for the ``type flagword type''}{tab:typeflag}

unless at least one of the variable bits (bits 3,4,5,6) is set,
the identifier will correspond
to a constant (either a named constant or a literal value).

Several examples of how identifiers are laid out in memory are
incorporated into Figure \ref{fig:space-mem} on page \pageref{fig:space-mem}
and into Figure \ref{fig:space2-mem} on page \pageref{fig:space2-mem}

\subsection{The state offset data structure}
\label{sec:soff}

Current and new state offset information is stored in data
of the following type:
\begin{codeexample}
typedef short ssvar_value_type;
typedef struct t__state_offset
   \{
      ssvar_value_type minval;
      ssvar_value_type maxval;
      bitsize_type bit_offset;
      bitsize_type bit_length;
   \} state_offset_type;
\end{codeexample}


Two bit strings
are maintained during rule generation, namely the source and destination
states.   The offset and length of a particular state are used to locate
the value for a specific state-space variable within the state in question.

The ``minval'' field specifies the minimum value that a state
space variable is allowed to take on.

The ``maxval'' field specifies the maximum value that a state
space variable is allowed to take on.  The difference ``maxval$~-~$minval''
can be at most 255.

The ``bit\_offset'' field specifies the offset into the bit string where
the packing/unpacking of a state-space variable begins.

The ``bit\_length'' field specifies the length of the packed state-space
variable in number of bits.   The length can be at most 8.   The value
actually packed into the space is not the value itself but rather the
difference ``actualvalue$~-~$minval''.

Several examples of how state offsets are laid out in memory are
incorporated into Figure \ref{fig:space-mem} on page \pageref{fig:space-mem}
and into Figure \ref{fig:space2-mem} on page \pageref{fig:space2-mem}

\subsection{The token information data structure}
\label{sec:tokinf}

The lexical scanner translates an input text file into a sequence of tokens.
Each token is stored in an element of the following data type:
\begin{codeexample}
typedef struct t__token_info
   {
      identifier_info_type *id_info_ptr;
      short linnum;
      short pos;
      token tok;
      rwtype rw;
      char id[IDENT_MAXNCH_P];
   } token_info_type;
\end{codeexample}


The ``id\_info\_ptr'' field is a pointer to the corresponding entry in the
identifier table.

The ``linnum'' field contains the line number on which the token occurred.

The ``pos'' field contains the column number (indexed beginning with zero)
at which the first character of the token appears.

The ``tok'' field contains the token itself.   Examples of tokens
are ``TK\_RW'' for a reserved word, ``TK\_ID'' for an identifier, ``TK\_REAL''
for a literal real value, ``TK\_SUB'' for the subtraction ``$-$'' token, etc.
Tokens are defined in the ``tokdefs.h'' file.

The ``rw'' field contains ``RW\_NULL'' unless the ``tok'' field is ``TK\_RW''
in which case it contains the value corresponding to the reserved word.
Examples are ``RW\_TRANTO'', ``RW\_START'', ``RW\_BY'', ``RW\_IN'', etc.
Reserved words are defined in the ``rwdefs.h'' file.

The ``id'' field contains the characters string for the token itself.  If the
token is a formal macro parameter, then this string begins with a dollar
(\$) sign followed by the encoded formal parameter number (beginning with one).
If the token is a literal value, then this string begins with a pound (\#)
sign followed by the character string for the number as it was typed into the
input file.
If the token is a literal character string value (as in an INPUT prompt
message), then this string contains ``\verb|#""|''.
If the token is an identifier, then this string contains the
name of the identifier.

Several examples of tokens laid out in memory are
incorporated into Figure \ref{fig:func-mem} on page \pageref{fig:func-mem}
and into Figure xxx on page xxx.

\subsection{The expression data structure}
\label{sec:expr}

Expressions occur in the syntax of many different statements and
clauses in the \acronym{ASSIST} language.
Among these are the \rw{SPACE}, \rw{START}, \rw{ASSERT},
\rw{DEATHIF}, \rw{PRUNEIF}, \rw{IF}, \rw{FOR}, and constant
definition statements.

Expressions results can be either whole, real, or Boolean.   The same data
structures are used regardless of the evaluation type of the expression.
The following data structures are used:
\begin{codeexample}
typedef struct t__expression
   \{
       operation_type *postfix_ops;
       operation_type *infix_ops;
       operand_type *operands;
       short n_postfix_ops;
       short n_infix_ops;
       short n_operands;
       short source_code_line_number;
       Boolean in_error;
       type_flagword_type rtntype;
   \} expression_type;
\end{codeexample}


The ``postfix\_ops'' field is a pointer to an array of postfix operations and
operators.   If an element in the array is an \word{OP\_VAL} or
an \word{OP\_INSVAL}, denoted as ``V'' and ``\_V\_'', respectively, then
a value is taken from
the operand array, otherwise element indicates either an arithmetic or
Boolean operation.

The ``infix\_ops'' field is a pointer to an array of infix operations and
operators.  It is similar to ``postfix\_ops'' except that it is used for
printing expressions instead of evaluating them.

The ``operands'' field is a pointer to an array of operands.   Each operand
is a pointer into the identifier table.

The ``n\_postfix\_ops'' field specifies the length of the ``postfix\_ops''
array.

The ``n\_infix\_ops'' field specifies the length of the ``infix\_ops''
array.

The ``n\_operands'' field specifies the length of the ``operands''
array.

The ``source\_code\_line\_number'' indicates the line number in the source
code (\extent{.ast}) file as listed in the log (\extent{.alog}) file
where the expression began.   It is used to print intelligent
error messages during both parsing and model generation phases.

The ``in\_error;'' field indicates that an error was detected while attempting
to parse an expression.   An attempt to evaluate an expression which
is ``in\_error'' may result in a core dump or a fatal traceback error.
The expression evaluator therefore returns the default value when attempting to
evaluate an expression in error.

The ``rtntype'' field indicates the type flags for the return value of an
expression.

As an example, consider the following expression:
\begin{logfileexample}
(00174): ... NELE + ( 4 - 2*NIX ) / 1.0 ** MU .....
\end{logfileexample}

Figure \ref{fig:expr-mem} illustrates how the
above expression will be stored in memory.

\startfig
\begin{fast_picture}{500}{220}
\setboxpos{50}{180}\savFboxpos
\leftwidetagbox{postfix}{}{72}{19}\outarrow{44}
         \setboxsizes{15}{72}
         \putboxafterarrow
             \valbox{V}\boxtoright\valbox{I$\rightarrow$R}\boxtoright\valbox{V}\boxtoright
             \valbox{V}\boxtoright\valbox{V}\boxtoright\valbox{*}\boxtoright
             \valbox{$-$}\boxtoright\valbox{I$\rightarrow$R}\boxtoright\valbox{V}\boxtoright
             \valbox{V}\boxtoright\valbox{R**R}\boxtoright\valbox{/}\boxtoright
             \valbox{$+$}
         \setboxsizes{9}{72}
\nextFbox\leftwidetagbox{infix}{}{72}{19}\outarrowdownright{31}{10}{13}
         \setboxsizes{15}{72}
         \putboxafterarrow
             \valbox{V}\boxtoright\valbox{$+$}\boxtoright\valbox{(}\boxtoright
             \valbox{V}\boxtoright\valbox{$-$}\boxtoright\valbox{V}\boxtoright
             \valbox{*}\boxtoright\valbox{V}\boxtoright\valbox{)}\boxtoright
             \valbox{/}\boxtoright\valbox{V}\boxtoright\valbox{**}\boxtoright
             \valbox{V}
         \setboxsizes{9}{72}
\nextFbox\leftwidetagbox{operands}{}{72}{19}
         \outarrowdownright{31}{20}{80}
         \stackDboxtoright
         \nextDbox\valbox{}\outarrowmsg{30}{NELE}
         \nextDbox\valbox{}\outarrowmsg{30}{\#4}
         \nextDbox\valbox{}\outarrowmsg{30}{\#2}
         \nextDbox\valbox{}\outarrowmsg{30}{NIX}
         \nextDbox\valbox{}\outarrowmsg{30}{\#1.0}
         \nextDbox\valbox{}\outarrowmsg{30}{MU}
\nextFbox\leftwidetagbox{\#~post}{13}{72}{19}
\nextFbox\leftwidetagbox{\#~in}{13}{72}{19}
\nextFbox\leftwidetagbox{\#~oper}{6}{72}{19}
\nextFbox\leftwidetagbox{line~\#}{174}{72}{19}
\nextFbox\leftwidetagbox{err~?}{FALSE}{72}{19}
\nextFbox\leftwidetagbox{return}{real}{72}{19}
\end{fast_picture}
\finishfig{Sample expression laid out in memory}{fig:expr-mem}

where ``I$\rightarrow$R'' stands for an explicit integer-to-real conversion
and where ``R**R'' stands for a real number raised to a real power.

When a postfix expression is evaluated, the next value is taken from the
``operands'' list whenever an \word{OP\_VAL} or an \word{OP\_INSVAL}
postfix operation is encountered.  These two operations are
used to instruct the evaluation that an operand comes next.  Any other
symbol is consequently interpreted as either an arithmetic or Boolean
operation.

The lists work in a similar manner when an infix expression is printed.

The postfix list is used when an expression is evaluated.   Expressions are
evaluated in START and FOR statements.   Boolean expressions are evaluated
during rule generation whenever a test (such as a DEATHIF, IF, ASSERT, etc.)
is made.

The infix list is used when an expression is printed.   Expressions are
printed if the user asks for a map with the VMS \vmsoption{MAP} or the UNIX
\unixoption{MAP} command line option.
Expressions are also printed for certain
``\word{DEBUG\$}'' options, such as ``\word{DEBUG\$ PARSE\$}''.
Rate expressions are
printed to the model file.


\subsection{The SPACE statement data structure}
\label{sec:space}

The \rw{SPACE} statement is parsed
and stored in data structures of the following types:
\begin{codeexample}
typedef struct t__state_space_picture
   \{
       vars_union_type *varu;
       Boolean *is_nested;
       short nvaru;
   \} state_space_picture_type;
\end{codeexample}

\begin{codeexample}
typedef union t__vars_union
   \{
       identifier_info_type *id_info;
       state_space_picture_type *nested_space_picture;
       relative_address_type relative_address;
   \} vars_union_type;
\end{codeexample}


\begin{codeexample}
typedef struct t__identifier_info
  \{
     pointer_union_type ptr;    /* address in memory / function-parm-count */
     union
        \{
           struct qqbothidinfqq
              \{
                 dim_pair_type first;
                 dim_pair_type second;
              \} dims;
           dim_pair_type body;
        \} index;
     short scope_level;         /* scope level (negative iff. inactive) */
     char name[IDENT_MAXNCH_P]; /* identifier to search for */
     type_flagword_type flags;  /* type information */
  \} identifier_info_type;
\end{codeexample}
where the data structure for a dimension pair is of the following type:
\begin{codeexample}
typedef struct t__dim_pair
  \{
    Subscript lower;/* lower dimension (ARRAY) body-index (FUNCTION/IMPLICIT)*/
    Subscript upper;/* upper dimension (ARRAY) body-index (FUNCTION/IMPLICIT)*/
  \} dim_pair_type;
\end{codeexample}

and where the data structure for a pointer union is of the following type:
\begin{codeexample}
typedef union t__pointer_union
   \{
     relative_address_type relative_address;
     short parameter_count;
     void *vvv;               /* included for completeness */
     Boolean *bbb;            /* used when BOOL_TYPE */
     state_offset_type *sss;  /* used when SSVAR_TYPE */
     char *ccc;               /* used when CHAR_TYPE */
     int_type *iii;           /* used when INT_TYPE */
     real_type *rrr;          /* used when REAL_TYPE */
   \} pointer_union_type;
\end{codeexample}


\begin{codeexample}
typedef short ssvar_value_type;
typedef struct t__state_offset
   \{
      ssvar_value_type minval;
      ssvar_value_type maxval;
      bitsize_type bit_offset;
      bitsize_type bit_length;
   \} state_offset_type;
\end{codeexample}


The ``varu'' field is a pointer to an array of pointers.   Each pointer in
the array of pointers points to either an element in the identifier table
or a nested state-space picture, depending upon whether the ``is\_nested''
element is \rw{FALSE} or \rw{TRUE}, respectively.

The ``is\_nested'' field is a pointer to an array of Booleans which indicate
whether the corresponding position in the picture is a variable or a nested
state-space picture.

The ``nvaru'' field gives a count of the number of items in the ``varu'' array
which is the same as the number of items in the ``is\_nested'' array.

The ``id\_info'' field of the pointer union is used when ``is\_nested'' is
FALSE.  The ``nested\_space\_picture'' field of the pointer union is used
when ``is\_nested'' is TRUE.   The ``relative\_address'' field is used when
the picture is stored in the object (.aobj) file while memory is freed and
re-allocated to conserve space.

The other two structures are described in detail in the preceding sections.

As an example, consider the following recursively nested \rw{SPACE} statement:
\begin{logfileexample}
(00009): NR = 2; \\
(00010): SPACE = (NP,NFP,(UR:1..NR,UX:ARRAY[1..NR] OF BOOLEAN)); \\
\end{logfileexample}

Figure \ref{fig:space-mem} illustrates how the
above \rw{SPACE} statement will be
stored in memory.

\startfig
\begin{fast_picture}{500}{340}
\setboxpos{40}{300}\savFboxpos
\lefttagbox{vars}{}\outarrowdown{160}{20}
      \stackBbox
      \nextBbox\lefttagbox{NP}{}\outarrowdown{200}{20}
          \stackwideCbox{23}
          \nextCbox\leftwidetagbox{ptr}{}{72}{23}\outarrowdown{40}{20}
               \stackDbox
               \nextDbox\righttagbox{0}{min}
               \nextDbox\righttagbox{255}{max}
               \nextDbox\righttagbox{0}{offset}
               \nextDbox\righttagbox{8}{length}
          \nextCbox\leftwidetagbox{dims.first}{simple}{62}{23}
          \nextCbox\leftwidetagbox{.second}{simple}{72}{23}
          \nextCbox\leftwidetagbox{scope}{n/a}{72}{23}
          \nextCbox\leftwidetagbox{name}{``NP''}{72}{23}
          \nextCbox\leftwidetagbox{flags}{int~,~ssv}{72}{23}
      \nextBbox\lefttagbox{NFP}{}\outarrowdownrightdown{110}{135}{90}{15}
          \stackwideCbox{23}
          \nextCbox\leftwidetagbox{ptr}{}{72}{23}\outarrowdown{40}{20}
               \stackDbox
               \nextDbox\righttagbox{0}{min}
               \nextDbox\righttagbox{255}{max}
               \nextDbox\righttagbox{8}{offset}
               \nextDbox\righttagbox{8}{length}
          \nextCbox\leftwidetagbox{dims.first}{simple}{72}{23}
          \nextCbox\leftwidetagbox{.second}{simple}{72}{23}
          \nextCbox\leftwidetagbox{scope}{n/a}{72}{23}
          \nextCbox\leftwidetagbox{name}{``NFP''}{72}{23}
          \nextCbox\leftwidetagbox{flags}{int~,~ssv}{72}{23}
      \nextBbox\lefttagbox{()}{}\outarrowdownleftdown{20}{25}{180}{10}
   \stackAbox
   \nextAbox\lefttagbox{vars}{}\outarrowdown{135}{20}
      \stackBbox
      \nextBbox\lefttagbox{UR}{}\outarrowdown{85}{55}
          \stackwideCbox{23}
          \nextCbox\leftwidetagbox{ptr}{}{72}{23}\outarrowdown{40}{20}
               \stackDbox
               \nextDbox\righttagbox{1}{min}
               \nextDbox\righttagbox{2}{max}
               \nextDbox\righttagbox{16}{offset}
               \nextDbox\righttagbox{1}{length}
          \nextCbox\leftwidetagbox{dims.first}{simple}{72}{23}
          \nextCbox\leftwidetagbox{.second}{simple}{72}{23}
          \nextCbox\leftwidetagbox{scope}{n/a}{72}{23}
          \nextCbox\leftwidetagbox{name}{``UR''}{72}{23}
          \nextCbox\leftwidetagbox{flags}{int~,~ssv}{72}{23}
      \nextBbox\lefttagbox{UX[*]}{}\outarrowdownleftdown{25}{25}{130}{10}
          \stackwideCbox{44}
          \nextCbox\leftwidetagbox{ptr}{}{72}{44}\outarrowdown{70}{20}
               \stackwideDbox{19}
               \nextDbox\rightwidetagbox{FALSE}{min}{19}{72}
               \nextDbox\rightwidetagbox{TRUE}{max}{19}{72}
               \nextDbox\rightwidetagbox{17}{offset}{19}{72}
               \nextDbox\rightwidetagbox{1}{length}{19}{72}
          \nextCbox\leftwidetagbox{dims.first}{1..2}{72}{44}
          \nextCbox\leftwidetagbox{.second}{simple}{72}{44}
          \nextCbox\leftwidetagbox{scope}{n/a}{72}{44}
          \nextCbox\leftwidetagbox{name}{``UX''}{72}{44}
          \nextCbox\leftwidetagbox{flags}{bool~,~ssv~,~array}{72}{44}
   \nextAbox\lefttagbox{nested?}{}\outarrowdown{50}{10}
      \stackwideBbox{19}
      \nextBbox\leftwidetagbox{UR}{FALSE}{72}{19}
      \nextBbox\leftwidetagbox{UX[*]}{FALSE}{72}{19}
   \nextAbox\lefttagbox{\#~vars}{2}
\nextFbox\lefttagbox{nested?}{}\outarrowdown{60}{10}
      \stackwideBbox{19}
      \nextBbox\leftwidetagbox{NP}{FALSE}{72}{19}
      \nextBbox\leftwidetagbox{NFP}{FALSE}{72}{19}
      \nextBbox\leftwidetagbox{()}{TRUE}{72}{19}
\nextFbox\lefttagbox{\#~vars}{3}
\end{fast_picture}
\finishfig{SPACE statement laid out in memory}{fig:space-mem}

For another example of a \rw{SPACE} statement, consider:
\begin{logfileexample}
(00009): NPMAX = 5;
(00010): SPACE = (NP:NPMAX..NPMAX,NFP:0..NPMAX,Q:BOOLEAN,
                  ELE:ARRAY[21..30] OF 20..23,NELE:0..10)
\end{logfileexample}

Figure \ref{fig:space2-mem} illustrates how the above SPACE statement will be
stored in memory.

\startfig
\begin{fast_picture}{500}{350}
\setboxpos{40}{330}\savFboxpos
\lefttagbox{vars}{}\outarrowdown{120}{30}
      \stackBbox
      \nextBbox\lefttagbox{NP}{}\outarrowdown{240}{30}
          \stackwideCbox{23}
          \nextCbox\leftwidetagbox{ptr}{}{72}{23}\outarrowdown{40}{20}
               \stackDbox
               \nextDbox\righttagbox{5}{min}
               \nextDbox\righttagbox{5}{max}
               \nextDbox\righttagbox{0}{offset}
               \nextDbox\righttagbox{1}{length}
          \nextCbox\leftwidetagbox{dims.first}{simple}{72}{23}
          \nextCbox\leftwidetagbox{.second}{simple}{72}{23}
          \nextCbox\leftwidetagbox{scope}{n/a}{72}{23}
          \nextCbox\leftwidetagbox{name}{``NP''}{72}{23}
          \nextCbox\leftwidetagbox{flags}{int~,~ssv}{72}{23}
      \nextBbox\lefttagbox{NFP}{}\outarrowdown{100}{10}
          \stackwideCbox{23}
          \nextCbox\leftwidetagbox{ptr}{}{72}{23}\outarrowdown{40}{20}
               \stackDbox
               \nextDbox\righttagbox{0}{min}
               \nextDbox\righttagbox{5}{max}
               \nextDbox\righttagbox{1}{offset}
               \nextDbox\righttagbox{3}{length}
          \nextCbox\leftwidetagbox{dims.first}{simple}{72}{23}
          \nextCbox\leftwidetagbox{.second}{simple}{72}{23}
          \nextCbox\leftwidetagbox{scope}{n/a}{72}{23}
          \nextCbox\leftwidetagbox{name}{``NFP''}{72}{23}
          \nextCbox\leftwidetagbox{flags}{int~,~ssv}{72}{23}
      \nextBbox\lefttagbox{Q}{}\outarrowdownrightdown{25}{120}{200}{20}
          \stackwideCbox{23}
          \nextCbox\leftwidetagbox{ptr}{}{72}{23}\outarrowdown{46}{20}
               \stackwideDbox{19}
               \nextDbox\rightwidetagbox{FALSE}{min}{19}{72}
               \nextDbox\rightwidetagbox{TRUE}{max}{19}{72}
               \nextDbox\rightwidetagbox{4}{offset}{19}{72}
               \nextDbox\rightwidetagbox{1}{length}{19}{72}
          \nextCbox\leftwidetagbox{dims.first}{simple}{72}{23}
          \nextCbox\leftwidetagbox{.second}{simple}{72}{23}
          \nextCbox\leftwidetagbox{scope}{n/a}{72}{23}
          \nextCbox\leftwidetagbox{name}{``Q''}{72}{23}
          \nextCbox\leftwidetagbox{flags}{bool~,~ssv}{72}{23}
      \nextBbox\lefttagbox{ELE[*]}{}\outarrowdownrightdown{20}{110}{40}{10}
          \stackwideCbox{41}
          \nextCbox\leftwidetagbox{ptr}{}{72}{41}\outarrowdown{55}{20}
               \stackDbox
               \nextDbox\righttagbox{20}{min}
               \nextDbox\righttagbox{23}{max}
               \nextDbox\righttagbox{5}{offset}
               \nextDbox\righttagbox{2}{length}
          \nextCbox\leftwidetagbox{dims.first}{21..30}{72}{41}
          \nextCbox\leftwidetagbox{.second}{simple}{72}{41}
          \nextCbox\leftwidetagbox{scope}{n/a}{72}{41}
          \nextCbox\leftwidetagbox{name}{``ELE''}{72}{41}
          \nextCbox\leftwidetagbox{flags}{int~,~ssv~,~array}{72}{41}
      \nextBbox\lefttagbox{NELE}{}\outarrowdownleftdown{15}{50}{125}{50}
          \stackwideCbox{23}\stackwideDbox{23}
          \nextCbox\leftwidetagbox{ptr}{}{72}{23}\outarrowdown{40}{20}
               \stackDbox
               \nextDbox\righttagbox{0}{min}
               \nextDbox\righttagbox{10}{max}
               \nextDbox\righttagbox{25}{offset}
               \nextDbox\righttagbox{4}{length}
          \nextCbox\leftwidetagbox{dims.first}{simple}{72}{23}
          \nextCbox\leftwidetagbox{.second}{simple}{72}{23}
          \nextCbox\leftwidetagbox{scope}{n/a}{72}{23}
          \nextCbox\leftwidetagbox{name}{``NELE''}{72}{23}
          \nextCbox\leftwidetagbox{flags}{int~,~ssv}{72}{23}
\nextFbox\lefttagbox{nested?}{}\outarrowdown{50}{30}
      \stackwideBbox{19}
      \nextBbox\leftwidetagbox{NP}{FALSE}{72}{19}
      \nextBbox\leftwidetagbox{NFP}{FALSE}{72}{19}
      \nextBbox\leftwidetagbox{Q}{FALSE}{72}{19}
      \nextBbox\leftwidetagbox{ELE[*]}{FALSE}{72}{19}
      \nextBbox\leftwidetagbox{NELE}{FALSE}{72}{19}
\nextFbox\lefttagbox{\#~vars}{4}
\end{fast_picture}
\finishfig{SPACE statement laid out in memory}{fig:space2-mem}

Notice that, in Figure \ref{fig:space2-mem} that,
the bit length for NP was one even
though the value for NP is 5.   Although the number 5 cannot be
stored in one bit, the range 5..5 is of length one and the number 1 can be
stored in one bit.
Also note that ELE[~] is said to be of length 2 because each element of ELE[~]
is of length 2.  The array actually occupies 20 bits, two bits for each of the
ten elements in the array.   If the offset of ELE[~] which is 5 is compared to
the offset of NELE which is 25, the difference is 20 which is correct.


\subsection{The START statement data structure}
\label{sec:start}

The \rw{START} statement is parsed
and stored in a data structure of the following type:
\begin{codeexample}
typedef struct t__space_expression
   \{
       expression_type *exprs;
       operand_type *vars;
       short n_vars;
   \} space_expression_type;
\end{codeexample}


The ``exprs'' field is a pointer to an array of expressions.   The array
contains one expression for each scalar variable in the space,
two for each singly subscripted array variable, and three for each doubly
subscripted array variable.   For arrays, the subscripts always precede
the assigned value.

The ``vars'' field is a pointer to an array of pointers.   Each pointer in
the array of pointers points to an identifier in the identifier table.
Each of the ``vars'' corresponds to one scalar l-value.    There will be
one var for each scalar state-space variable in the space and one for each
element of each array in the space.

The ``n\_vars'' field is the count of the number
of scalar l-values in the \rw{START}
expression.  Since the empty-field is not
allowed in a \rw{START} statement, this
number should always equal the number of positions at its level in the nested
state space.

As an example, consider the same recursively nested \rw{SPACE} statement and
corresponding \rw{START} statement:
\begin{codeexample}
(00009): NR = 2;
(00010): SPACE = (NP,NFP,(UR:1..NR,UX:ARRAY[1..NR] OF BOOLEAN));
(00011): START = (1+2*NR,0,(2,NR OF NR<3));
\end{codeexample}
The layout of the above \rw{START} statement is illustrated
in Figure \ref{fig:start-mem}

\startfig
\begin{fast_picture}{500}{200}
\setboxpos{135}{170}\savFboxpos
\lefttagbox{exprs}{}\outarrowdown{212}{20}
      \stackwideBbox{28}
      \nextBbox\widevalbox{$1+2*NR$}{28}
      \nextBbox\widevalbox{$0$}{28}
      \nextBbox\widevalbox{$2$}{28}
      \nextBbox\widevalbox{$[1]$}{28}
      \nextBbox\widevalbox{$NR<3$}{28}
      \nextBbox\widevalbox{$[2]$}{28}
      \nextBbox\widevalbox{$NR<3$}{28}
\nextFbox\lefttagbox{vars}{}\outarrowdown{90}{20}
      \stackBbox
      \nextBbox\valbox{}\outarrowmsg{30}{NP}
      \nextBbox\valbox{}\outarrowmsg{30}{NFP}
      \nextBbox\valbox{}\outarrowmsg{30}{UR}
      \nextBbox\valbox{}\outarrowmsg{30}{UX}
      \nextBbox\valbox{}\outarrowmsg{30}{UX}
\nextFbox\lefttagbox{n\_vars}{5}
\end{fast_picture}
\finishfig{START statement laid out in memory}{fig:start-mem}

Notice that, although $n\_vars=5$, there are seven expressions in the
array pointed to by $exprs$.   This is because two of the variables in the
array pointed to by $vars$ are for \word{UX} which is an array
variable.   The array variable name \word{UX} is repeated in the $vars$
list as many times as there are elements in the array.

Two expressions are
stored in the array when an array variable
is encountered in a positional state node.
The first of the two expressions is always the
subscript and
the second is always the value to be stored in the state space.

After the start statement is parsed, an unconditional \rw{TRANTO} is generated
and stored for transition to the initial state.

\subsection{The current state bit string}
\label{sec:bitpack}

When transitions between states occur, one bit string each is kept for 
both the source and destination
states.   Bit string lengths are rounded up so that each bit string will
start on a byte boundary.   Bit strings are packed for optimum use of memory.
Register variables are used when packing and unpacking the bit strings to
achieve maximum speed yet retain portability of the code.  If
the \rw{ONEDEATH} option is \rw{OFF}, then an extra bit is added before
rounding to indicate a death state.

Bit strings are packed into an array pointed to by a pointer of the following
type:
\begin{codeexample}
typedef unsigned char *state_bitstring_type;
\end{codeexample}


Consider the following \rw{SPACE} statement:
\begin{logfileexample}
(00022): SPACE = (NP:0..6,NWP:3..6,Q:BOOLEAN,ELE:ARRAY[1..5] OF 10..25)
\end{logfileexample}

Two examples using the previous \rw{SPACE} statement are given in
Figures \ref{fig:bitpack-a} and \ref{fig:bitpack-b}.   Both of these
examples have an ``unused'' portion which is necessary to align the next
state on a byte boundary.   If death states are not aggregated (lumped),
then the first of these unused bits is used to flag the death states.
In models where a state-space node exactly fits with no unused space, an
extra byte is required for the flag bit when death states are not lumped.
The default is to lump death states (\rw{ONEDEATH ON}) according to the
\rw{DEATHIF} sequence in the input file.

\startfig
\begin{fast_picture}{500}{150}
\setboxsizes{7}{72}
\put(0,110){\makebox(500,20){$(6,5,TRUE,10,12,15,24,25)$}}
\setboxpos{10}{50}
    \savAboxpos\bitsinbyte{1}{1}{0}{1}{0}{1}{0}{0}
    \savBboxpos\bitsinbyte{0}{0}{0}{0}{1}{0}{0}{1}
    \savCboxpos\bitsinbyte{0}{1}{1}{1}{1}{0}{1}{1}
    \savDboxpos\bitsinbyte{1}{1}{}{}{}{}{}{}
    \bitdummylabel{5-3=2}{10-10=0}{12-10=2}{15-10=5}{24-10=14}{25-10=15}
\setboxsizes{9}{72}
\end{fast_picture}
\finishfig{Example illustrating packing of a state-space node}{fig:bitpack-a}

\startfig
\begin{fast_picture}{500}{150}
\setboxsizes{7}{72}
\put(0,110){\makebox(500,20){$(6,6,FALSE,16,25,12,25,11)$}}
\setboxpos{10}{50}
    \savAboxpos\bitsinbyte{1}{1}{0}{1}{1}{0}{0}{1}
    \savBboxpos\bitsinbyte{1}{0}{1}{1}{1}{1}{0}{0}
    \savCboxpos\bitsinbyte{1}{0}{1}{1}{1}{1}{0}{0}
    \savDboxpos\bitsinbyte{0}{1}{}{}{}{}{}{}
    \bitdummylabel{6-3=3}{16-10=6}{25-10=15}{12-10=2}{25-10=15}{11-10=1}
\setboxsizes{9}{72}
\end{fast_picture}
\finishfig{Another example of packing of a state-space node}{fig:bitpack-b}

Because the bound difference cannot exceed 32767 (fifteen bits) and because
a byte is eight bits, a single
state-space element cannot span more than three bytes.   Because of this
assumption, the current code to pack and unpack the state bits will have
to be modified to work correctly on 6-bit architectures such as CDC.

\subsection{Macro definitions and data structures}
\label{sec:macros}

There are two kinds of macro definitions in the ASSIST language.  The first
is the \rw{IMPLICIT} variable definition.   The second is the
\rw{FUNCTION} definition.   The data structures
for both of these are identical.
Their only
difference is that, after parsing the \rw{IMPLICIT} variable definition, the
formal parameter list is thrown away and the index list is retained.  The
formal parameter list is retained on the \rw{FUNCTION} definition.

The semantics for macro definitions disallows reference to variables in the
body of the definition except via the parameter list.   Named constants,
literal values, reserved words, and symbols may all be referenced in the
body token list without having to be listed in the parameter list.

An \rw{IMPLICIT} variable definition's parameter list may contain only
state-space variables.   Its index list may contain only dummy identifiers
which are not used as space variables or named constants.

A \rw{FUNCTION} definition's parameter list may contain only dummy identifiers
which are not used as space variables or named constants.

The macro expansion stack is used for both \rw{IMPLICIT} and \rw{FUNCTION}
definitions and consists of a stack of elements of the
following data type:
\begin{codeexample}
typedef struct t__macro_expansion_info
   \{
       token_info_type *passed_token_list;
       unsigned short *passed_token_offset;
       unsigned short *passed_token_counts;
       short now_passed_ix;
       short now_passed_count;
       short passed_parameter_count;
       short now_body_ix;
       short ovf_body_ix;
       short pos;
       short linnum;
   \} macro_expansion_info_type;
\end{codeexample}
where the data structure for a dimension pair is of the following type:
\begin{codeexample}
typedef struct t__dim_pair
  \{
    Subscript lower;/* lower dimension (ARRAY) body-index (FUNCTION/IMPLICIT)*/
    Subscript upper;/* upper dimension (ARRAY) body-index (FUNCTION/IMPLICIT)*/
  \} dim_pair_type;
\end{codeexample}

and where the data structure for a pointer union is of the following type:
\begin{codeexample}
typedef union t__pointer_union
   \{
     relative_address_type relative_address;
     short parameter_count;
     void *vvv;               /* included for completeness */
     Boolean *bbb;            /* used when BOOL_TYPE */
     state_offset_type *sss;  /* used when SSVAR_TYPE */
     char *ccc;               /* used when CHAR_TYPE */
     int_type *iii;           /* used when INT_TYPE */
     real_type *rrr;          /* used when REAL_TYPE */
   \} pointer_union_type;
\end{codeexample}



The first two sub-sections in this section deal with the definitions of
the two kinds of macros.  The third and last sub-section deals with how
macros are expanded when they are invoked.

\subsubsection{The IMPLICIT statement data structure}
\label{sec:implicit}

The \rw{IMPLICIT} definition
statement is parsed and stored in data structures
of the following types:
\begin{codeexample}
typedef struct t__identifier_info
  \{
     pointer_union_type ptr;    /* address in memory / function-parm-count */
     union
        \{
           struct qqbothidinfqq
              \{
                 dim_pair_type first;
                 dim_pair_type second;
              \} dims;
           dim_pair_type body;
        \} index;
     short scope_level;         /* scope level (negative iff. inactive) */
     char name[IDENT_MAXNCH_P]; /* identifier to search for */
     type_flagword_type flags;  /* type information */
  \} identifier_info_type;
\end{codeexample}
where the data structure for a dimension pair is of the following type:
\begin{codeexample}
typedef struct t__dim_pair
  \{
    Subscript lower;/* lower dimension (ARRAY) body-index (FUNCTION/IMPLICIT)*/
    Subscript upper;/* upper dimension (ARRAY) body-index (FUNCTION/IMPLICIT)*/
  \} dim_pair_type;
\end{codeexample}

and where the data structure for a pointer union is of the following type:
\begin{codeexample}
typedef union t__pointer_union
   \{
     relative_address_type relative_address;
     short parameter_count;
     void *vvv;               /* included for completeness */
     Boolean *bbb;            /* used when BOOL_TYPE */
     state_offset_type *sss;  /* used when SSVAR_TYPE */
     char *ccc;               /* used when CHAR_TYPE */
     int_type *iii;           /* used when INT_TYPE */
     real_type *rrr;          /* used when REAL_TYPE */
   \} pointer_union_type;
\end{codeexample}


\begin{codeexample}
typedef struct t__token_info
   {
      identifier_info_type *id_info_ptr;
      short linnum;
      short pos;
      token tok;
      rwtype rw;
      char id[IDENT_MAXNCH_P];
   } token_info_type;
\end{codeexample}

and the body is stored in the array pointed to by:
\begin{codeexample}
extern token_info_type *function_body_storage;
\end{codeexample}

which is allocated dynamically before the parse phase begins and freed again
before the rule generation phase begins.


As an example, consider the following \rw{IMPLICIT} definition statement:

\subsubsection{The FUNCTION statement data structure}
\label{sec:func}

The \rw{FUNCTION} definition
statement is parsed and stored in data structures
of the following types:
\begin{codeexample}
typedef struct t__identifier_info
  \{
     pointer_union_type ptr;    /* address in memory / function-parm-count */
     union
        \{
           struct qqbothidinfqq
              \{
                 dim_pair_type first;
                 dim_pair_type second;
              \} dims;
           dim_pair_type body;
        \} index;
     short scope_level;         /* scope level (negative iff. inactive) */
     char name[IDENT_MAXNCH_P]; /* identifier to search for */
     type_flagword_type flags;  /* type information */
  \} identifier_info_type;
\end{codeexample}
where the data structure for a dimension pair is of the following type:
\begin{codeexample}
typedef struct t__dim_pair
  \{
    Subscript lower;/* lower dimension (ARRAY) body-index (FUNCTION/IMPLICIT)*/
    Subscript upper;/* upper dimension (ARRAY) body-index (FUNCTION/IMPLICIT)*/
  \} dim_pair_type;
\end{codeexample}

and where the data structure for a pointer union is of the following type:
\begin{codeexample}
typedef union t__pointer_union
   \{
     relative_address_type relative_address;
     short parameter_count;
     void *vvv;               /* included for completeness */
     Boolean *bbb;            /* used when BOOL_TYPE */
     state_offset_type *sss;  /* used when SSVAR_TYPE */
     char *ccc;               /* used when CHAR_TYPE */
     int_type *iii;           /* used when INT_TYPE */
     real_type *rrr;          /* used when REAL_TYPE */
   \} pointer_union_type;
\end{codeexample}


\begin{codeexample}
typedef struct t__token_info
   {
      identifier_info_type *id_info_ptr;
      short linnum;
      short pos;
      token tok;
      rwtype rw;
      char id[IDENT_MAXNCH_P];
   } token_info_type;
\end{codeexample}

and the body is stored in the array pointed to by:
\begin{codeexample}
extern token_info_type *function_body_storage;
\end{codeexample}

which is allocated dynamically before the parse phase begins and freed again
before the rule generation phase begins.


As an example, consider the following \rw{FUNCTION} definition statement:
\begin{codeexample}
(00012): MAXVAL = 10;
(00013): FUNCTION FOO(I,X) = X**(MAXVAL-I);
\end{codeexample}

Figure \ref{fig:func-mem} illustrates how the above \rw{FUNCTION} definition
statement will be stored in memory.

\renewcommand{\qqon}{\begin{scriptsize}\begin{sf}}
\renewcommand{\qqoff}{\end{sf}\end{scriptsize}}

\startfig
\begin{fast_picture}{500}{600}
\setboxpos{60}{560}\savFboxpos
\leftwidetagbox{ptr.\#parms}{2}{72}{27}
\nextFbox\leftwidetagbox{lower}{$offset$}{72}{27}
         \dotoff\shiftboxcen{27}\outarrow{39}\doton
\nextFbox\leftwidetagbox{upper}{$offset+9$}{72}{27}
         \dotoff\shiftboxcen{27}\outarrowdownright{19}{520}{20}\doton
\nextFbox\leftwidetagbox{scope}{0}{72}{27}
\nextFbox\leftwidetagbox{name}{``FOO''}{72}{27}
\nextFbox\leftwidetagbox{flags}{function}{72}{27}
\setboxheight{5}\qqon
\setboxpos{153}{580}\savCboxpos\setboxpos{193}{580}\savDboxpos
\unsavCboxpos\putuponebox\widenoboxtall{$\vdots$}{48}{20}
%
   \nextCbox\nextDbox\leftwidetagbox{id~ptr}{}{72}{28}\outarrowdown{70}{20}
            \groundunderarrow
   \nextCbox\nextDbox\leftwidetagbox{line~\#}{13}{72}{28}
   \nextCbox\nextDbox\leftwidetagbox{pos}{20}{72}{28}
   \nextCbox\nextDbox\leftwidetagbox{tok}{TK\_LP}{72}{28}
   \nextCbox\nextDbox\leftwidetagbox{rw}{RW\_NULL}{72}{28}
   \nextDbox\leftwidetagbox{id}{``(''}{72}{28}
   \nextCbox\leftwidetalltagbox{}{}{72}{48}{60}
%
   \nextCbox\nextDbox\leftwidetagbox{id~ptr}{}{72}{28}\outarrowdown{180}{20}
        \setboxheight{10}\qqoff
        \stackwideEbox{31}
        \nextEbox\leftwidetagbox{ptr}{}{72}{31}\outarrowdown{60}{20}
                 \groundunderarrow
        \nextEbox\leftwidetagbox{lower}{simple}{72}{31}
        \nextEbox\leftwidetagbox{upper}{simple}{72}{31}
        \nextEbox\leftwidetagbox{scope}{dummy\#2}{72}{31}
        \nextEbox\leftwidetagbox{name}{``X''}{72}{31}
        \nextEbox\leftwidetagbox{flags}{unknown}{72}{31}
        \setboxheight{5}\qqon
   \nextCbox\nextDbox\leftwidetagbox{line~\#}{13}{72}{28}
   \nextCbox\nextDbox\leftwidetagbox{pos}{20}{72}{28}
   \nextCbox\nextDbox\leftwidetagbox{tok}{TK\_MACPARM}{72}{28}
   \nextCbox\nextDbox\leftwidetagbox{rw}{RW\_NULL}{72}{28}
   \nextDbox\leftwidetagbox{id}{``\$2''}{72}{28}
   \nextCbox\leftwidetalltagbox{}{}{72}{48}{60}
%
   \nextCbox\nextDbox\leftwidetagbox{id~ptr}{}{72}{28}\outarrowdown{70}{20}
            \groundunderarrow
   \nextCbox\nextDbox\leftwidetagbox{line~\#}{13}{72}{28}
   \nextCbox\nextDbox\leftwidetagbox{pos}{21}{72}{28}
   \nextCbox\nextDbox\leftwidetagbox{tok}{TK\_POW}{72}{28}
   \nextCbox\nextDbox\leftwidetagbox{rw}{RW\_NULL}{72}{28}
   \nextDbox\leftwidetagbox{id}{``**''}{72}{28}
   \nextCbox\leftwidetalltagbox{}{}{72}{48}{60}
%
   \nextCbox\nextDbox\leftwidetagbox{id~ptr}{}{72}{28}\outarrowdown{70}{20}
            \groundunderarrow
   \nextCbox\nextDbox\leftwidetagbox{line~\#}{13}{72}{28}
   \nextCbox\nextDbox\leftwidetagbox{pos}{23}{72}{28}
   \nextCbox\nextDbox\leftwidetagbox{tok}{TK\_LP}{72}{28}
   \nextCbox\nextDbox\leftwidetagbox{rw}{RW\_NULL}{72}{28}
   \nextDbox\leftwidetagbox{id}{``(''}{72}{28}
   \nextCbox\leftwidetalltagbox{}{}{72}{48}{60}
%
   \nextCbox\nextDbox\leftwidetagbox{id~ptr}{}{72}{28}\outarrowdown{180}{20}
        \setboxheight{10}\qqoff
        \stackwideEbox{31}
        \nextEbox\leftwidetagbox{ptr}{}{72}{31}\outarrow{60}
             \stackGboxtoright
             \nextGbox\valbox{10}
        \nextEbox\leftwidetagbox{lower}{simple}{72}{31}
        \nextEbox\leftwidetagbox{upper}{simple}{72}{31}
        \nextEbox\leftwidetagbox{scope}{0}{72}{31}
        \nextEbox\leftwidetagbox{name}{``MAXVAL''}{72}{31}
        \nextEbox\leftwidetagbox{flags}{unknown}{72}{31}
        \setboxheight{5}\qqon
   \nextCbox\nextDbox\leftwidetagbox{line~\#}{13}{72}{28}
   \nextCbox\nextDbox\leftwidetagbox{pos}{24}{72}{28}
   \nextCbox\nextDbox\leftwidetagbox{tok}{TK\_ID}{72}{28}
   \nextCbox\nextDbox\leftwidetagbox{rw}{RW\_NULL}{72}{28}
   \nextDbox\leftwidetagbox{id}{``MAXVAL''}{72}{28}
   \nextCbox\leftwidetalltagbox{}{}{72}{48}{60}
%
   \nextCbox\nextDbox\leftwidetagbox{id~ptr}{}{72}{28}\outarrowdown{70}{20}
            \groundunderarrow
   \nextCbox\nextDbox\leftwidetagbox{line~\#}{13}{72}{28}
   \nextCbox\nextDbox\leftwidetagbox{pos}{30}{72}{28}
   \nextCbox\nextDbox\leftwidetagbox{tok}{TK\_SUB}{72}{28}
   \nextCbox\nextDbox\leftwidetagbox{rw}{RW\_NULL}{72}{28}
   \nextDbox\leftwidetagbox{id}{``$-$''}{72}{28}
   \nextCbox\leftwidetalltagbox{}{}{72}{48}{60}
%
   \nextCbox\nextDbox\leftwidetagbox{id~ptr}{}{72}{28}
        \outarrowdownrightdown{100}{35}{80}{20}
        \setboxheight{10}\qqoff
        \stackwideEbox{31}
        \nextEbox\leftwidetagbox{ptr}{}{72}{31}\outarrowdown{60}{20}
                 \groundunderarrow
        \nextEbox\leftwidetagbox{lower}{simple}{72}{31}
        \nextEbox\leftwidetagbox{upper}{simple}{72}{31}
        \nextEbox\leftwidetagbox{scope}{dummy\#1}{72}{31}
        \nextEbox\leftwidetagbox{name}{``I''}{72}{31}
        \nextEbox\leftwidetagbox{flags}{unknown}{72}{31}
        \setboxheight{5}\qqon
   \nextCbox\nextDbox\leftwidetagbox{line~\#}{13}{72}{28}
   \nextCbox\nextDbox\leftwidetagbox{pos}{31}{72}{28}
   \nextCbox\nextDbox\leftwidetagbox{tok}{TK\_MACPARM}{72}{28}
   \nextCbox\nextDbox\leftwidetagbox{rw}{RW\_NULL}{72}{28}
   \nextDbox\leftwidetagbox{id}{``\$1''}{72}{28}
   \nextCbox\leftwidetalltagbox{}{}{72}{48}{60}
%
   \nextCbox\nextDbox\leftwidetagbox{id~ptr}{}{72}{28}\outarrowdown{70}{20}
            \groundunderarrow
   \nextCbox\nextDbox\leftwidetagbox{line~\#}{13}{72}{28}
   \nextCbox\nextDbox\leftwidetagbox{pos}{32}{72}{28}
   \nextCbox\nextDbox\leftwidetagbox{tok}{TK\_RP}{72}{28}
   \nextCbox\nextDbox\leftwidetagbox{rw}{RW\_NULL}{72}{28}
   \nextDbox\leftwidetagbox{id}{``)''}{72}{28}
   \nextCbox\leftwidetalltagbox{}{}{72}{48}{60}
%
   \nextCbox\nextDbox\leftwidetagbox{id~ptr}{}{72}{28}\outarrowdown{70}{20}
            \groundunderarrow
   \nextCbox\nextDbox\leftwidetagbox{line~\#}{13}{72}{28}
   \nextCbox\nextDbox\leftwidetagbox{pos}{33}{72}{28}
   \nextCbox\nextDbox\leftwidetagbox{tok}{TK\_RP}{72}{28}
   \nextCbox\nextDbox\leftwidetagbox{rw}{RW\_NULL}{72}{28}
   \nextDbox\leftwidetagbox{id}{``)''}{72}{28}
   \nextCbox\leftwidetalltagbox{}{}{72}{48}{60}
%
\nextCbox\nextCbox
\unsavCboxpos\widenoboxtall{$\vdots$}{48}{20}
\setboxheight{10}\qqoff
%
\end{fast_picture}
\finishfig{Layout of FUNCTION definition in memory}{fig:func-mem}

Note that the upper index is one greater than the last index for the parameter
list so that a loop on the parameters can continue while the index is strictly
less than this value.

\subsubsection{The macro expansion stack data structure}
\label{sec:expand}

\renewcommand{\qqon}{\begin{scriptsize}\begin{sf}}
\renewcommand{\qqoff}{\end{sf}\end{scriptsize}}

When macros (\rw{IMPLICIT} variables and \rw{FUNCTION}s) are expanded, the
body of the expansion is pushed onto the ``macro expansion stack''.    When
parsing advances to the next token, the macro expansion stack is first checked
before reading from the input stream.   If the stack is non-empty, then the
next token in the body on the top of the stack is taken.  If the body list
on top of the stack has already been exhausted, then the stack is popped.
When the stack is empty, the next token is read from the input stream.

The macro expansion stack consists of a stack of elements of the
following data type:
\begin{codeexample}
typedef struct t__macro_expansion_info
   \{
       token_info_type *passed_token_list;
       unsigned short *passed_token_offset;
       unsigned short *passed_token_counts;
       short now_passed_ix;
       short now_passed_count;
       short passed_parameter_count;
       short now_body_ix;
       short ovf_body_ix;
       short pos;
       short linnum;
   \} macro_expansion_info_type;
\end{codeexample}
where the data structure for a dimension pair is of the following type:
\begin{codeexample}
typedef struct t__dim_pair
  \{
    Subscript lower;/* lower dimension (ARRAY) body-index (FUNCTION/IMPLICIT)*/
    Subscript upper;/* upper dimension (ARRAY) body-index (FUNCTION/IMPLICIT)*/
  \} dim_pair_type;
\end{codeexample}

and where the data structure for a pointer union is of the following type:
\begin{codeexample}
typedef union t__pointer_union
   \{
     relative_address_type relative_address;
     short parameter_count;
     void *vvv;               /* included for completeness */
     Boolean *bbb;            /* used when BOOL_TYPE */
     state_offset_type *sss;  /* used when SSVAR_TYPE */
     char *ccc;               /* used when CHAR_TYPE */
     int_type *iii;           /* used when INT_TYPE */
     real_type *rrr;          /* used when REAL_TYPE */
   \} pointer_union_type;
\end{codeexample}



Each macro expansion information record contains an
array of lists of calling expression tokens as well as a pointer
to the list of tokens making up the body of the macro.  For example,
consider:
\begin{codeexample}
(00070): FUNCTION F(X) = X * (X-2.0*X);\label{xmp:expansion-stack}
(00071): FUNCTION G(A,B,C) = A * (B - C);
         ...
(00099): ...  + G(F(2.0+QQQ),F(2.0-QQQ),F((Q1+Q2)/2.0)) ...
\end{codeexample}

In the previous invocation of function G() there are nested invocations of
function F().   During the parsing of $F(2.0+QQQ)$, a brief description
(a more detailed description using a simpler example will appear later)
of the contents of the stack is diagrammed in Figure \ref{fig:overview-stk}.

\startfig
\begin{fast_picture}{500}{150}
\setboxpos{60}{120}\savFboxpos
\nextFbox\leftwidetagbox{top}{body:~~$(\$1*(\$1-2.0*\$1))$}{72}{85}
         \wideboxtoright{85}\valbox{}\outarrow{55}
                 \stackBboxtoright
                 \nextBbox\rightwidetagbox{$2.0+QQQ$}{\$1}{55}{72}
\nextFbox\leftwidetagbox{bottom}{body:~~$(\$1*(\$2-\$3))$}{72}{85}
         \wideboxtoright{85}\valbox{}\outarrowdownright{30}{20}{25}
                 \stackBboxtoright
                 \nextBbox\rightwidetagbox{$F(2.0+QQQ)$}{\$1}{55}{72}
                 \nextBbox\rightwidetagbox{$F(2.0-QQQ)$}{\$2}{55}{72}
                 \nextBbox\rightwidetagbox{$F((Q1+Q2)/2.0)$}{\$3}{55}{72}
\end{fast_picture}
\finishfig{Overview of expansion stack during nested function invocation}{fig:overview-stk}

When a parameter reference, such as \$1, is
encountered, the ``now\_passed\_ix''
is changed from negative one less the offset to the index
of the parameter (0 for \$1, 1 for
\$2, 2 for \$3, etc.).   When all tokens for the parameter have been
exhausted, then the index is changed back to negative one less the offset
and the next token
is taken from the body once again.   When all tokens for the body have
been exhausted, then the stack is popped.

The arithmetic ``negative one less the offset'' is used because the number
zero is a valid offset and negative zero is the same as positive zero.   The
extra offset of negative one is therefore necessary so that the set of
negatives is disjoint from the set of positives.

Parentheses are inserted when the count of the number of tokens making up
a calling parameter is greater than one.   For example, the following
translations will be performed:
\begin{typedef_tabbing}
wwwww \= $F(A+B)$ \= $\longrightarrow$ \= wwwwwwwwwwww \kill
\> $F(MU)$ \> $\longrightarrow$ \> $MU*(MU-2.0)$ \\
\> $F(A+B)$ \> $\longrightarrow$ \> $(A+B)*((A+B)-2.0)$  \\
\end{typedef_tabbing}

In the above examples, no parentheses were added in the first example because
the calling parameter ``$MU$'' is only one token long.  Parentheses were added
in the second example since ``$A+B$'' is three tokens long (more than one).

The following data types are referenced by the elements on the stack:
\begin{codeexample}
typedef struct t__token_info
   {
      identifier_info_type *id_info_ptr;
      short linnum;
      short pos;
      token tok;
      rwtype rw;
      char id[IDENT_MAXNCH_P];
   } token_info_type;
\end{codeexample}

\begin{codeexample}
typedef struct t__identifier_info
  \{
     pointer_union_type ptr;    /* address in memory / function-parm-count */
     union
        \{
           struct qqbothidinfqq
              \{
                 dim_pair_type first;
                 dim_pair_type second;
              \} dims;
           dim_pair_type body;
        \} index;
     short scope_level;         /* scope level (negative iff. inactive) */
     char name[IDENT_MAXNCH_P]; /* identifier to search for */
     type_flagword_type flags;  /* type information */
  \} identifier_info_type;
\end{codeexample}
where the data structure for a dimension pair is of the following type:
\begin{codeexample}
typedef struct t__dim_pair
  \{
    Subscript lower;/* lower dimension (ARRAY) body-index (FUNCTION/IMPLICIT)*/
    Subscript upper;/* upper dimension (ARRAY) body-index (FUNCTION/IMPLICIT)*/
  \} dim_pair_type;
\end{codeexample}

and where the data structure for a pointer union is of the following type:
\begin{codeexample}
typedef union t__pointer_union
   \{
     relative_address_type relative_address;
     short parameter_count;
     void *vvv;               /* included for completeness */
     Boolean *bbb;            /* used when BOOL_TYPE */
     state_offset_type *sss;  /* used when SSVAR_TYPE */
     char *ccc;               /* used when CHAR_TYPE */
     int_type *iii;           /* used when INT_TYPE */
     real_type *rrr;          /* used when REAL_TYPE */
   \} pointer_union_type;
\end{codeexample}



As an example, consider the following FUNCTION definition statement:
\begin{logfileexample}
(00012): MAXVAL = 10;
(00013): FUNCTION FOO(I,X) = X**(MAXVAL-I);
\end{logfileexample}

and the following reference to this function:
\begin{logfileexample}
(00016): IF (...) TRANTO ...
(00017):    BY FOO(12-2*IX,OMEGA/LAMBDA) ...
\end{logfileexample}

The illustration in Figure \ref{fig:expand-stack}
details how the above FUNCTION definition
statement will be pushed onto the macro expansion stack.

\startfig
\begin{fast_picture}{500}{570}
\setboxpos{60}{530}\savFboxpos
\nextFbox\leftwidetagbox{token~list}{}{108}{33}\outarrowdownright{200}{20}{40}
         \setboxheight{5}\qqon
         \setboxpos{333}{530}\savCboxpos\setboxpos{373}{530}\savDboxpos
         \unsavCboxpos\putuponebox\widenoboxtall{$\vdots$}{48}{20}
         %
         \nextCbox\nextDbox\leftwidetagbox{id~ptr}{}{72}{28}
                  \outarrowdown{70}{20}\groundunderarrow
         \nextCbox\nextDbox\leftwidetagbox{line~\#}{17}{72}{28}
         \nextCbox\nextDbox\leftwidetagbox{pos}{10}{72}{28}
         \nextCbox\nextDbox\leftwidetagbox{tok}{TK\_INT}{72}{28}
         \nextCbox\nextDbox\leftwidetagbox{rw}{RW\_NULL}{72}{28}
         \nextDbox\leftwidetagbox{id}{``\#12''}{72}{28}
         \nextCbox\leftwidetalltagbox{}{}{72}{48}{60}
         %
         \nextCbox\nextDbox\leftwidetagbox{id~ptr}{}{72}{28}
                  \outarrowdown{70}{20}\groundunderarrow
         \nextCbox\nextDbox\leftwidetagbox{line~\#}{17}{72}{28}
         \nextCbox\nextDbox\leftwidetagbox{pos}{12}{72}{28}
         \nextCbox\nextDbox\leftwidetagbox{tok}{TK\_SUB}{72}{28}
         \nextCbox\nextDbox\leftwidetagbox{rw}{RW\_NULL}{72}{28}
         \nextDbox\leftwidetagbox{id}{``$-$''}{72}{28}
         \nextCbox\leftwidetalltagbox{}{}{72}{48}{60}
         %
         \nextCbox\nextDbox\leftwidetagbox{id~ptr}{}{72}{28}
                  \outarrowdown{70}{20}\groundunderarrow
         \nextCbox\nextDbox\leftwidetagbox{line~\#}{17}{72}{28}
         \nextCbox\nextDbox\leftwidetagbox{pos}{13}{72}{28}
         \nextCbox\nextDbox\leftwidetagbox{tok}{TK\_INT}{72}{28}
         \nextCbox\nextDbox\leftwidetagbox{rw}{RW\_NULL}{72}{28}
         \nextDbox\leftwidetagbox{id}{``\#2''}{72}{28}
         \nextCbox\leftwidetalltagbox{}{}{72}{48}{60}
         %
         \nextCbox\nextDbox\leftwidetagbox{id~ptr}{}{72}{28}
                  \outarrowdown{70}{20}\groundunderarrow
         \nextCbox\nextDbox\leftwidetagbox{line~\#}{17}{72}{28}
         \nextCbox\nextDbox\leftwidetagbox{pos}{14}{72}{28}
         \nextCbox\nextDbox\leftwidetagbox{tok}{TK\_MUL}{72}{28}
         \nextCbox\nextDbox\leftwidetagbox{rw}{RW\_NULL}{72}{28}
         \nextDbox\leftwidetagbox{id}{``*''}{72}{28}
         \nextCbox\leftwidetalltagbox{}{}{72}{48}{60}
         %
         \nextCbox\nextDbox\leftwidetagbox{id~ptr}{}{72}{28}
                  \outarrowdown{70}{20}\groundunderarrow
         \nextCbox\nextDbox\leftwidetagbox{line~\#}{17}{72}{28}
         \nextCbox\nextDbox\leftwidetagbox{pos}{15}{72}{28}
         \nextCbox\nextDbox\leftwidetagbox{tok}{TK\_ID}{72}{28}
         \nextCbox\nextDbox\leftwidetagbox{rw}{RW\_NULL}{72}{28}
         \nextDbox\leftwidetagbox{id}{``IX''}{72}{28}
         \nextCbox\leftwidetalltagbox{}{}{72}{48}{60}
         %
         \nextCbox\nextDbox\leftwidetagbox{id~ptr}{}{72}{28}
                  \outarrowdown{70}{20}\groundunderarrow
         \nextCbox\nextDbox\leftwidetagbox{line~\#}{17}{72}{28}
         \nextCbox\nextDbox\leftwidetagbox{pos}{18}{72}{28}
         \nextCbox\nextDbox\leftwidetagbox{tok}{TK\_ID}{72}{28}
         \nextCbox\nextDbox\leftwidetagbox{rw}{RW\_NULL}{72}{28}
         \nextDbox\leftwidetagbox{id}{``OMEGA''}{72}{28}
         \nextCbox\leftwidetalltagbox{}{}{72}{48}{60}
         %
         \nextCbox\nextDbox\leftwidetagbox{id~ptr}{}{72}{28}
                  \outarrowdown{70}{20}\groundunderarrow
         \nextCbox\nextDbox\leftwidetagbox{line~\#}{17}{72}{28}
         \nextCbox\nextDbox\leftwidetagbox{pos}{23}{72}{28}
         \nextCbox\nextDbox\leftwidetagbox{tok}{TK\_DVD}{72}{28}
         \nextCbox\nextDbox\leftwidetagbox{rw}{RW\_NULL}{72}{28}
         \nextDbox\leftwidetagbox{id}{``/''}{72}{28}
         \nextCbox\leftwidetalltagbox{}{}{72}{48}{60}
         %
         \nextCbox\nextDbox\leftwidetagbox{id~ptr}{}{72}{28}
                  \outarrowdown{70}{20}\groundunderarrow
         \nextCbox\nextDbox\leftwidetagbox{line~\#}{17}{72}{28}
         \nextCbox\nextDbox\leftwidetagbox{pos}{24}{72}{28}
         \nextCbox\nextDbox\leftwidetagbox{tok}{TK\_ID}{72}{28}
         \nextCbox\nextDbox\leftwidetagbox{rw}{RW\_NULL}{72}{28}
         \nextDbox\leftwidetagbox{id}{``LAMBDA''}{72}{28}
         \nextCbox\leftwidetalltagbox{}{}{72}{48}{60}
         %
         \nextCbox\nextCbox
         \unsavCboxpos\widenoboxtall{$\vdots$}{48}{20}
         \setboxheight{10}\qqoff
\nextFbox\leftwidetagbox{token~offset}{}{108}{33}\outarrowdown{130}{40}
         \stackGbox
         \nextGbox\valbox{$0$}
                  \dotoff\shiftboxcen{9}\outarrowupright{61}{50}{40}\doton
         \nextGbox\valbox{$5$}
                  \dotoff\shiftboxcen{9}\outarrowdownright{61}{230}{40}\doton
\nextFbox\leftwidetagbox{token~counts}{}{108}{33}\outarrowdown{80}{20}
         \stackGbox
         \nextGbox\valbox{$5$}
         \nextGbox\valbox{$3$}
\nextFbox\leftwidetagbox{now~passed~index}{$-1-offset$}{108}{33}
\nextFbox\leftwidetagbox{now~passed~count}{$-1$}{108}{33}
\nextFbox\leftwidetagbox{passed~count}{$2$}{108}{33}
\nextFbox\leftwidetagbox{now~body~ix}{$-1$}{108}{33}
\nextFbox\leftwidetagbox{ovf~body~ix}{$offset+9$}{108}{33}
\nextFbox\leftwidetagbox{pos}{6}{108}{33}
\nextFbox\leftwidetagbox{line~\#}{17}{108}{33}
\end{fast_picture}
\finishfig{Detail of expansion stack during function invocation}{fig:expand-stack}

Note that all of the identifier table pointers are null.  This is because
passed parameters are just sequences of tokens.   No identifier information
is needed when a token is pulled off of the macro expansion stack.
Identifier information is always looked up after a token is retrieved
regardless of whether it is retrieved from the macro expansion stack or from
the input file.


\subsection{The VARIABLE statement data structure}
\label{sec:calc}

The \rw{VARIABLE} statements are parsed and stored in data stuctures
of the following types:
\begin{codeexample}
typedef struct t__calc_assign
   \{
       identifier_info_type *idinfo;  /* <ident> := */
       expression_type *expr;         /*            <expr> */
   \} calc_assign_type;
\end{codeexample}

\begin{codeexample}
typedef struct t__expression
   \{
       operation_type *postfix_ops;
       operation_type *infix_ops;
       operand_type *operands;
       short n_postfix_ops;
       short n_infix_ops;
       short n_operands;
       short source_code_line_number;
       Boolean in_error;
       type_flagword_type rtntype;
   \} expression_type;
\end{codeexample}

\begin{codeexample}
typedef struct t__identifier_info
  \{
     pointer_union_type ptr;    /* address in memory / function-parm-count */
     union
        \{
           struct qqbothidinfqq
              \{
                 dim_pair_type first;
                 dim_pair_type second;
              \} dims;
           dim_pair_type body;
        \} index;
     short scope_level;         /* scope level (negative iff. inactive) */
     char name[IDENT_MAXNCH_P]; /* identifier to search for */
     type_flagword_type flags;  /* type information */
  \} identifier_info_type;
\end{codeexample}
where the data structure for a dimension pair is of the following type:
\begin{codeexample}
typedef struct t__dim_pair
  \{
    Subscript lower;/* lower dimension (ARRAY) body-index (FUNCTION/IMPLICIT)*/
    Subscript upper;/* upper dimension (ARRAY) body-index (FUNCTION/IMPLICIT)*/
  \} dim_pair_type;
\end{codeexample}

and where the data structure for a pointer union is of the following type:
\begin{codeexample}
typedef union t__pointer_union
   \{
     relative_address_type relative_address;
     short parameter_count;
     void *vvv;               /* included for completeness */
     Boolean *bbb;            /* used when BOOL_TYPE */
     state_offset_type *sss;  /* used when SSVAR_TYPE */
     char *ccc;               /* used when CHAR_TYPE */
     int_type *iii;           /* used when INT_TYPE */
     real_type *rrr;          /* used when REAL_TYPE */
   \} pointer_union_type;
\end{codeexample}



As an example, consider:
\begin{logfileexample}
(00010): VARIABLE NWP[NFP] = NP-NFP;
\end{logfileexample}

The layout of the above \rw{VARIABLE} is illustrated
in Figure \ref{fig:variable}.

\startfig
\begin{fast_picture}{200}{50}
\setboxpos{40}{30}\putuponebox\savFboxpos
\nextFbox\lefttagbox{idinfo}{}\outarrowmsg{100}{NWP}
\nextFbox\lefttagbox{expr}{}\outarrowmsg{40}{NP-NFP}
\end{fast_picture}
\finishfig{VARIABLE statement laid out in memory}{fig:variable}

\subsection{The cross-reference-map entry data structure}
\label{sec:xref}

When a cross-reference map is requested with the \option{xref} 
command line option, a file is created with entries of the following
type:
\begin{codeexample}
typedef struct t__cross_reference_entry
   \{
      unsigned short linnum;
      unsigned short pos;
      char name[XREF_IDENT_MAXNCH_P];
      char refcode;
   \} cross_reference_entry_type;
\end{codeexample}


The ``linnum'' field holds the \acronym{ASSIST} input file line number
as listed in the log file.

The ``pos'' field holds the character column position on the line where
the name begins.

The ``name'' field holds the name of the item being cross referenced.
Usually the name is just the identifier name.   Cross reference listings
also detail \rw{ELSE's} and \rw{ENDIF's} with their corresponding \rw{IF's}
as well as \rw{ENDFOR's} with their corresponding \rw{FOR's}.

The ``refcode'' is a character which indicates the type of reference:
\begin{itemize}
\item 'D' stands for a declaration.
\item 'S' stands for ``set'' and indicates where the ``name'' takes on a value.
\item 'U' stands for ``use'' and indicates where the value of ``name'' is
      used to compute something else.
\end{itemize}

Since there are no pointers in this data structure, a diagram was deemed
unnecessary.


\section{Data Structures used to generate a model file}
\label{chap:ruledata}

\subsection{Introduction to model generation data structures}
\label{sec:codeintro}

Four copies of the rule section code are stored in memory for fast and 
efficient generation of a model.   The four copies are for:
\begin{itemize}
\item \rw{ASSERT} assertions
\item \rw{DEATHIF} failures
\item \rw{PRUNEIF} checks
\item \rw{TRANTO} transitions
\end{itemize}
These four copies of the rule section are preceded by a preamble section of
code which contains the \rw{START} state transition and pointers to the
four copies of the rule code listed above.

Each of the four copies has code for its own type in addition to code for
other types not in a copy of their own.   For example, the assertion
copy of the code contains ASSERT, IF, and FOR statements as well as some
internal statements (which are not part of the ASSIST language per se.)
such as GOTO, GOSUB and RETURN.   Operation codes are detailed in Chapter
\ref{chap:pseudo} on page \pageref{chap:pseudo}.

The body of a block \rw{IF} and the body of a \rw{FOR} are implemented
in the pseudo-code language as a subroutine.  The \rw{THEN}, \rw{ELSE}, and
\rw{FOR} keywords imply a GOSUB instruction.   The \rw{ELSE}, \rw{ENDIF}, and
\rw{ENDFOR} imply the RETURN instruction corresponding to the respective
keywords implying the GOSUB instruction.

Code is stored in data structures of the following types:
\begin{codeexample}
typedef struct t__instruction_pointer_union
   \{
       void *vvv;                      /* to cast to block_if_type, etc. */
       relative_address_type reladdr;  /* relative address of code */
   \} instruction_pointer_union_type;
\end{codeexample}

\begin{codeexample}
typedef struct t__instruction
   \{
       instruction_pointer_union_type ptr;
       opcode_type opcode;      /* instruction operation code */
   \} instruction_type;          /*    for_loop_type, assert_type, ... */
\end{codeexample}

\begin{codeexample}
typedef struct t__block_if
   \{
       expression_type *then_test;    /* boolean expression for THEN */
       instruction_type *then_clause; /* code for THEN clause */
       instruction_type *else_clause; /* code for ELSE clause */
   \} block_if_type;
\end{codeexample}

\begin{codeexample}
typedef struct t__set_range
   \{
       expression_type *lower_bound;
       expression_type *upper_bound;
   \} set_range_type;
\end{codeexample}

\begin{codeexample}
typedef struct t__for_loop
   \{
       identifier_info_type *ident;  /* index variable */
       set_range_type *set_ranges;   /* pointer to array of IN ranges */
       short set_range_count;        /* count of number of IN ranges */
       instruction_type *body;       /* pointer to BODY of loop */
   \} for_loop_type;
\end{codeexample}

\begin{codeexample}
typedef struct t__state_space_picture
   \{
       vars_union_type *varu;
       Boolean *is_nested;
       short nvaru;
   \} state_space_picture_type;
\end{codeexample}

\begin{codeexample}
typedef union t__vars_union
   \{
       identifier_info_type *id_info;
       state_space_picture_type *nested_space_picture;
       relative_address_type relative_address;
   \} vars_union_type;
\end{codeexample}


\begin{codeexample}
typedef union t__node_union
   \{
       state_space_picture_type pix;      /* 10 bytes  (max(10,10)=10 bytes) */
   \} node_union_type;
\end{codeexample}

\begin{codeexample}
typedef struct t__space_expression
   \{
       expression_type *exprs;
       operand_type *vars;
       short n_vars;
   \} space_expression_type;
\end{codeexample}

\begin{codeexample}
typedef struct t__tranto_clause
   \{
       space_expression_type sex;   /* list of space transition expressions */
       expression_type *rate_exprs; /* ptr to array of rate expressions */
       short n_rate_exprs;          /* count of rate expressions */
       short source_code_line_number;
   \} tranto_clause_type;
\end{codeexample}

\begin{codeexample}
typedef struct t__booltest
   \{
       expression_type *expr;         /* boolean expr to ASSERT,DEATHIF,etc. */
       short source_code_line_number; /* line number in listing file */
       short lumping_sequence;        /* sequence index (0..n-1) in source */
   \} booltest_type;                   /* e.g., first DEATHIF, second DEATHIF */
\end{codeexample}

\begin{codeexample}
typedef booltest_type assert_type;
\end{codeexample}

\begin{codeexample}
typedef booltest_type deathif_type;
\end{codeexample}

\begin{codeexample}
typedef booltest_type pruneif_type;
\end{codeexample}


Note that the
pointer ``void *vvv'' in the ``instruction\_pointer\_union\_type'' data type
is cast to a pointer of the appropriate type as
illustrated for ``instruction\_type inst;'' in the following examples:
\begin{codeexample}
(block_if_type *) inst.vvv
(for_loop_type *) inst.vvv
(tranto_clause_type *) inst.vvv
(assert_type *) inst.vvv
(deathif_type *) inst.vvv
(pruneif_type *) inst.vvv
\end{codeexample}

Examples of instructions are given in the sections which follow.

\subsection{The ASSERT statement data structure}
\label{sec:assert}

The ASSERT statement is parsed
and stored in data structures of the following types:
\begin{codeexample}
typedef struct t__instruction_pointer_union
   \{
       void *vvv;                      /* to cast to block_if_type, etc. */
       relative_address_type reladdr;  /* relative address of code */
   \} instruction_pointer_union_type;
\end{codeexample}

\begin{codeexample}
typedef struct t__instruction
   \{
       instruction_pointer_union_type ptr;
       opcode_type opcode;      /* instruction operation code */
   \} instruction_type;          /*    for_loop_type, assert_type, ... */
\end{codeexample}

\begin{codeexample}
typedef struct t__booltest
   \{
       expression_type *expr;         /* boolean expr to ASSERT,DEATHIF,etc. */
       short source_code_line_number; /* line number in listing file */
       short lumping_sequence;        /* sequence index (0..n-1) in source */
   \} booltest_type;                   /* e.g., first DEATHIF, second DEATHIF */
\end{codeexample}

\begin{codeexample}
typedef booltest_type assert_type;
\end{codeexample}


The ``expr'' field is a pointer to the expression to test for conformance.

The ``source\_code\_line\_number'' indicates the line number in the source
code (\extent{.ast}) file as listed in the log (\extent{.alog}) file
where the statement began.   It is used to print intelligent
error messages during both parsing and model generation phases.

The ``sequence\_index'' gives the \rw{ASSERT} sequence number.   This number
is zero for the first \rw{ASSERT} which was parsed, one for the second
\rw{ASSERT} which was parsed, etc.   The sequence number is used in order
to lump death and prune states.   The sequence number is defined
for \rw{ASSERT} statements but is currently not referenced since \rw{ASSERT}
statements are not lumped.  It is independent of the number of \rw{DEATHIF}
and \rw{PRUNEIF} statements encountered since they have their own sequence
index counters.

As an example, consider the following log file excerpt showing only
the \rw{ASSERT} statements:
\begin{logfileexample}
(00099): ASSERT ...
         \vdots
(00112): ASSERT ...
(00113): ASSERT ...
         \vdots
(00125): ASSERT ...
(00126): ASSERT ...
         \vdots
(00139): ASSERT ...
         \vdots
(00147): ASSERT NP>NFP;
\end{logfileexample}

Figure \ref{fig:assert-mem} pictures the memory layout of the \rw{ASSERT} from
line 147.  Note that the sequence index is six because sequences begin with
the number zero.

For a more complete illustration of how an expression is laid out in
memory, see Figure \ref{fig:expr-mem} on page \pageref{fig:expr-mem}.

\startfig
\begin{fast_picture}{500}{120}
\setboxpos{135}{100}\savFboxpos
\leftwidetagbox{ptr}{}{72}{29}\outarrowdown{100}{20}
      \stackBbox
      \nextBbox\lefttagbox{expr}{}\outarrowmsg{50}{$NP>NFP$}
      \nextBbox\lefttagbox{line~\#}{147}
      \nextBbox\lefttagbox{seq~\#}{6}
\nextFbox\leftwidetagbox{opcode}{ASSERT}{72}{29}
\end{fast_picture}
\finishfig{Sample ASSERT laid out in memory}{fig:assert-mem}

\subsection{The DEATHIF statement data structure}
\label{sec:deathif}

The DEATHIF statement is parsed
and stored in data structures of the following types:
\begin{codeexample}
typedef struct t__instruction_pointer_union
   \{
       void *vvv;                      /* to cast to block_if_type, etc. */
       relative_address_type reladdr;  /* relative address of code */
   \} instruction_pointer_union_type;
\end{codeexample}

\begin{codeexample}
typedef struct t__instruction
   \{
       instruction_pointer_union_type ptr;
       opcode_type opcode;      /* instruction operation code */
   \} instruction_type;          /*    for_loop_type, assert_type, ... */
\end{codeexample}

\begin{codeexample}
typedef struct t__booltest
   \{
       expression_type *expr;         /* boolean expr to ASSERT,DEATHIF,etc. */
       short source_code_line_number; /* line number in listing file */
       short lumping_sequence;        /* sequence index (0..n-1) in source */
   \} booltest_type;                   /* e.g., first DEATHIF, second DEATHIF */
\end{codeexample}

\begin{codeexample}
typedef booltest_type deathif_type;
\end{codeexample}


The ``expr'' field is a pointer to the expression to test for conformance.

The ``source\_code\_line\_number'' indicates the line number in the source
code (\extent{.ast}) file as listed in the log (\extent{.alog}) file
where the statement began.   It is used to print intelligent
error messages during both parsing and model generation phases.

The ``sequence\_index'' gives the \rw{DEATHIF} sequence number.   This number
is zero for the first \rw{DEATHIF} which was parsed, one for the second
\rw{DEATHIF} which was parsed, etc.   The sequence number is used in order
to lump death and prune states.  It is independent of the number of \rw{ASSERT}
and \rw{PRUNEIF} statements encountered since they have their own sequence
index counters.

As an example, consider the following \rw{DEATHIF} statement preceded by an
\rw{IMPLICIT} definition which is referenced in the \rw{DEATHIF}:
\begin{logfileexample}
(00146): IMPLICIT NWP(NP,NFP) = NP-NFP;
(00147): DEATHIF NFP>NWP;
\end{logfileexample}

Figure \ref{fig:deathif-mem} pictures the memory layout of the \rw{DEATHIF}
from line 147.
Note that the sequence index is zero because sequences begin with
the number zero and there are no \rw{DEATHIF} statements preceding it.
Notice also that the \rw{IMPLICIT} variable macro expansion was made
before the expression was parsed and stored.

For a more complete illustration of how an expression is laid out in
memory, see Figure \ref{fig:expr-mem} on page \pageref{fig:expr-mem}.

\startfig
\begin{fast_picture}{500}{120}
\setboxpos{135}{100}\savFboxpos
\leftwidetagbox{ptr}{}{72}{29}\outarrowdown{100}{20}
      \stackBbox
      \nextBbox\lefttagbox{expr}{}\outarrowmsg{50}{$NFP>(NP-NFP)$}
      \nextBbox\lefttagbox{line~\#}{147}
      \nextBbox\lefttagbox{seq~\#}{0}
\nextFbox\leftwidetagbox{opcode}{DEATHIF}{72}{29}
\end{fast_picture}
\finishfig{Sample DEATHIF laid out in memory}{fig:deathif-mem}

\subsection{The PRUNEIF statement data structure}
\label{sec:pruneif}

The PRUNEIF statement is parsed
and stored in data structures of the following types:
\begin{codeexample}
typedef struct t__instruction_pointer_union
   \{
       void *vvv;                      /* to cast to block_if_type, etc. */
       relative_address_type reladdr;  /* relative address of code */
   \} instruction_pointer_union_type;
\end{codeexample}

\begin{codeexample}
typedef struct t__instruction
   \{
       instruction_pointer_union_type ptr;
       opcode_type opcode;      /* instruction operation code */
   \} instruction_type;          /*    for_loop_type, assert_type, ... */
\end{codeexample}

\begin{codeexample}
typedef struct t__booltest
   \{
       expression_type *expr;         /* boolean expr to ASSERT,DEATHIF,etc. */
       short source_code_line_number; /* line number in listing file */
       short lumping_sequence;        /* sequence index (0..n-1) in source */
   \} booltest_type;                   /* e.g., first DEATHIF, second DEATHIF */
\end{codeexample}

\begin{codeexample}
typedef booltest_type pruneif_type;
\end{codeexample}


The ``expr'' field is a pointer to the expression to test for conformance.

The ``source\_code\_line\_number'' indicates the line number in the source
code (\extent{.ast}) file as listed in the log (\extent{.alog}) file
where the statement began.   It is used to print intelligent
error messages during both parsing and model generation phases.

The ``sequence\_index'' gives the \rw{PRUNEIF} sequence number.   This number
is zero for the first \rw{PRUNEIF} which was parsed, one for the second
\rw{PRUNEIF} which was parsed, etc.   The sequence number is used in order
to lump death and prune states.  It is independent of the number of \rw{ASSERT}
and \rw{DEATHIF} statements encountered since they have their own sequence
index counters.

As an example, consider the following PRUNEIF statements:
\begin{logfileexample}
(00101): PRUNEIF  ...
         \vdots
(00122): PRUNEIF  ...
         \vdots
(00177): PRUNEIF NFP>3;
\end{logfileexample}

Figure \ref{fig:pruneif-mem} illustrates the memory layout
of the \rw{PRUNEIF} on line 177.
Note that the sequence index is two because sequences begin with
the number zero.

For a more complete illustration of how an expression is laid out in
memory, see Figure \ref{fig:expr-mem} on page \pageref{fig:expr-mem}.

\startfig
\begin{fast_picture}{500}{120}
\setboxpos{135}{100}\savFboxpos
\leftwidetagbox{ptr}{}{72}{29}\outarrowdown{100}{20}
      \stackBbox
      \nextBbox\lefttagbox{expr}{}\outarrowmsg{50}{$NFP>3$}
      \nextBbox\lefttagbox{line~\#}{177}
      \nextBbox\lefttagbox{seq~\#}{2}
\nextFbox\leftwidetagbox{opcode}{PRUNEIF}{72}{29}
\end{fast_picture}
\finishfig{Sample PRUNEIF laid out in memory}{fig:pruneif-mem}

\subsection{The TRANTO statement data structure}
\label{sec:tranto}

The \rw{TRANTO} statement clause is parsed
and stored in a data structures of the following types:
\begin{codeexample}
typedef struct t__tranto_clause
   \{
       space_expression_type sex;   /* list of space transition expressions */
       expression_type *rate_exprs; /* ptr to array of rate expressions */
       short n_rate_exprs;          /* count of rate expressions */
       short source_code_line_number;
   \} tranto_clause_type;
\end{codeexample}

\begin{codeexample}
typedef struct t__space_expression
   \{
       expression_type *exprs;
       operand_type *vars;
       short n_vars;
   \} space_expression_type;
\end{codeexample}


There are two formats for the \rw{TRANTO} clause destination.   The first format
is a list of assignment statements.   The second format is a space expression.

An example of the first (list) format follows:
\begin{logfileexample}
(00009): NR = 2;
(00010): SPACE = (NP,NFP,(UR:1..NR,UX:ARRAY[1..NR] OF BOOLEAN));
         \vdots
(00017):  ...  TRANTO NFP++,UX[NR+1-III]=TRUE BY FAST III*DELTA;
\end{logfileexample}
This list format example above is
illustrated in Figure \ref{fig:tranto-list-mem}.

\startfig
\begin{fast_picture}{500}{140}
\setboxpos{115}{120}\putuponebox\savFboxpos
\setboxpos{75}{80}\savHboxpos
\leftwidetalltagbox{sex}{}{72}{29}{60}
    \nextFbox\lefttagbox{exprs}{}\outarrowdown{325}{20}
          \stackwideBbox{46}
          \nextBbox\widevalbox{$NFP++$}{46}
          \nextBbox\widevalbox{$[NR+1-III]$}{46}
          \nextBbox\widevalbox{$TRUE$}{46}
    \nextFbox\lefttagbox{vars}{}\outarrowdown{190}{20}
          \stackBbox
          \nextBbox\valbox{}\outarrowmsg{30}{NFP}
          \nextBbox\valbox{}\outarrowmsg{30}{UX[$\uparrow$]}
    \nextFbox\lefttagbox{n\_vars}{2}
\nextHbox\leftwidetagbox{rate-exprs}{}{72}{29}\outarrowdown{100}{50}
          \stackwideBbox{36}
          \nextBbox\widevalbox{$III*DELTA$}{36}
\nextHbox\leftwidetagbox{\#~rate-exprs}{1 (FAST)}{72}{29}
\nextHbox\leftwidetagbox{line~\#}{17}{72}{29}
\end{fast_picture}
\finishfig{TRANTO clause (list format) laid out in memory}{fig:tranto-list-mem}

Notice that, although $n\_vars=2$, there are three expressions in the
array pointed to by $sex.exprs$.   This is because one of the variables in the
array pointed to by $sex.vars$ is for UX which is an array variable.
Two expressions are
stored in the array when an array variable
is encountered in a positional state node.
The first of the two expressions is always the
subscript and
the second is always the value to be stored in the state space.
\begin{tabbing}
\\
\end{tabbing}


An example of the second (space expression) format follows:
\begin{logfileexample}
(00009): NR = 2;
(00010): SPACE = (NP,NFP,(UR:1..NR,UX:ARRAY[1..NR] OF BOOLEAN));
         \vdots
(00017):  ...  TRANTO (,NFP+1,(,2 OF TRUE)) BY FAST III*DELTA;
\end{logfileexample}
Notice that, when entering a destination state, multiple commas in a row
indicate that the values for the state-space variable(s) occupying
the position(s) in question are to remain unchanged after the state transition
is made.
This space expression format example above is
illustrated in Figure \ref{fig:tranto-sexp-mem}.

\startfig
\begin{fast_picture}{500}{140}
\setboxpos{115}{120}\putuponebox\savFboxpos
\setboxpos{75}{80}\savHboxpos
\leftwidetalltagbox{sex}{}{72}{29}{60}
    \nextFbox\lefttagbox{exprs}{}\outarrowdown{335}{20}
          \stackwideBbox{36}
          \nextBbox\widevalbox{$NFP+1$}{36}
          \nextBbox\widevalbox{$[1]$}{36}
          \nextBbox\widevalbox{$TRUE$}{36}
          \nextBbox\widevalbox{$[2]$}{36}
          \nextBbox\widevalbox{$TRUE$}{36}
    \nextFbox\lefttagbox{vars}{}\outarrowdown{190}{20}
          \stackBbox
          \nextBbox\valbox{}\outarrowmsg{30}{NFP}
          \nextBbox\valbox{}\outarrowmsg{30}{UX[$\uparrow$]}
          \nextBbox\valbox{}\outarrowmsg{30}{UX[$\uparrow$]}
    \nextFbox\lefttagbox{n\_vars}{3}
\nextHbox\leftwidetagbox{rate-exprs}{}{72}{29}\outarrowdown{100}{50}
          \stackwideBbox{36}
          \nextBbox\widevalbox{$III*DELTA$}{36}
\nextHbox\leftwidetagbox{\#~rate-exprs}{1 (FAST)}{72}{29}
\nextHbox\leftwidetagbox{line~\#}{17}{72}{29}
\end{fast_picture}
\finishfig{TRANTO clause (space expression format) laid out in memory}{fig:tranto-sexp-mem}

\subsection{The VARIABLE statement data structure}
\label{sec:calc}

The \rw{VARIABLE} statements are parsed and stored in data stuctures
of the following types:
\begin{codeexample}
typedef struct t__calc_assign
   \{
       identifier_info_type *idinfo;  /* <ident> := */
       expression_type *expr;         /*            <expr> */
   \} calc_assign_type;
\end{codeexample}

\begin{codeexample}
typedef struct t__expression
   \{
       operation_type *postfix_ops;
       operation_type *infix_ops;
       operand_type *operands;
       short n_postfix_ops;
       short n_infix_ops;
       short n_operands;
       short source_code_line_number;
       Boolean in_error;
       type_flagword_type rtntype;
   \} expression_type;
\end{codeexample}

\begin{codeexample}
typedef struct t__identifier_info
  \{
     pointer_union_type ptr;    /* address in memory / function-parm-count */
     union
        \{
           struct qqbothidinfqq
              \{
                 dim_pair_type first;
                 dim_pair_type second;
              \} dims;
           dim_pair_type body;
        \} index;
     short scope_level;         /* scope level (negative iff. inactive) */
     char name[IDENT_MAXNCH_P]; /* identifier to search for */
     type_flagword_type flags;  /* type information */
  \} identifier_info_type;
\end{codeexample}
where the data structure for a dimension pair is of the following type:
\begin{codeexample}
typedef struct t__dim_pair
  \{
    Subscript lower;/* lower dimension (ARRAY) body-index (FUNCTION/IMPLICIT)*/
    Subscript upper;/* upper dimension (ARRAY) body-index (FUNCTION/IMPLICIT)*/
  \} dim_pair_type;
\end{codeexample}

and where the data structure for a pointer union is of the following type:
\begin{codeexample}
typedef union t__pointer_union
   \{
     relative_address_type relative_address;
     short parameter_count;
     void *vvv;               /* included for completeness */
     Boolean *bbb;            /* used when BOOL_TYPE */
     state_offset_type *sss;  /* used when SSVAR_TYPE */
     char *ccc;               /* used when CHAR_TYPE */
     int_type *iii;           /* used when INT_TYPE */
     real_type *rrr;          /* used when REAL_TYPE */
   \} pointer_union_type;
\end{codeexample}



As an example, consider:
\begin{logfileexample}
(00010): VARIABLE NWP[NFP] = NP-NFP;
\end{logfileexample}

The layout of the above \rw{VARIABLE} is illustrated
in Figure \ref{fig:variable}.

\startfig
\begin{fast_picture}{200}{50}
\setboxpos{40}{30}\putuponebox\savFboxpos
\nextFbox\lefttagbox{idinfo}{}\outarrowmsg{100}{NWP}
\nextFbox\lefttagbox{expr}{}\outarrowmsg{40}{NP-NFP}
\end{fast_picture}
\finishfig{VARIABLE statement laid out in memory}{fig:variable}

\subsection{The Block IF statement data structure}
\label{sec:blockif}

The block IF statement is parsed and stored in data
structures of the following types:
\begin{codeexample}
typedef struct t__block_if
   \{
       expression_type *then_test;    /* boolean expression for THEN */
       instruction_type *then_clause; /* code for THEN clause */
       instruction_type *else_clause; /* code for ELSE clause */
   \} block_if_type;
\end{codeexample}

\begin{codeexample}
typedef struct t__expression
   \{
       operation_type *postfix_ops;
       operation_type *infix_ops;
       operand_type *operands;
       short n_postfix_ops;
       short n_infix_ops;
       short n_operands;
       short source_code_line_number;
       Boolean in_error;
       type_flagword_type rtntype;
   \} expression_type;
\end{codeexample}

\begin{codeexample}
typedef struct t__instruction_pointer_union
   \{
       void *vvv;                      /* to cast to block_if_type, etc. */
       relative_address_type reladdr;  /* relative address of code */
   \} instruction_pointer_union_type;
\end{codeexample}

\begin{codeexample}
typedef struct t__instruction
   \{
       instruction_pointer_union_type ptr;
       opcode_type opcode;      /* instruction operation code */
   \} instruction_type;          /*    for_loop_type, assert_type, ... */
\end{codeexample}


The ``then\_test'' field is a pointer to a Boolean expression to be evaluated
in order to decide whether to execute the \rw{THEN}
or \rw{ELSE} clause code.   If the
expression pointed to evaluates to \rw{TRUE}, then the \rw{THEN} clause code
is executed, otherwise the \rw{ELSE} clause code is executed.

The ``then\_clause'' field is a pointer to the beginning of the subroutine
which contains the rule-section instructions of the \rw{THEN} clause
and pertaining to the current code
section, i.e., a block \rw{IF} in the \rw{DEATHIF} section will
have a \rw{THEN} clause which points to a subroutine in the \rw{DEATHIF}
section.

The ``else\_clause'' field is a pointer to the beginning of the subroutine
which contains the rule-section instructions of the \rw{ELSE} clause
and pertaining to the current code section.

As an example, consider the following block if:
\begin{logfileexample}
(0010):    IF B=6 THEN;
(0011):       TRANTO A=A-1 BY FOO1;
(0012):       TRANTO A=A-1,B=B-1 BY FOO2;
(0013):    ENDIF;
\end{logfileexample}

Figure \ref{fig:blockif-mem} illustrates the memory layout of the block \rw{IF}
beginning on line 10.   This memory layout is also detailed with an excerpt
from the memory load
map (\option{loadmap} option) in figure \ref{fig:blockif-map}.
Note that the \rw{ENDIF} matches the \rw{THEN} and
not the \rw{IF}.
\startfig
\begin{fast_picture}{270}{300}
\setboxpos{0}{270}
\savFboxpos\widevalbox{}{20}\outarrowdownleftdown{0}{15}{30}{5}
     \stackwideAbox{30}\nextAbox\widevalbox{$B=6$}{30}
     \unsavFboxpos\putuponetallbox{13}\widenoboxtall{IF}{20}{10}
\unsavFboxpos\wideboxtoright{20}\widevalbox{}{20}\savFboxpos
     \outarrowdownleftdown{0}{50}{70}{40}
     \stackwideAbox{30}
         \nextAbox\widevalbox{TRANTO}{30}\wideboxtoright{30}
              \savBboxpos
                  \widevalbox{}{20}\boxcendot
                  \boxcenlineup{25}\shiftboxcenup{25}
                       \dotoff\outarrowdownrightdown{120}{65}{50}{20}\doton
                       \stackwideCbox{30}
                           \nextCbox\widevalbox{$A-1$}{30}
              \unsavBboxpos\wideboxtoright{20}\savBboxpos
                  \widevalbox{}{20}\boxcendot
                  \boxcenlineup{15}\shiftboxcenup{15}
                       \dotoff\outarrowdownrightdown{50}{65}{20}{10}\doton
                       \stackwideCbox{10}
                            \nextCbox\widevalbox{$A$}{10}
              \unsavBboxpos\wideboxtoright{20}\valbox{1}
         \nextAbox\widevalbox{TRANTO}{30}\wideboxtoright{30}
              \savBboxpos
                  \widevalbox{}{20}\boxcendot
                  \boxcenlinedown{40}\shiftboxcenup{-40}
                       \dotoff\outarrowdownrightdown{50}{50}{120}{20}\doton
                       \stackwideCbox{30}
                            \nextCbox\widevalbox{$A-1$}{30}
                            \nextCbox\widevalbox{$B-1$}{30}
              \unsavBboxpos\wideboxtoright{20}\savBboxpos
                  \widevalbox{}{20}\boxcendot
                  \boxcenlinedown{30}\shiftboxcenup{-30}
                       \boxcenlineright{30}\shiftboxcen{30}
                       \boxcenlinedown{45}\shiftboxcenup{-45}
                       \boxcenlineright{140}\shiftboxcen{140}
                       \boxcenlinedown{90}\shiftboxcenup{-90}
                       \boxcenlineleft{80}\shiftboxcen{-80}
                       \boxcenlineup{65}\shiftboxcenup{65}
                       \boxcenlineleft{20}\shiftboxcen{-20}
                       \dotoff\droparrow{10}\doton
                       \stackwideCbox{10}
                            \nextCbox\widevalbox{$A$}{10}
                            \nextCbox\widevalbox{$B$}{10}
              \unsavBboxpos\wideboxtoright{20}\valbox{2}
     \unsavFboxpos\putuponetallbox{13}\widenoboxtall{THEN}{20}{10}
\unsavFboxpos\wideboxtoright{20}\widevalbox{}{20}\savFboxpos
     \outarrowdownrightdown{0}{30}{50}{10}\groundunderarrow
     \unsavFboxpos\putuponetallbox{13}\widenoboxtall{ELSE}{20}{10}
\end{fast_picture}
\finishfig{Sample block IF laid out in memory}{fig:blockif-mem}
\startfig
\begin{logfileexample}

00000211:   [post=(00000465,3),in=(00000468,3),op=(000003B7,2),
             line#=10,err=F,returns(0x0a,(<boolean>,<expr-var>))]
             B=6
00000227:   [post=(0000046B,3),in=(0000046E,3),op=(000003BF,2),
             line#=11,err=F,returns(0x0b,(<integer>,<expr-var>))]
            A-1
00000253:   [post=(00000473,3),in=(00000476,3),op=(000003CF,2),
             line#=12,err=F,returns(0x0b,(<integer>,<expr-var>))]
            A-1
00000269:   [post=(00000479,3),in=(0000047C,3),op=(000003D7,2),
             line#=12,err=F,returns(0x0b,(<integer>,<expr-var>))]
            B-1
0000023D:   [post=(00000471,1),in=(00000472,1),op=(000003CB,1),
             line#=11,err=F,returns(0x04,(<real>))]
            FOO1
0000027F:   [post=(0000047F,1),in=(00000480,1),op=(000003E7,1),
             line#=12,err=F,returns(0x04,(<real>))]
            FOO2

000003C7:   A<6>
000003DF:   A<6>
000003E3:   B<7>

000004FF:   (TRANTO (exprs=00000227,vars=000003C7,#vars=1)
                 BY 1@0000023D (line#11))
00000511:   (TRANTO (exprs=00000253,vars=000003DF,#vars=2)
                 BY 1@0000027F (line#12))

0000055F:   (IF 00000211 THEN GOSUB 00000619)

00000619:   TRANTO 000004FF
00000623:   TRANTO 00000511
0000062D:   RETURN

\end{logfileexample}
\finishfig{Sample memory map of corresponding block IF}{fig:blockif-map}

\subsection{The FOR loop statement data structure}
\label{sec:forloop}

The FOR statement is parsed and stored in data structures of the
following types:
\begin{codeexample}
typedef struct t__for_loop
   \{
       identifier_info_type *ident;  /* index variable */
       set_range_type *set_ranges;   /* pointer to array of IN ranges */
       short set_range_count;        /* count of number of IN ranges */
       instruction_type *body;       /* pointer to BODY of loop */
   \} for_loop_type;
\end{codeexample}

\begin{codeexample}
typedef struct t__set_range
   \{
       expression_type *lower_bound;
       expression_type *upper_bound;
   \} set_range_type;
\end{codeexample}

\begin{codeexample}
typedef struct t__expression
   \{
       operation_type *postfix_ops;
       operation_type *infix_ops;
       operand_type *operands;
       short n_postfix_ops;
       short n_infix_ops;
       short n_operands;
       short source_code_line_number;
       Boolean in_error;
       type_flagword_type rtntype;
   \} expression_type;
\end{codeexample}

\begin{codeexample}
typedef struct t__instruction_pointer_union
   \{
       void *vvv;                      /* to cast to block_if_type, etc. */
       relative_address_type reladdr;  /* relative address of code */
   \} instruction_pointer_union_type;
\end{codeexample}

\begin{codeexample}
typedef struct t__instruction
   \{
       instruction_pointer_union_type ptr;
       opcode_type opcode;      /* instruction operation code */
   \} instruction_type;          /*    for_loop_type, assert_type, ... */
\end{codeexample}

\begin{codeexample}
typedef struct t__identifier_info
  \{
     pointer_union_type ptr;    /* address in memory / function-parm-count */
     union
        \{
           struct qqbothidinfqq
              \{
                 dim_pair_type first;
                 dim_pair_type second;
              \} dims;
           dim_pair_type body;
        \} index;
     short scope_level;         /* scope level (negative iff. inactive) */
     char name[IDENT_MAXNCH_P]; /* identifier to search for */
     type_flagword_type flags;  /* type information */
  \} identifier_info_type;
\end{codeexample}
where the data structure for a dimension pair is of the following type:
\begin{codeexample}
typedef struct t__dim_pair
  \{
    Subscript lower;/* lower dimension (ARRAY) body-index (FUNCTION/IMPLICIT)*/
    Subscript upper;/* upper dimension (ARRAY) body-index (FUNCTION/IMPLICIT)*/
  \} dim_pair_type;
\end{codeexample}

and where the data structure for a pointer union is of the following type:
\begin{codeexample}
typedef union t__pointer_union
   \{
     relative_address_type relative_address;
     short parameter_count;
     void *vvv;               /* included for completeness */
     Boolean *bbb;            /* used when BOOL_TYPE */
     state_offset_type *sss;  /* used when SSVAR_TYPE */
     char *ccc;               /* used when CHAR_TYPE */
     int_type *iii;           /* used when INT_TYPE */
     real_type *rrr;          /* used when REAL_TYPE */
   \} pointer_union_type;
\end{codeexample}



The ``ident'' field is a pointer into the identifier table.   This is the
index variable which varies for all values in the set.

The ``set\_ranges'' field is a pointer to an array of set ranges (lower/upper
range integer value pairs).

The ``set\_range\_count'' field is a count of the number of set ranges in
the ``set\_ranges'' array.

The ``body'' field is a pointer to the beginning of the subroutine which
contains the rule-section instructions which fall between the \rw{FOR} and
the \rw{ENDFOR} and which pertain to the current code section.


As an example, consider the following for:
\begin{logfileexample}
(0005): FOR J IN [1..3,7..9]
(0006):     IF NC[10-J] = 0 TRANTO (20,8 OF 0) BY (J*3)*LAMBDA;
(0007): ENDFOR;
\end{logfileexample}

Figure \ref{fig:forloop-mem} illustrates the memory layout of the \rw{FOR}
beginning on line 5.   This memory layout is also detailed with an excerpt
from the memory load
map (\option{loadmap} option) in figure \ref{fig:forloop-map}.


\startfig
\begin{fast_picture}{270}{300}
\setarrow{50}{270}
\stackwideAbox{30}\nextAbox\leftwidetagbox{FOR}{}{18}{30}\outarrowmsg{50}{J}
                  \nextAbox\widevalbox{}{30}
                           \outarrowdownrightdown{40}{30}{160}{10}
                           \stackwideDbox{20}
                                \nextDbox\widevalbox{1 $..$ 3}{20}
                                \nextDbox\widevalbox{7 $..$ 9}{20}
                  \nextAbox\widevalbox{2}{30}
                  \nextAbox\widevalbox{}{30}
                           \outarrowdown{20}{30}
                           \stackwideDbox{30}
                                \nextDbox\widevalbox{IF}{30}
                                   \wideboxtoright{30}
                                   \widevalbox{}{30}\outarrowdown{60}{30}
                                      \stackwideEbox{130}
                                      \nextEbox\widevalbox{NC[10-J] $=$ 0}{130}
   \nextEbox\leftwidetagbox{then}{TRANTO (20,8 OF 0) BY (J*3)*LAMBDA;}{20}{130}
   \nextEbox\leftwidetagbox{else}{}{20}{130}
\end{fast_picture}
\finishfig{Sample FOR laid out in memory}{fig:forloop-mem}

\startfig
\begin{logfileexample}
    00000218:   [post=(00000597,1),in=(00000598,1),op=(000004D0,1),
                 line#=5,err=F,returns(0x03,(<integer>))]
                 1
    00000230:   [post=(00000599,1),in=(0000059A,1),op=(000004D4,1),
                 line#=5,err=F,returns(0x03,(<integer>))]
                 3
    00000248:   [post=(0000059B,1),in=(0000059C,1),op=(000004D8,1),
                 line#=5,err=F,returns(0x03,(<integer>))]
                 7
    00000260:   [post=(0000059D,1),in=(0000059E,1),op=(000004DC,1),
                 line#=5,err=F,returns(0x03,(<integer>))]
                 9
    00000278:   [post=(0000059F,7),in=(000005A6,8),op=(000004E0,4),
                 line#=6,err=F,returns(0x0a,(<boolean>,<expr-var>))]
                 NC[10-J]=0
    00000290:   [post=(000005AE,1),in=(000005AF,1),op=(000004F0,1),
                 line#=6,err=F,returns(0x03,(<integer>))]
                 1
    ...

    00000440:   [post=(000005D2,6),in=(000005D8,7),op=(0000055C,3),
                 line#=6,err=F,returns(0x0c,(<real>,<expr-var>))]
                 (J*3)*LAMBDA

    ...

    000005E4:   (00000218 .. 00000230)   (00000248 .. 00000260)

    0000061C:   (TRANTO (exprs=00000290,vars=00000538,#vars=9) BY 1@00000440 (line#6))

    ...

    00000630:   (IF 00000278 THEN GOSUB 000006E8)

    0000063C:   (J<16> IN [2@000005E4] GOSUB 00000700)

    ...


    000006E8:   TRANTO 0000061C
    000006F4:   RETURN
    00000700:   IF 00000630
    0000070C:   RETURN

    00000718:   LOOP 0000063C
    00000724:   RETURN
\end{logfileexample}
\finishfig{Sample memory map of corresponding FOR}{fig:forloop-map}

\subsection{The FOR index repetition information data structure}
\label{sec:dostuff}

When generating a model file inside of the body of a \rw{FOR}, the index
variable is stored in a data structure of the following type:
\begin{codeexample}
typedef struct t__do_code_stuff
    \{
         identifier_info_type *do_idinfo;
         Subscript do_index;
    \} do_code_stuff_type;
\end{codeexample}

Which references the following data types:
\begin{codeexample}
typedef struct t__identifier_info
  \{
     pointer_union_type ptr;    /* address in memory / function-parm-count */
     union
        \{
           struct qqbothidinfqq
              \{
                 dim_pair_type first;
                 dim_pair_type second;
              \} dims;
           dim_pair_type body;
        \} index;
     short scope_level;         /* scope level (negative iff. inactive) */
     char name[IDENT_MAXNCH_P]; /* identifier to search for */
     type_flagword_type flags;  /* type information */
  \} identifier_info_type;
\end{codeexample}
where the data structure for a dimension pair is of the following type:
\begin{codeexample}
typedef struct t__dim_pair
  \{
    Subscript lower;/* lower dimension (ARRAY) body-index (FUNCTION/IMPLICIT)*/
    Subscript upper;/* upper dimension (ARRAY) body-index (FUNCTION/IMPLICIT)*/
  \} dim_pair_type;
\end{codeexample}

and where the data structure for a pointer union is of the following type:
\begin{codeexample}
typedef union t__pointer_union
   \{
     relative_address_type relative_address;
     short parameter_count;
     void *vvv;               /* included for completeness */
     Boolean *bbb;            /* used when BOOL_TYPE */
     state_offset_type *sss;  /* used when SSVAR_TYPE */
     char *ccc;               /* used when CHAR_TYPE */
     int_type *iii;           /* used when INT_TYPE */
     real_type *rrr;          /* used when REAL_TYPE */
   \} pointer_union_type;
\end{codeexample}



The ``do\_idinfo'' field is a pointer to the identifier table entry for
the \rw{FOR} index variable.

The ``do\_index'' field holds the value of the index variable which is
currently in effect.

When a \rw{FOR} instruction is encountered, the subroutine for the body
of the construct is executed for each value in the set of values to use.
The index variable and the current value are stored in
the ``do\_stuff\_type'' data structure and passed to the subroutine
evaluator.

\subsection{The space expression list data structure}
\label{sec:elist}

% A typical \acronym{ASSIST} input file can contain many subroutines containing
% sequences of rules.   A subroutine could be the body of a \rw{FOR}, the
% body of a \rw{THEN} or \rw{ELSE} clause, or the main part of the rule
% section itself.
% 
% Each subroutine of source code is parsed and held temporarily in
% expression lists using data structures of the following types:
% 

When parsing a list of expressions for the destination of a \rw{TRANTO},
information about the state-space variable which is being modified is
stored in an ``elist'' data structure of the following type:
\begin{codeexample}
typedef struct t__elist
   \{  /* ordering of shorts is important */
      short iii;     /* first.lower */
      short iiiend;  /* first.upper */
      short jjj;     /* second.lower */
      short jjjend;  /* second.upper */
      short knt;
      short *which;
      identifier_info_type *idinfo;
      dim_pair_type q1st; /* first */
      dim_pair_type q2nd; /* second */
      dim_pair_type *q_1st_or_2nd;
      Boolean const1;
      Boolean const2;
      Boolean *qconst;
      Boolean is_var;
   \} elist_type;

static elist_type elist;
\end{codeexample}

which makes use of additional data structures of the following types:
\begin{codeexample}
typedef struct t__dim_pair
  \{
    Subscript lower;/* lower dimension (ARRAY) body-index (FUNCTION/IMPLICIT)*/
    Subscript upper;/* upper dimension (ARRAY) body-index (FUNCTION/IMPLICIT)*/
  \} dim_pair_type;
\end{codeexample}

\begin{codeexample}
typedef struct t__identifier_info
  \{
     pointer_union_type ptr;    /* address in memory / function-parm-count */
     union
        \{
           struct qqbothidinfqq
              \{
                 dim_pair_type first;
                 dim_pair_type second;
              \} dims;
           dim_pair_type body;
        \} index;
     short scope_level;         /* scope level (negative iff. inactive) */
     char name[IDENT_MAXNCH_P]; /* identifier to search for */
     type_flagword_type flags;  /* type information */
  \} identifier_info_type;
\end{codeexample}
where the data structure for a dimension pair is of the following type:
\begin{codeexample}
typedef struct t__dim_pair
  \{
    Subscript lower;/* lower dimension (ARRAY) body-index (FUNCTION/IMPLICIT)*/
    Subscript upper;/* upper dimension (ARRAY) body-index (FUNCTION/IMPLICIT)*/
  \} dim_pair_type;
\end{codeexample}

and where the data structure for a pointer union is of the following type:
\begin{codeexample}
typedef union t__pointer_union
   \{
     relative_address_type relative_address;
     short parameter_count;
     void *vvv;               /* included for completeness */
     Boolean *bbb;            /* used when BOOL_TYPE */
     state_offset_type *sss;  /* used when SSVAR_TYPE */
     char *ccc;               /* used when CHAR_TYPE */
     int_type *iii;           /* used when INT_TYPE */
     real_type *rrr;          /* used when REAL_TYPE */
   \} pointer_union_type;
\end{codeexample}



The ``iii'' field is used for the lower bound of the first subscript of an
array.
It is used in a loop to check during parse time for an index out of bounds.

The ``iiiend'' field is used for the upper bound of the first subscript of
an array.
It is used in a loop to check during parse time for an index out of bounds.

The ``jjj'' field is used for the lower bound of the second subscript of
an array.
It is used in a loop to check during parse time for an index out of bounds.

The ``jjjend'' field is used for the upper bound of the second subscript of
an array.
It is used in a loop to check during parse time for an index out of bounds.

The ``knt'' field is used to hold the number of subscripts.   The number
one is used for a singly subscripted array and the number two is used
for a doubly subscripted array.   The number \rw{SIMPLE\_IDENTIFIER} is
used for non-array state-space variables.

The ``which'' field is a pointer to either
the first (``iii'') or second (``jjj'') subscript bounds.

The ``idinfo'' field points into the identifier table to indicate which
state-space variable is being updated during the transition.

The ``q1st'' field is a copy of the dimension pair for the first subscript
of the state-space variable being updated during the transition.

The ``q2nd'' field is a copy of the dimension pair for the second subscript
of the state-space variable being updated during the transition.

The ``q\_1st\_or\_2nd'' field is a pointer to either
the ``q1st'' or the ``q2nd'' field.

The ``const1'' field indicates whether the index reference for the first
subscript to the state-space variable being updated is a constant expression.

The ``const2'' field indicates whether the index reference for the second
subscript to the state-space variable being updated is a constant expression.

The ``qconst'' field is a pointer to either the ``const1'' or the ``const2''
field.

The ``is\_var'' field indicates that at least one of the first and the second
subscript expressions is non-constant.

\subsection{The built-in function parameter information data structure}
\label{sec:bin}

When parsing built-in functions, there is a lookup table with information
about the quantity, types, and kinds of valid parameters which can
be passed to each built-in function in question.   Each entry in the table
is of the following type:
\begin{codeexample}
typedef struct t__built_in_parm_info
   \{
       short parameter_count;
       type_flagword_type parameter_type;
       type_flagword_type return_type;
       operation_type opcode;
       operation_type aux_opcode;
       char label[16];
   \} built_in_parm_info_type;
\end{codeexample}



The ``parameter\_count'' field indicates the number of parameters that
the built-in function requires.    An error message is printed if the
number of parameters passed does not equal the number of parameters
expected.    Certain special values (usually negative numbers to distinguish
from actual counts) are allowed.   Special values include ``VARIAB\_LENG''
which is used for list functions which can take 1 or more parameters and
``ARR\_N\_IX'' for functions which require the name of an array as the first
parameter and an index as the second parameter.    There are currently no
built-in functions which are of length ``ARR\_N\_IX'' although the concept
applies to some of the arithmetic logic unit (ALU) operations such as
ROWCOUNT, COLCOUNT, ROWSUM, COLSUM, ROWMIN, COLMIN, ROWMAX, COLMAX, ROWANY,
COLANY, ROWALL, and COLALL.

The ``parameter\_type'' field indicates the type of parameter(s) required.
Unless a type promotion can be made, an error message is printed
if the wrong type of a parameter is passed to a built-in function.   The
ARRAY\_TYPE bit is set if the name of an array is legal.
If the ``parameter\_count'' is VARIAB\_LENG and the ARRAY\_TYPE bit is set,
the an array is optional.
If the ``parameter\_count'' is positive and the ARRAY\_TYPE bit is set,
then an array is required.
If the simple type portion (low three bits) is zero, then the function
can accept integers and/or reals.

The ``return\_type'' is in the range 0..7 and gives the simple type of the 
value which is returned by the function.   If set to EMPTY\_TYPE, then
the ``opcode'' field applies when all parameters are integers and the
``aux\_opcode'' field applies when at least one parameter is a real.
In the case of Boolean functions such as \rw{COUNT}, if set to EMPTY\_TYPE,
then the ``opcode'' field applies when more than one parameter is passed
and the ``aux\_opcode'' field applies when only one parameter is passed.

The ``opcode'' field specifies the operation code to use
under most circumstances
in the postfix
expression to evaluate the function result.

The ``aux\_opcode'' field specifies the operation code to use
under certain special circumstances
in the postfix
expression to evaluate the function result.

The ``label'' field gives the textual name of the built-in function as
a convienience for printing out error and warning messages.

\subsection{The value union data structure}
\label{sec:valunion}

When evaluating postfix expressions, it is necessary to maintain a value stack
in addition to an operation stack.   Since the values on the stack could
be of various types, a union is used for each stack entry.    This union
is called the ``value union'' and is of the following type:

\begin{codeexample}
typedef union t__value_union
   \{
     Boolean bbb;            /* used when BOOL_TYPE */
     char ccc;               /* used when CHAR_TYPE */
     int_type iii;           /* used when INT_TYPE */
     real_type rrr;          /* used when REAL_TYPE */
     state_offset_type sss;  /* used when SSVAR_TYPE */
     pointer_union_type ptr; /* used when ARRAY_TYPE bit is set */
#if defined(INT_32_BIT) || defined(INT_16_BIT)
     struct qqiiis\{
        int_type iiia;
        int_type iiib;\} pair;
#endif
   \} value_union_type;
\end{codeexample}

which makes use of additional data structures of the following types:
\begin{codeexample}
typedef union t__pointer_union
   \{
     relative_address_type relative_address;
     short parameter_count;
     void *vvv;               /* included for completeness */
     Boolean *bbb;            /* used when BOOL_TYPE */
     state_offset_type *sss;  /* used when SSVAR_TYPE */
     char *ccc;               /* used when CHAR_TYPE */
     int_type *iii;           /* used when INT_TYPE */
     real_type *rrr;          /* used when REAL_TYPE */
   \} pointer_union_type;
\end{codeexample}

\begin{codeexample}
typedef short ssvar_value_type;
typedef struct t__state_offset
   \{
      ssvar_value_type minval;
      ssvar_value_type maxval;
      bitsize_type bit_offset;
      bitsize_type bit_length;
   \} state_offset_type;
\end{codeexample}


The ``bbb'' field applies when Boolean data is in the stack element in
question.

The ``ccc'' field applies when a single character is in the stack element in
question.   Although the current version of \acronym{ASSIST} does not use
character data in expressions, this field is included in the union for
completeness.

The ``iii'' field applies when integer (long) data is in the stack element in
question.

The ``rrr'' field applies when real (double) data is in the stack element in
question.

The ``sss'' field applies when state-space variable offset data is in
the stack element in question.

The ``ptr'' field applies when the starting address of an array is in
the stack element in question.

The ``pair.iiia'' and ``pair.iiib'' fields are used for efficent copying,
pushing, and poppping of stack elements.   On VAX and SUN systems, two
assignment statements is much faster than a single call to ``memcpy''.
There are two versions of the ``val\_union\_cpy'' macro are defined
in ``cm\_types.h''.   One which applies only on systems with 16 or 32 bit
integers and performs two assignment statements, and the other which
applies on all other architectures and does a memcpy.

\subsection{The binary operand pair data structure}
\label{sec:bopi}

Certain data structures are used to hold information about the left and right
side of infix/postfix operands as pertaining to the current expression being
parsed by the recursive-descent parser.   These data structures are:
\begin{codeexample}
typedef struct t__binary_operand_item_info
   \{
      short ixpo;
      short ixin;
      type_flagword_type type;
      type_flagword_type comp;
      type_flagword_type spec;
   \} binary_operand_item_info_type;

typedef struct t__binary_operand_pair_info
   \{
      binary_operand_item_info_type item[2];
      type_flagword_type ans;
      type_flagword_type spcans;
   \} binary_operand_pair_info_type;
\end{codeexample}

;
The ``ixpo'' field gives the index in the
postfix list of the
current expression being parsed
where the unary/binary operand operation (``V'') can be found.

The ``ixin'' field gives the index in the
infix list of the
current expression being parsed
where the unary/binary operand operation (``V'') can be found.

The ``type'' field gives the type of the operand in question.

The ``comp'' field gives the computational (or simple) type of the
operand in question.   The computational type consists of the low three bits of
the type.

The ``spec'' field give the special type of the operand in question.
The special type consists of all bits of the type except for the computational
bits.

The ``item'' field is an array containing the above information for each
of both the left and right operands of the binary operation being parsed.
In the case of a unary operation, the second slot in the array is not used
and is not guaranteed to contain any meaningful information.

The ``ans'' field is used to store the resultant type of the operation being
parsed.   This starts out as a computational type but is OR'ed with the
``spcans'' field, depending upon which operation is being parsed, before
being returned to the calling function.

The ``spcans'' field is used to store the special type bits of the operation
being parsed.

\subsection{The reserved word operator lookup data structure}
\label{sec:rwoplup}

The reserved word operator lookup table is used by the lexical token
scanner to translate from a reserved word which stands for an arithmetic
or logical operation to the operation itself.    Each entry in the table
is of the following type:
\begin{codeexample}
typedef struct t__rw_operator_lookup
   \{
      rwtype rwsrc;
      token tokdest;
   \} rw_operator_lookup_type;
\end{codeexample}


The ``rwsrc'' field specifies the scanned reserved word which must
be translated from a reserved-word token to an arithmetic or logical
operation token.

The ``tokdest'' field specifies the corresponding token for the translation.

For example, the \rw{RW\_AND} reserved word which is scanned
from the word \rw{AND} in the input file is translated to
the token \rw{TK\_AND} which stands for the logical \rw{``\&''} operation.

\subsection{The reserved word lookup data structure}
\label{sec:rwlup}

The reserved word lookup table is used by the lexical token
scanner to translate from a scanned identifier to its corresponding
reserved word.   Identifiers which are not in the table are treated as
identifiers.   Identifiers which are in the table are translated to
reserved words.  Each entry in the table
is of the following type:
\begin{codeexample}
typedef struct t__reserved_word_lookup
   \{
      char text[15];
      rwtype rw;
   \} reserved_word_lookup_type;
\end{codeexample}


The ``text'' field contains the name of the identifier which must be
translated to a reserved word.

The ``rw'' field contains the corresponding reserved word to substitute
for the identifier during the translation.


For example, the identifier ``\rw{TRANTO}'' is translated into the 
reserved word ``\rw{RW\_TRANTO}''.

\subsection{The token lookup data structure}
\label{sec:tokenlup}

The token lookup table is used by the lexical token
scanner to translate from a symbolic character sequence
to its corresponding token.
Each entry in the table is of the following type:
\begin{codeexample}
typedef struct t__token_lookup
   \{
     char token[3];
     token value;
   \} token_lookup_type;
\end{codeexample}


The ``token'' field gives the text of the symbolic token which must be
translated into a single token.

The ``value'' field give the corresponding token which must be substituted
for the contiguous character sequence in the symbolic ``token''.

For example, the contiguous character sequence ``$==$'' is translated
into the ``TK\_BOOL\_EQ'' token.   Also, the contiguous character
sequence ``$<=$'' is translated into the ``TK\_LE'' token and the sequence
``$**$'' is translated into the ``TK\_POW'' token for exponentiation.

\subsection{The scanning character information data structure}
\label{sec:scanning}

The scanning character information data structure is used to store
information about the current character being scanned and the lookahead
character.   The \acronym{ASSIST} parser, in most instances, is a single
character lookahead scanner.    In some Boolean expressions, the scanner
must look ahead to the character following a matching right parenthesis.
This is done using ``ftell'' to remember where in the file the scanner
left off and ``fseek'' to get back after looking ahead.

The data structure is of the following type:
\begin{codeexample}
typedef struct t__scanning_character_info
   \{
        short current_ch_lno;
        short lookahead_ch_lno;
        short current_ch_pos;
        short lookahead_ch_pos;
        char current_ch;
        char lookahead_ch;
   \} scanning_character_info_type;
\end{codeexample}


The ``current\_ch\_lno'' field gives the input file line number on which
the current character resides.

The ``lookahead\_ch\_lno'' field gives the input file line number on which
the lookahead character resides.

The ``current\_ch\_pos'' field gives the index in the ``current\_ch\_lno''
where the current character resides.

The ``lookahead\_ch\_pos'' field gives the index in the ``lookahead\_ch\_lno''
where the lookahead character resides.

The ``current\_ch'' field gives the current character itself.

The ``lookahead\_ch'' field gives the lookahead character itself.

\subsection{Mapping of a program into memory}
\label{sec:codeexample}

After an input file has been parsed, ASSIST exits if any syntax errors
were detected.   In the absence of any syntax errors, the rules and related
information are mapped into memory in preparation for model generation.

Memory is allocated and divided into the following sections which are
listed in sequential order:
\begin{itemize}
\item Real constants
\item Integer constants
\item Boolean constants
\item State-space variable offsets
\item Character strings
\item Expressions
\item Expression operands
\item Expression operations
\item Space variable information
\item Set range expression pointers
\item State-space picture data
\item TRANTO clause data
\item Block and TRANTO if data
\item For range data
\item Preamble code
\item ASSERT code
\item DEATHIF code
\item PRUNEIF code
\item TRANTO code
\item Identifier table
\end{itemize}

Consider the following example of a complete {\bf ASSIST} input file listing:
\begin{logfileexample}
(0001): \{
(0002)X     This system describes an NP-ad (e.g, quintad) with processors which
(0003)X     fail at the rate LAMBDA and recover at the FAST rate DELTA.
(0004)X \}
(0005): LAMBDA = 3.0E-4;
(0006): DELTA = 1.0E10;
(0007): NP = 5;
(0008): SPACE = (NWP:0..NP,NFP:0..NP,FAILED:ARRAY[1..NP] OF BOOLEAN);
(0009): START = (NP,0,NP OF FALSE);
(0010): DEATHIF (NFP>=NWP);
(0011): FOR IX IN [1..NP]
(0012):     IF (NWP>0) THEN
(0013):        IF (NOT FAILED[IX]) TRANTO NWP--,NFP++,FAILED[IX]=TRUE BY LAMBDA;
(0014):     ENDIF;
(0015):     IF (NFP>0) THEN
(0016):        IF (FAILED[IX]) TRANTO NWP++,NFP--,FAILED[IX]=FALSE BY FAST DELTA;
(0017):     ENDIF;
(0018): ENDFOR;
(0019): DEATHIF (NFP>=NWP);
\end{logfileexample}

This example is interesting because, although it is fairly straightforward,
it has IF's nested inside of a FOR as well as two IF's in sequence.

Note the inclusion of the redundant \rw{DEATHIF's} on lines 10 and 19.  This
was intentional in order to illustrate layou in memory due to placement in
the input file.

Because memory is mapped using a lot of pointers and symbology, a general
description with illustrations will precede the more detailed ones.
\\
The following is a very general synopsis of how the data and code will be
laid out in memory:
The following is a very general synopsis of how the data and code will be
laid out in memory:
\[ \left. {\mbox{ \begin{tabular}{c}
3.000000000000000e$-$04 \\
3.000000000000000e$-$04 \\
1.000000000000000e$+$10 \\
1.000000000000000e$+$10 \\
\end{tabular} }} \right\}~~~ \mbox{reals} \]
\[ \left. {\mbox{ \begin{tabular}{c}
  5 \\
  5 \\
  0 \\
  1 \\
  2 \\
  3 \\
  4 \\
  0 \\
\end{tabular} }} \right\} ~~ \mbox{integers} \]
\[ \left. {\mbox{ \begin{tabular}{c}
 FALSE \\
 TRUE \\
 FALSE \\
 FALSE \\
 FALSE \\
\end{tabular} }} \right\} ~~ \mbox{Booleans} \]
\[ \left. {\mbox{ \begin{tabular}{c}
 (0..5,3@0) \\
 (0..5,3@3) \\
 (0..1,1@6) \\
\end{tabular} }} \right\} ~~ \mbox{state offsets} \]
\[ \left. {\mbox{ \begin{tabular}{c}
   NP \\
   0 \\
   1 \\
   FALSE \\
   2 \\
   FALSE \\
   3 \\
   FALSE \\
   4 \\
   FALSE \\
   5 \\
   FALSE \\
   (NFP$>$=NWP) \\
   1 \\
   NP \\
   (NWP$>$0) \\
   ($\sim$FAILED[IX]) \\
   NWP$-$$-$ \\
   NFP$+$$+$ \\
   IX \\
   TRUE \\
   LAMBDA \\
   (NFP$>$0) \\
   (FAILED[IX]) \\
   NWP$+$$+$ \\
   NFP$-$$-$ \\
   IX \\
   FALSE \\
   DELTA \\
   (NFP$>$=NWP) \\
\end{tabular} }} \right\} ~~ \mbox{expressions} \]

The preceding data is followed by code data structures which is in turn
followed by the code itself.   In actuality, the expression tokens are not
stored in memory in the sequence in which they are printed.   There are
pointers which point to an infix operation list, a postfix operation list,
and and identifier/value operand list.   The following diagram shows this
in more detail:
\begin{fast_picture}{500}{220}
\setboxpos{50}{190}\savFboxpos\savGboxpos
\widenobox{...}{1}
% 000001BD: & (000003E1,4) & (000003E5,7) & (00000355,2) & 13 & F & $<$boolean$>$,$<$expr$-$var$>$ \\
\nextGbox\nextFbox
          \leftwidetagbox{000001BD:}{000003E1}{72}{24}
                          \savBboxpos\fallarrowright{140}{148}
                                \putboxafterarrow\valbox{V}
                                \boxtoright\valbox{V}
                                \boxtoright\valbox{[~]}
                                \boxtoright\valbox{$\sim$}
                          \unsavBboxpos\wideboxtoright{24}
                          \valbox{4}\boxtoright
          \widevalbox{000003E5}{24}\wideboxtoright{24}
                          \savBboxpos\fallarrowright{110}{83}
                                \putboxafterarrow\valbox{(}
                                \boxtoright\valbox{$\sim$}
                                \boxtoright\valbox{V}
                                \boxtoright\valbox{[}
                                \boxtoright\valbox{V}
                                \boxtoright\valbox{]}
                                \boxtoright\valbox{)}
                          \unsavBboxpos\valbox{7}\boxtoright
          \widevalbox{00000355}{24}\wideboxtoright{24}
                          \fallarrowright{30}{18}
                          \putboxafterarrow\widevalbox{$<$12$>$}{14}
                                \fallarrow{20}\messageunderarrow{FAILED~~~~~~}
                          \wideboxtoright{14}\widevalbox{$<$16$>$}{14}
                                \fallarrow{20}\messageunderarrow{~~IX}
                          \wideboxtoright{14}\valbox{2}
                          \boxtoright\valbox{13}
                          \boxtoright\valbox{F}
                          \boxtoright\widevalbox{bool}{15}
%     \tagbox{000001BD:}{000003E1}{72}{19}\wideboxtoright{19}\valbox{4}
%     \setboxsizes{15}{72}
%     \putboxafterarrow
%        \valbox{V}\boxtoright\valbox{i$\rightarrow$r}\boxtoright\valbox{V}\boxtoright
%        \valbox{V}\boxtoright\valbox{V}\boxtoright\valbox{*}\boxtoright
%        \valbox{$-$}\boxtoright\valbox{I$\rightarrow$R}\boxtoright\valbox{V}\boxtoright
%        \valbox{V}\boxtoright\valbox{R**R}\boxtoright\valbox{/}\boxtoright
%        \valbox{$+$}
%    \setboxsizes{9}{72}
%\nextfbox\leftwidetagbox{infix}{}{72}{19}\outarrowdownright{31}{10}{13}
%    \setboxsizes{15}{72}
%    \putboxafterarrow
%        \valbox{V}\boxtoright\valbox{$+$}\boxtoright\valbox{(}\boxtoright
%        \valbox{V}\boxtoright\valbox{$-$}\boxtoright\valbox{V}\boxtoright
%        \valbox{*}\boxtoright\valbox{V}\boxtoright\valbox{)}\boxtoright
%        \valbox{/}\boxtoright\valbox{V}\boxtoright\valbox{**}\boxtoright
%        \valbox{V}
%    \setboxsizes{9}{72}
%\nextFbox\leftwidetagbox{operands}{}{72}{19}
%    \outarrowdownright{31}{20}{80}
%    \stackDboxtoright
%    \nextDbox\valbox{}\outarrowmsg{30}{NELE}
%    \nextDbox\valbox{}\outarrowmsg{30}{\#4}
%    \nextDbox\valbox{}\outarrowmsg{30}{\#2}
%    \nextDbox\valbox{}\outarrowmsg{30}{NIX}
%    \nextDbox\valbox{}\outarrowmsg{30}{\#1.0}
%    \nextDbox\valbox{}\outarrowmsg{30}{MU}
\nextGbox\widenobox{...}{1}
\end{fast_picture}

The
following is a {\bf raw} map of memory after the above example
has been parsed:

\begin{center}\begin{tabular}{|r|r|c|}
\hline
Real constants: & 00000000: & 3.000000000000000e$-$04 \\
                & 00000004: & 3.000000000000000e$-$04 \\
                & 00000010: & 1.000000000000000e$+$10 \\
                & 00000014: & 1.000000000000000e$+$10 \\
\hline
Integer constants: & 00000020: &  5 \\
                   & 00000024: &  5 \\
                   & 00000028: &  0 \\
                   & 0000002C: &  1 \\
                   & 00000030: &  2 \\
                   & 00000034: &  3 \\
                   & 00000038: &  4 \\
                   & 0000003C: &  0 \\
\hline
Boolean constants: & 00000040: & FALSE \\
                   & 00000041: & TRUE \\
                   & 00000042: & FALSE \\
                   & 00000043: & FALSE \\
                   & 00000044: & FALSE \\
\hline
State offset constants: & 00000045: & (0..5,3@0) \\
                        & 0000004D: & (0..5,3@3) \\
                        & 00000055: & (0..1,1@6) \\
\hline
\end{tabular}\end{center}
\begin{center}\begin{tabular}{|r|c|c|c|c|c|c|}
\hline
\multicolumn{7}{|c|}{Expressions:} \\
\hline
address & postfix & infix & operand & line & err & returns \\
\hline\hline
0000005D: & (000003B5,1) & (000003B6,1) & (000002F1,1) & 9 & F & $<$integer$>$ \\
00000073: & (000003B7,1) & (000003B8,1) & (000002F5,1) & 9 & F & $<$integer$>$ \\
00000089: & (000003B9,1) & (000003BA,1) & (000002F9,1) & 9 & F & $<$integer$>$ \\
0000009F: & (000003BB,1) & (000003BC,1) & (000002FD,1) & 9 & F & $<$bool$>$ \\
000000B5: & (000003BD,1) & (000003BE,1) & (00000301,1) & 9 & F & $<$integer$>$ \\
000000CB: & (000003BF,1) & (000003C0,1) & (00000305,1) & 9 & F & $<$bool$>$ \\
000000E1: & (000003C1,1) & (000003C2,1) & (00000309,1) & 9 & F & $<$integer$>$ \\
000000F7: & (000003C3,1) & (000003C4,1) & (0000030D,1) & 9 & F & $<$bool$>$ \\
0000010D: & (000003C5,1) & (000003C6,1) & (00000311,1) & 9 & F & $<$integer$>$ \\
00000123: & (000003C7,1) & (000003C8,1) & (00000315,1) & 9 & F & $<$bool$>$ \\
00000139: & (000003C9,1) & (000003CA,1) & (00000319,1) & 9 & F & $<$integer$>$ \\
0000014F: & (000003CB,1) & (000003CC,1) & (0000031D,1) & 9 & F & $<$bool$>$ \\
00000165: & (000003CD,3) & (000003D0,5) & (0000033D,2) & 10 & F & $<$bool$>$$<$expr$-$var$>$ \\
0000017B: & (000003D5,1) & (000003D6,1) & (00000345,1) & 11 & F & $<$integer$>$ \\
00000191: & (000003D7,1) & (000003D8,1) & (00000349,1) & 11 & F & $<$integer$>$ \\
000001A7: & (000003D9,3) & (000003DC,5) & (0000034D,2) & 12 & F & $<$bool$>$,$<$expr$-$var$>$ \\
000001BD: & (000003E1,4) & (000003E5,7) & (00000355,2) & 13 & F & $<$bool$>$,$<$expr$-$var$>$ \\
000001D3: & (000003EC,2) & (000003EE,2) & (0000035D,1) & 13 & F & $<$null$>$ \\
000001E9: & (000003F0,2) & (000003F2,2) & (00000361,1) & 13 & F & $<$null$>$ \\
000001FF: & (000003F4,1) & (000003F5,1) & (00000365,1) & 13 & F & $<$null$>$ \\
00000215: & (000003F6,1) & (000003F7,1) & (00000369,1) & 13 & F & $<$bool$>$ \\
0000022B: & (000003F8,1) & (000003F9,1) & (00000379,1) & 13 & F & $<$real$>$ \\
00000241: & (000003FA,3) & (000003FD,5) & (0000037D,2) & 15 & F & $<$bool$>$,$<$expr$-$var$>$ \\
00000257: & (00000402,3) & (00000405,6) & (00000385,2) & 16 & F & $<$bool$>$,$<$expr$-$var$>$ \\
0000026D: & (0000040B,2) & (0000040D,2) & (0000038D,1) & 16 & F & $<$null$>$ \\
00000283: & (0000040F,2) & (00000411,2) & (00000391,1) & 16 & F & $<$null$>$ \\
00000299: & (00000413,1) & (00000414,1) & (00000395,1) & 16 & F & $<$null$>$ \\
000002AF: & (00000415,1) & (00000416,1) & (00000399,1) & 16 & F & $<$bool$>$ \\
000002C5: & (00000417,1) & (00000418,1) & (000003A9,1) & 16 & F & $<$real$>$ \\
000002DB: & (00000419,3) & (0000041C,5) & (000003AD,2) & 19 & F & $<$bool$>$,$<$expr$-$var$>$ \\
\hline
\end{tabular}\end{center}
\begin{center}\begin{tabular}{|r|c||r|c||r|c|}
\hline
\multicolumn{6}{|c|}{Expression operands:} \\
\hline
address & id & address & id & address & id \\
\hline\hline
000002F1: & $<$7$>$  & 00000335: & $<$12$>$ & 00000379: & $<$3$>$ \\
000002F5: & $<$8$>$  & 00000339: & $<$12$>$ & 0000037D: & $<$11$>$ \\
000002F9: & $<$9$>$  & 0000033D: & $<$11$>$ & 00000381: & $<$8$>$ \\
000002FD: & $<$0$>$  & 00000341: & $<$10$>$ & 00000385: & $<$12$>$ \\
00000301: & $<$13$>$ & 00000345: & $<$9$>$  & 00000389: & $<$16$>$ \\
00000305: & $<$0$>$  & 00000349: & $<$7$>$  & 0000038D: & $<$10$>$ \\
00000309: & $<$14$>$ & 0000034D: & $<$10$>$ & 00000391: & $<$11$>$ \\
0000030D: & $<$0$>$  & 00000351: & $<$8$>$  & 00000395: & $<$16$>$ \\
00000311: & $<$15$>$ & 00000355: & $<$12$>$ & 00000399: & $<$0$>$ \\
00000315: & $<$0$>$  & 00000359: & $<$16$>$ & 0000039D: & $<$10$>$ \\
00000319: & $<$6$>$  & 0000035D: & $<$10$>$ & 000003A1: & $<$11$>$ \\
0000031D: & $<$0$>$  & 00000361: & $<$11$>$ & 000003A5: & $<$12$>$ \\
00000321: & $<$10$>$ & 00000365: & $<$16$>$ & 000003A9: & $<$5$>$ \\
00000325: & $<$11$>$ & 00000369: & $<$1$>$  & 000003AD: & $<$11$>$ \\
00000329: & $<$12$>$ & 0000036D: & $<$10$>$ & 000003B1: & $<$10$>$ \\
0000032D: & $<$12$>$ & 00000371: & $<$11$>$ & & \\
00000331: & $<$12$>$ & 00000375: & $<$12$>$ & & \\
\hline
\end{tabular}\end{center}
\begin{center}\begin{tabular}{|r|c|c|c|c|c|c|c|c|}
\hline
\multicolumn{9}{|c|}{Expression operations:} \\
\hline
address & \multicolumn{8}{c|}{operations} \\
\hline\hline
000003B5$-$000003BC: & V    & V      & V    & V    & V    & V    & V    & V  \\
000003BD$-$000003C4: & V    & V      & V    & V    & V    & V    & V    & V  \\
000003C5$-$000003CC: & V    & V      & V    & V    & V    & V    & V    & V  \\
000003CD$-$000003D4: & V    & V      & $>=$ & (    & V    & $>=$ & V    & )  \\
000003D5$-$000003DC: & V    & V      & V    & V    & V    & V    & $>$  & (  \\
000003DD$-$000003E4: & V  & $>$  & V   & )    & V    & V    & []   & $\sim$ \\
000003E5$-$000003EC: & (    & $\sim$ & V    & [    & V    & ]    & )    & V  \\
000003ED$-$000003F4: & $--$ & V      & $--$ & V    & $++$ & V    & $++$ & V  \\
000003F5$-$000003FC: & V    & V   & V    & V    & V    & V    & V    & $>$  \\
000003FD$-$00000404: & (    & V    & $>$  & V    & )    & V    & V    & []  \\
00000405$-$0000040C: & (    & V    & [    & V    & ]    & )    & V    & $++$ \\
0000040D$-$00000414: & V    & $++$   & V    & $--$ & V    & $--$ & V    & V  \\
00000415$-$0000041C: & V    & V      & V    & V    & V    & V    & $>=$ & (  \\
0000041D$-$00000420: & V    & $>=$   & V    & )    &      &      &      &    \\
\hline
\end{tabular}\end{center}
\begin{center}\begin{tabular}{|r|c|}
\hline
\multicolumn{2}{|c|}{Space variable information:} \\
\hline
00000421: & $<$10$>$ \\
00000425: & $<$11$>$ \\
00000429: & $<$12$>$ \\
\hline
\end{tabular}\end{center}
\begin{center}\begin{tabular}{|r|c|}
\hline
\multicolumn{2}{|c|}{FOR set range expression pointers:} \\
\hline
0000042D: & (0000017B .. 00000191) \\
\hline
\end{tabular}\end{center}
\begin{center}\begin{tabular}{|r|c|}
\hline
\multicolumn{2}{|c|}{state-space PICTURE data:} \\
\hline
00000435: & (NWP:0..5,3@0,NFP:0..5,3@3,FAILED[1..5]:0..1,1@6) \\
\hline
\end{tabular}\end{center}
\begin{center}\begin{tabular}{|r|c|}
\hline
\multicolumn{2}{|c|}{ASSERT/DEATHIF/PRUNEIF boolean tests:} \\
\hline
 address & expression/source line number \\
\hline\hline
0000043F: & (expr=00000165,line\#10) \\
00000445: & (expr=000002DB,line\#19) \\
\hline
\end{tabular}\end{center}
\begin{center}\begin{tabular}{|r|cccc|}
\hline
\multicolumn{5}{|c|}{TRANTO clause data structures:} \\
\hline\hline
address & (TRANTO (vars,exprs,\#vars) & BY & \#exprs~@~expr & (line\#)) \\
\hline
0000044B: & (TRANTO (00000321,0000005D,7) & BY & n/a & (line 9)) \\
0000045D: & (TRANTO (0000036D,000001D3,3) & BY & 1~@~0000022B & (line 13)) \\
0000046F: & (TRANTO (0000039D,0000026D,3) & BY FAST & 1~@~000002C5 & (line 16)) \\
\hline
\end{tabular}\end{center}
\begin{center}\begin{tabular}{|r|c|}
\hline
\multicolumn{2}{|c|}{block and tranto IF data structures:} \\
\hline\hline
address & IF ... THEN ... [ELSE ...] \\
\hline
00000481: & (IF 000001BD THEN GOSUB 0000052B) \\
0000048D: & (IF 000001A7 THEN GOSUB 0000053D) \\
00000499: & (IF 00000257 THEN GOSUB 0000054F) \\
000004A5: & (IF 00000241 THEN GOSUB 00000561) \\
\hline
\end{tabular}\end{center}
\begin{center}\begin{tabular}{|r|c|}
\hline
\multicolumn{2}{|c|}{FOR range data:} \\
\hline
000004B1: &  ($<$16$>$ IN [1~@~0000042D] GOSUB 00000573) \\
\hline
\end{tabular}\end{center}
\begin{center}\begin{tabular}{|r|c|}
\hline
\multicolumn{2}{|c|}{model generation code, PREAMBLE section:} \\
\hline
000004BF: & BEGIN 000004FE ! ASSERT section \\
000004C8: & BEGIN 00000507 ! DEATHIF section \\
000004D1: & BEGIN 00000522 ! PRUNEIF section \\
000004DA: & BEGIN 0000058E ! TRANTO section \\
000004E3: & SPACE 00000435 \\
000004EC: & START 0000044B \\
000004F5: & END \\
\hline
\end{tabular}\end{center}
\begin{center}\begin{tabular}{|r|c|}
\hline
\multicolumn{2}{|c|}{model generation code, ASSERT section:} \\
\hline
000004FE: & RETURN \\
\hline
\end{tabular}\end{center}
\begin{center}\begin{tabular}{|r|c|}
\hline
\multicolumn{2}{|c|}{model generation code, DEATHIF section:} \\
\hline
00000507: & DEATHIF 0000043F \\
00000510: & DEATHIF 00000445 \\
00000519: & RETURN \\
\hline
\end{tabular}\end{center}
\begin{center}\begin{tabular}{|r|c|}
\hline
\multicolumn{2}{|c|}{model generation code, PRUNEIF section:} \\
\hline
00000522: & RETURN \\
\hline
\end{tabular}\end{center}
\begin{center}\begin{tabular}{|r|c|}
\hline
\multicolumn{2}{|c|}{model generation code, TRANTO section:} \\
\hline
0000052B: & TRANTO 0000045D \\
00000534: & RETURN \\
\hline
0000053D: & IF 00000481 \\
00000546: & RETURN \\
\hline
0000054F: & TRANTO 0000046F \\
00000558: & RETURN \\
\hline
00000561: & IF 00000499 \\
0000056A: & RETURN \\
\hline
00000573: & IF 0000048D \\
0000057C: & IF 000004A5 \\
00000585: & RETURN \\
\hline
0000058E: & LOOP 000004B1 \\
00000597: & RETURN \\
\hline
\end{tabular}\end{center}
\begin{center}\begin{tabular}{|c|c|c|}
\hline
\multicolumn{3}{|c|}{Identifier table:} \\
\hline
   $<$0$>$ & = & (00000040,SCALAR,0,"FALSE",0x02,[$<$boolean$>$]) \\
   $<$1$>$ & = & (00000041,SCALAR,0,"TRUE",0x02,[$<$boolean$>$]) \\
   $<$2$>$ & = & (00000000,SCALAR,0,"\#3.0E$-$4",0x04,[$<$real$>$]) \\
   $<$3$>$ & = & (00000008,SCALAR,0,"LAMBDA",0x04,[$<$real$>$]) \\
   $<$4$>$ & = & (00000010,SCALAR,0,"\#1.0E10",0x04,[$<$real$>$]) \\
   $<$5$>$ & = & (00000018,SCALAR,0,"DELTA",0x04,[$<$real$>$]) \\
   $<$6$>$ & = & (00000020,SCALAR,0,"\#5",0x03,[$<$integer$>$]) \\
   $<$7$>$ & = & (00000024,SCALAR,0,"NP",0x03,[$<$integer$>$]) \\
   $<$8$>$ & = & (00000028,SCALAR,0,"\#0",0x03,[$<$integer$>$]) \\
   $<$9$>$ & = & (0000002C,SCALAR,0,"\#1",0x03,[$<$integer$>$]) \\
   $<$10$>$ & = & (00000045,SCALAR,0,"NWP",0x23,[$<$integer$>$,$<$ss$-$var$>$]) \\
   $<$11$>$ & = & (0000004D,SCALAR,0,"NFP",0x23,[$<$integer$>$,$<$ss$-$var$>$]) \\
   $<$12$>$ & = & (00000055,ARRAY[1..5],0,"FAILED",0xa2,[$<$boolean$>$,$<$ss$-$var$>$,$<$array$>$]) \\
   $<$13$>$ & = & (00000030,SCALAR,0,"\#2",0x03,[$<$integer$>$]) \\
   $<$14$>$ & = & (00000034,SCALAR,0,"\#3",0x03,[$<$integer$>$]) \\
   $<$15$>$ & = & (00000038,SCALAR,0,"\#4",0x03,[$<$integer$>$]) \\
   $<$16$>$ & = & (0000003C,SCALAR,$-$1,"IX",0x03,[$<$integer$>$]) \\
\hline
\end{tabular}\end{center}

This example is interesting because, although it is fairly straightforward,
it has IF's nested inside of a FOR as well as two IF's in sequence.  The
following is a map of memory after the above example has been parsed:


\section{Hashing of state space}
\label{chap:hash}

The \acronym{ASSIST} rule generation algorithm uses a hashing algorithm
to hash from the a given state node n-tuple to the model file state number.
Before commencement of rule generation, memory is allocated for the hash
table and the state storage array.

The hash table is divided into two sections called the main table
and the extension table.   The height of the main table is always static
because it is the size of the hash table.
Both tables are made up of buckets.   Each bucket is \rw{bucket\_width} wide
as defined with the \option{bw} option.   By default, \option{bw=5}.
In case more than \rw{bucket\_width} states map to the same bucket, there
is a link at the end of the bucket which points to the next bucket in the
linked list.   The extension table is initially a free bucket pool.   When
a new bucket is needed to extend the main table, the new bucket is always
taken from the free pool.   When the free pool becomes empty and a new
bucket is required, an attempt is make to re-allocate a larger extension
table.   On MS-DOS systems, all available memory is allocated for the extension
bucket.   On all systems, an error message is printed out when there is no
available memory left for the re-allocation.

The state storage array holds the bit-encoded state-space nodes.   The index
into this array is the state number less some constant.   There are some
extra special states which are stored at the front of the table and the
death and prune states are omitted unless \rw{ONEDEATH} is \rw{OFF} in
which case only the prune states are omitted and the included death states
have an extra death-state flag bit set.

The hash table is packed into character arrays:

\begin{codeexample}
static unsigned char *state_storage;
static unsigned char *bucket_storage;
static unsigned char *bucket_extension_storage;
static unsigned char *bucket_extension_ovf;
static unsigned char *next_free_extension_bucket;
\end{codeexample}


The ``state\_storage'' array holds the bit-packed state nodes for each of the
states in the model.   The index into this array is computed as follows:
\[ i = x + h - s \]
where $i$ is the index into the array, $x$ is the state number output to the
model file, $h$ is the number of special state node entries in header, and
$s$ is the state number of the start state.   To get the byte index:
\[ i_{b} = i \times w \]
where $w$ is the number of bytes necessary to pack a state node.
See section \ref{sec:bitpack} on page \pageref{sec:bitpack} for detailed
information on how state nodes are packed.

The ``bucket\_storage'' array is used to store the main hash table and the
``bucket\_extension\_storage'' array is used to store the extension links
for the entries in the main hash table which have more than ``bucket\_width''
collisions.

In order to conserve memory, the values stored in the buckets are packed
as three-byte integers.   A three byte integer can store numbers in the
range -8,388,607 through 8,388,607.   For the purpose of illustration,
the fictitious ``C'' language type ``medium'' will be used to denote a
three byte integer:

\begin{codeexample}
typedef unsigned char medium[3];  /* three byte integer */
\end{codeexample}


The main hash table array is sub-divided as if there was a type as follows:

\begin{codeexample}
typedef struct t__main_table_bucket
   \{
       medium count;                 /* count of entries */
       medium nextlink;              /* link to next bucket */
       medium entry[bucket_width];   /* each state number */
   \} main_table_bucket_type;
\end{codeexample}


Note that the ``count'' is the cumulative total of all entries in all buckets.
If, for example, the count is 11 and the width is 5, then there are three
buckets in the chain.   The first two buckets, having 5 entries, will be full.
The third bucket will have the remaining entry.   The first bucket will be
in the main table and the remaining buckets will be found in the extension
table.   The extension hash table array is sub-divided as if there was a type
as follows:

\begin{codeexample}
typedef struct t__extension_table_bucket
   \{
       medium nextlink;              /* link to next bucket */
       medium entry[bucket_width];   /* each state number */
   \} extension_table_bucket_type;
\end{codeexample}


Consider a system with a combined total of 6 \rw{DEATHIF} and \rw{PRUNEIF}
statements.   Then the start state will be state 7.
Suppose that states 67 and 103 hash to bucket \#1, no states hash to
bucket \#2, the start state number 7 hashes to bucket \#3, six states
numbers 73, 82, 91, 101, 104, and 122 hash to bucket \#$n-1$, and that
state number 197 hashes to the last bucket.
The diagram in figure \ref{fig:hashtable}
illustrates, for a bucket width of 5, how these collisions
are hashed into the buckets.

Note that the link is drawn to the right side of
each bucket even though it is physically located as the second ``medium''
in the main table bucket and the first ``medium'' in the extension table
bucket.   This makes the arrows easier to draw and makes the picture less
cluttered.
\startfig
\begin{fast_picture}{500}{300}
\setboxpos{100}{270}\savFboxpos\setarrow{100}{270}
\stackwideAbox{30}
\nextAbox\savBboxpos\leftwidetagbox{bucket \#1}{$2$}{72}{20}
         \unsavBboxpos\wideboxtoright{20}\savBboxpos\widevalbox{$67$}{20}
         \unsavBboxpos\wideboxtoright{20}\savBboxpos\widevalbox{$103$}{20}
         \unsavBboxpos\wideboxtoright{20}\savBboxpos\widevalbox{n/a}{20}
         \unsavBboxpos\wideboxtoright{20}\savBboxpos\widevalbox{n/a}{20}
         \unsavBboxpos\wideboxtoright{20}\savBboxpos\widevalbox{n/a}{20}
         \unsavBboxpos\wideboxtoright{20}\savBboxpos\widevalbox{}{20}
         \outarrowdown{120}{120}\groundunderarrow
\nextAbox\savBboxpos\leftwidetagbox{bucket \#2}{$0$}{72}{20}
         \unsavBboxpos\wideboxtoright{20}\savBboxpos\widevalbox{n/a}{20}
         \unsavBboxpos\wideboxtoright{20}\savBboxpos\widevalbox{n/a}{20}
         \unsavBboxpos\wideboxtoright{20}\savBboxpos\widevalbox{n/a}{20}
         \unsavBboxpos\wideboxtoright{20}\savBboxpos\widevalbox{n/a}{20}
         \unsavBboxpos\wideboxtoright{20}\savBboxpos\widevalbox{n/a}{20}
         \unsavBboxpos\wideboxtoright{20}\savBboxpos\widevalbox{}{20}
         \outarrowdown{120}{100}
\nextAbox\savBboxpos\leftwidetagbox{bucket \#3}{$1$}{72}{20}
         \unsavBboxpos\wideboxtoright{20}\savBboxpos\widevalbox{$7$}{20}
         \unsavBboxpos\wideboxtoright{20}\savBboxpos\widevalbox{n/a}{20}
         \unsavBboxpos\wideboxtoright{20}\savBboxpos\widevalbox{n/a}{20}
         \unsavBboxpos\wideboxtoright{20}\savBboxpos\widevalbox{n/a}{20}
         \unsavBboxpos\wideboxtoright{20}\savBboxpos\widevalbox{n/a}{20}
         \unsavBboxpos\wideboxtoright{20}\savBboxpos\widevalbox{}{20}
         \outarrowdown{120}{80}
\nextAbox\savBboxpos\leftwidetagbox{}{$\cdots$}{72}{20}
         \unsavBboxpos\wideboxtoright{20}\savBboxpos\widevalbox{$\cdots$}{20}
         \unsavBboxpos\wideboxtoright{20}\savBboxpos\widevalbox{$\cdots$}{20}
         \unsavBboxpos\wideboxtoright{20}\savBboxpos\widevalbox{$\cdots$}{20}
         \unsavBboxpos\wideboxtoright{20}\savBboxpos\widevalbox{$\cdots$}{20}
         \unsavBboxpos\wideboxtoright{20}\savBboxpos\widevalbox{$\cdots$}{20}
         \unsavBboxpos\wideboxtoright{20}\savBboxpos\widevalbox{$\cdots$}{20}
\nextAbox\savBboxpos\leftwidetagbox{bucket \#$n-1$}{$6$}{72}{20}
         \unsavBboxpos\wideboxtoright{20}\savBboxpos\widevalbox{$73$}{20}
         \unsavBboxpos\wideboxtoright{20}\savBboxpos\widevalbox{$82$}{20}
         \unsavBboxpos\wideboxtoright{20}\savBboxpos\widevalbox{$91$}{20}
         \unsavBboxpos\wideboxtoright{20}\savBboxpos\widevalbox{$101$}{20}
         \unsavBboxpos\wideboxtoright{20}\savBboxpos\widevalbox{$104$}{20}
         \unsavBboxpos\wideboxtoright{20}\savBboxpos\widevalbox{}{20}
         \outarrowdownleftdownright{60}{50}{340}{30}{20}
\nextAbox\savBboxpos\leftwidetagbox{bucket \#$n$}{$1$}{72}{20}
         \unsavBboxpos\wideboxtoright{20}\savBboxpos\widevalbox{$197$}{20}
         \unsavBboxpos\wideboxtoright{20}\savBboxpos\widevalbox{n/a}{20}
         \unsavBboxpos\wideboxtoright{20}\savBboxpos\widevalbox{n/a}{20}
         \unsavBboxpos\wideboxtoright{20}\savBboxpos\widevalbox{n/a}{20}
         \unsavBboxpos\wideboxtoright{20}\savBboxpos\widevalbox{n/a}{20}
         \unsavBboxpos\wideboxtoright{20}\savBboxpos\widevalbox{}{20}
         \outarrowdown{120}{20}
\nextAbox\nextAbox
\nextAbox\savBboxpos\leftwidetagbox{extension \#1}{$122$}{72}{20}
         \unsavBboxpos\wideboxtoright{20}\savBboxpos\widevalbox{n/a}{20}
         \unsavBboxpos\wideboxtoright{20}\savBboxpos\widevalbox{n/a}{20}
         \unsavBboxpos\wideboxtoright{20}\savBboxpos\widevalbox{n/a}{20}
         \unsavBboxpos\wideboxtoright{20}\savBboxpos\widevalbox{n/a}{20}
         \unsavBboxpos\wideboxtoright{20}\savBboxpos\widevalbox{}{20}
         \outarrowdown{120}{100}\groundunderarrow
\nextAbox\savBboxpos\leftwidetagbox{extension \#2}{n/a}{72}{20}
         \unsavBboxpos\wideboxtoright{20}\savBboxpos\widevalbox{n/a}{20}
         \unsavBboxpos\wideboxtoright{20}\savBboxpos\widevalbox{n/a}{20}
         \unsavBboxpos\wideboxtoright{20}\savBboxpos\widevalbox{n/a}{20}
         \unsavBboxpos\wideboxtoright{20}\savBboxpos\widevalbox{n/a}{20}
         \unsavBboxpos\wideboxtoright{20}\savBboxpos\widevalbox{}{20}
         \outarrowdown{120}{80}
\nextAbox\savBboxpos\leftwidetagbox{extension \#3}{n/a}{72}{20}
         \unsavBboxpos\wideboxtoright{20}\savBboxpos\widevalbox{n/a}{20}
         \unsavBboxpos\wideboxtoright{20}\savBboxpos\widevalbox{n/a}{20}
         \unsavBboxpos\wideboxtoright{20}\savBboxpos\widevalbox{n/a}{20}
         \unsavBboxpos\wideboxtoright{20}\savBboxpos\widevalbox{n/a}{20}
         \unsavBboxpos\wideboxtoright{20}\savBboxpos\widevalbox{}{20}
         \outarrowdown{120}{60}
\nextAbox\savBboxpos\leftwidetagbox{}{$\cdots$}{72}{20}
         \unsavBboxpos\wideboxtoright{20}\savBboxpos\widevalbox{$\cdots$}{20}
         \unsavBboxpos\wideboxtoright{20}\savBboxpos\widevalbox{$\cdots$}{20}
         \unsavBboxpos\wideboxtoright{20}\savBboxpos\widevalbox{$\cdots$}{20}
         \unsavBboxpos\wideboxtoright{20}\savBboxpos\widevalbox{$\cdots$}{20}
         \unsavBboxpos\wideboxtoright{20}\savBboxpos\widevalbox{$\cdots$}{20}
\nextAbox\savBboxpos\leftwidetagbox{extension \#$n_{x}$}{n/a}{72}{20}
         \unsavBboxpos\wideboxtoright{20}\savBboxpos\widevalbox{n/a}{20}
         \unsavBboxpos\wideboxtoright{20}\savBboxpos\widevalbox{n/a}{20}
         \unsavBboxpos\wideboxtoright{20}\savBboxpos\widevalbox{n/a}{20}
         \unsavBboxpos\wideboxtoright{20}\savBboxpos\widevalbox{n/a}{20}
         \unsavBboxpos\wideboxtoright{20}\savBboxpos\widevalbox{}{20}
         \outarrowdown{120}{20}
\end{fast_picture}
\finishfig{Sample hash table laid out in memory}{fig:hashtable}

\begin{thebibliography}{25}

\bibitem{quick} Barkakati, Nabajyoti:
The Waite Group's QuickC Bible,
1989,
Howard W. Sams \& Company,
A division of Macmillan, Inc.
4300 West 62nd Street,
Indianapolis, Indiana 46268


\bibitem{ViewCARE} Bavuso, Salvatore J.:
A User's View of CARE III.  {\em 1984 Annual Reliability
and Maintainability Symposium}, January 1984.

\bibitem{CARE}  Bavuso, S. J.; and Petersen, P. L.:  {\em CARE III Model
Overview and User's Guide (First Revision)}.  NASA TM-86404, 1985.

\bibitem{Bavuso87} Bavuso, Salvatore J., et al:
Dependability Analysis of Typical Fault-Tolerant Architectures
Using HARP, NASA TP 2760, November 1987.

\bibitem{bavuso87IEEE} Bavuso, S. J., et. al.: Analysis of Typical
Fault-Tolerant Architectures Using HARP, {\em IEEE Transactions on
Reliability}, Vol. R-36, No. 2., June 1987.


\bibitem{butler86} Butler, Ricky W.:
The SURE Reliability Analysis Program, NASA TM-87593, February 1986.

\bibitem{butler85} Butler, Ricky W.: An Abstract Specification Language
for Markov Reliability Models.  NASA TM-86423, April 1985.

\bibitem{SURE} Butler, Ricky W.; and White, Allan L.: {\em SURE Reliability
Analysis: Program and Mathematics}, NASA TP-2764, March 1988.

\bibitem{PAWS} Butler, Ricky W; and Stevenson, Philip H.: {\em The PAWS and
STEM Reliability Analysis Programs}. NASA TM-100572, March 1988.

\bibitem{ART} Butler, Ricky W; and Johnson, Sally C.: {\em The Art of
Fault-Tolerant System Reliability Modeling}.  NASA TM-102623, March 1990.


\bibitem{HARP} Dugan, J. B.; Trivedi, K. S.; Smotherman, M. K.; and Geist,
R. M.: The Hybrid Automated Reliability Predictor, {\em Journal of
Guidance, Control, and Dynamics}, Vol. 9, No. 3, May-June 1986, pp. 319-331.


\bibitem{Finelli87} Finelli, Geroge B.:
Characterization of Fault Recovery through Fault Injection on FTMP.
{\em IEEE Transactions on Reliability}, Vol. R-36, No. 2, June 1987.


\bibitem{SIFT} Goldberg, Jack, et. al.: {\em Development and Analysis of the
Software Implemented Fault-Tolerance (SIFT) Computer}. NASA CR-172146, 1984.


\bibitem{Harper87} Harper, Richard E.:
Critical Issues in Ultra-Reliable Parallel Processing,
{\em Ph. D. Thesis, Massachusetts Institute of Technology}, June 1987.


\bibitem{assistman} Johnson, Sally C.: {\em ASSIST User's Manual}. NASA
TM-87735, August 1986.

\bibitem{assistpr} Johnson, Sally C.: Reliability Analysis of Large, Complex
Systems using ASSIST, {\em AIAA/IEEE 8th Digital Avionics Systems Conference},
San Jose, California, October 1988.


\bibitem{assist7man} Johnson, Sally C.:
{\em ASSIST User's Manual, Release 7.0}. NASA TM Pending, August 1991.

\bibitem{Kernighan} Kerhighan, Brian and Ritchie, Dennis:
THE C PROGRAMMING LANGUAGE,
Second edition,
1988,
AT\&T Bell Laboratories,
Prentice Hall,
Englewood Cliffs, New Jersey 07632.

\bibitem{Microsoft} Lafore, Robert
The Waite Group's Microsoft C programming for the IBM,
19xx,
Howard W. Sams \& Company,
A division of Macmillan, Inc.
4300 West 62nd Street,
Indianapolis, Indiana 46268
\bibitem{Siewiorek} Siewiorek, Daniel P.; and Swarz, Robert S.:
{\em The Theory and Practice of Reliable System Design}, Digital Press, 1982.

\bibitem{Trivedi85} Trivedi, Kishor; et al: Hybrid Modeling of Fault-Tolerant
Systems.  {\em Computers and Electrical Engineering, An International Journal},
vol. 11, no. 2 and 3, pp. 87-108, 1985.

\bibitem{white84} White, Allan L.:
Upper and Lower Bounds for Semi-Markov Reliability
Models of Reconfigurable Systems.  NASA CR-172340, 1984.

\bibitem{white85} White, Allan L.: {\em Synthetic Bounds for Semi-Markov 
Reliability Models}. NASA CR-178008, 1985.

\bibitem{white90} White, Allan L., and Palumbo Daniel L.: State Reduction
for Semi-Markov Reliability Models, {\em The 36th Annual Reliability and 
Maintainability Symposium}, Los Angeles, CA, January 1990.

\end{thebibliography} 


\startappendix
\section{BNF Language Description}
\label{apix:bnf}
This appendix\index{syntax}
gives a complete description of the \acronym{ASSIST} language
syntax using the ``Backus-Naur Form'' grammar.

\begin{bnf_tabbing}
\> \grammar{program} \' ::= \>
   \grammar{setup-section}~~
   \grammar{start-section}~~
   \grammar{rule-section} \\
\\
\> \grammar{setup-section} \' ::= \>
   \grammar{setup-stat-seq}~~
   \grammar{SPACE-stat} \\
\\
\> \grammar{start-section} \' ::= \>
   \grammar{start-stat-seq}~~
   \grammar{START-stat}~~
   \grammar{start-stat-seq} \\
\\
\> \grammar{rule-section} \' ::= \>
   \grammar{rule-stat-seq} \\
\\
\\
\\
\\
\> \grammar{setup-stat-seq} \' ::= \>
   ${\sf \varepsilon}$ \\
\> ~~$|$ \> \grammar{any-setup-sec-stat}~~
   \grammar{setup-stat-seq} \\
\\
\> \grammar{start-stat-seq} \' ::= \>
   ${\sf \varepsilon}$ \\
\> ~~$|$ \> \grammar{any-start-sec-stat}~~
   \grammar{start-stat-seq} \\
\\
\> \grammar{rule-stat-seq} \' ::= \>
   \grammar{any-rule-sec-stat} \\
\> ~~$|$ \> \grammar{any-rule-sec-stat}~~
   \grammar{rule-stat-seq} \\
\\
\\
\\
\> \grammar{any-setup-sec-stat} \' ::= \>
   \grammar{global-stat}\label{bnf:setup-section} \\
\> ~~$|$ \> \grammar{pre-rule-global-stat} \\
\\
\> \grammar{any-start-sec-stat} \' ::= \>
   \grammar{global-stat}\label{bnf:start-section} \\
\> ~~$|$ \> \grammar{pre-rule-global-stat} \\
\> ~~$|$ \> \grammar{dep-variable-def} \\
\> ~~$|$ \> \grammar{function-def} \\
\> ~~$|$ \> \grammar{impl-function-def} \\
\\
\> \grammar{any-rule-sec-stat} \' ::= \>
   \grammar{global-stat}\label{bnf:rule-section} \\
\> ~~$|$ \> \grammar{ASSERT-stat} \\
\> ~~$|$ \> \grammar{DEATHIF-stat} \\
\> ~~$|$ \> \grammar{PRUNEIF-stat} \\
\> ~~$|$ \> \grammar{TRANTO-stat} \\
\> ~~$|$ \> \grammar{IF-stat} \\
\> ~~$|$ \> \grammar{FOR-stat} \\
\end{bnf_tabbing}\newpage\begin{bnf_tabbing}
\> \grammar{pre-rule-global-stat} \' ::= \>
   \grammar{quoted-SURE-stat}\label{bnf:pre-rule} \\
\> ~~$|$ \> \grammar{constant-def-stat} \\
\> ~~$|$ \> \grammar{option-def-stat} \\
\> ~~$|$ \> \grammar{INPUT-stat} \\
\\
\> \grammar{global-stat} \' ::= \>
   \grammar{debug-stat}\label{bnf:global} \\
\> ~~$|$ \> \grammar{command-option-stat}~{\sf $^{\sharp}$} \\
\> ~~$|$ \> \grammar{empty-stat} \\
\\
\\
\> \grammar{any-statement} \' ::= \>
   \grammar{any-setup-sec-stat} \\
\> ~~$|$ \> \grammar{any-start-sec-stat} \\
\> ~~$|$ \> \grammar{any-rule-sec-stat} \\
\> ~~$|$ \> \grammar{SPACE-stat} \\
\> ~~$|$ \> \grammar{START-stat} \\
\\
\\
\> \grammar{reserved-word} \' ::= \>
   \grammar{sensitive-keyword} \\
\> ~~$|$ \> \grammar{built-in-func-name} \\
\> ~~$|$ \> \grammar{pre-defined-constant} \\
\> ~~$|$ \> \grammar{descriptive-operator} \\
\> ~~$|$ \> \grammar{statement-name} \\
\\
\\
\> \grammar{sensitive-keyword} \' ::= \>
   {\bf BY} \\
\> ~~$|$ \> {\bf FAST} \\
\> ~~$|$ \> {\bf THEN} \\
\> ~~$|$ \> {\bf ELSE} \\
\> ~~$|$ \> {\bf ENDIF} \\
\> ~~$|$ \> {\bf ENDFOR} \\
\> ~~$|$ \> {\bf WITH} \\
\> ~~$|$ \> {\bf OF} \\
\> ~~$|$ \> {\bf IN} \\
\> ~~$|$ \> {\bf ARRAY} \\
\> ~~$|$ \> {\bf ON} \\
\> ~~$|$ \> {\bf OFF} \\
\> ~~$|$ \> {\bf FULL} \\
\> ~~$|$ \> {\bf BOOLEAN} \\
\\
\> \grammar{pre-defined-constant} \' ::= \>
   \grammar{option-def-name} \\
\> ~~$|$ \> {\bf AUTOFAST} \\
\> ~~$|$ \> {\bf TRIMOMEGA} \\
\> ~~$|$ \> {\bf TRUE} \\
\> ~~$|$ \> {\bf FALSE} \\
\end{bnf_tabbing}\newpage\begin{bnf_tabbing}
\> \grammar{descriptive-operator} \' ::= \>
   {\bf AND} \\
\> ~~$|$ \> {\bf OR} \\
\> ~~$|$ \> {\bf NOT} \\
\> ~~$|$ \> {\bf MOD} \\
\> ~~$|$ \> {\bf CYC} \\
\> ~~$|$ \> {\bf DIV} \\
\\
\> \grammar{statement-name} \' ::= \> 
   \grammar{option-def-name} \\
\> ~~$|$ \> {\bf C\_OPTION} \\
\> ~~$|$ \> {\bf DEBUG\$} \\
\> ~~$|$ \> {\bf INPUT} \\
\> ~~$|$ \> {\bf SPACE} \\
\> ~~$|$ \> {\bf FUNCTION} \\
\> ~~$|$ \> {\bf IMPLICIT} \\
\> ~~$|$ \> {\bf VARIABLE} \\
\> ~~$|$ \> {\bf START} \\
\> ~~$|$ \> {\bf ASSERT} \\
\> ~~$|$ \> {\bf DEATHIF} \\
\> ~~$|$ \> {\bf PRUNEIF} \\
\> ~~$|$ \> {\bf PRUNIF} \\
\> ~~$|$ \> {\bf TRANTO} \\
\> ~~$|$ \> {\bf IF} \\
\> ~~$|$ \> {\bf FOR} \\
\\
\> \grammar{option-def-name} \' ::= \> {\bf ONEDEATH} \\
\> ~~$|$ \> {\bf COMMENT} \\
\> ~~$|$ \> {\bf ECHO} \\
\> ~~$|$ \> {\bf TRIM} \\
\\
\\
\\
\\
\\
\> \grammar{constant-def-stat} \' ::= \>
   \grammar{named-constant}~~{\bf =}~~
   \grammar{const-var-def-clause}~~{\bf ;} \\
\> \grammar{const-var-def-clause} \' ::= \>
   \grammar{constant-def-clause} \\
\> ~~$|$ \> 
   {\bf BOOLEAN}~~\grammar{constant-def-clause} \\
\\
\> \grammar{constant-def-clause} \' ::= \>
   \grammar{expr}~~{\bf ;} \\
\> ~~$|$ \> 
   {\bf ARRAY~~(}~~\grammar{expr-list-with-of}~~{\bf )~~;} \\
\> ~~$|$ \> \grammar{single-sub-array}~~{\bf ;} \\
\> ~~$|$ \> \grammar{double-sub-array}~~{\bf ;} \\
\\
\> \grammar{double-sub-array} \' ::= \>
   {\bf [}~~\grammar{sub-array-list}~~{\bf ]} \\
\\
\> \grammar{sub-array-list} \' ::= \>
   \grammar{single-sub-array}~~{\bf ,}~~
   \grammar{single-sub-array} \\
\> ~~$|$ \> 
   \grammar{single-sub-array}~~{\bf ,}~~
   \grammar{sub-array-list} \\
\\
\> \grammar{single-sub-array} \' ::= \>
   {\bf [}~~\grammar{expr-list-with-of}~~{\bf ]} \\
\end{bnf_tabbing}\newpage\begin{bnf_tabbing}
\blockiftabbing
\> \grammar{option-def-stat} \' ::= \>
   {\bf ONEDEATH}~~\grammar{flag-status}~~{\bf ;} \\
\> ~~$|$ \> {\bf COMMENT}~~\grammar{flag-status}~~{\bf ;} \\
\> ~~$|$ \> {\bf ECHO}~~\grammar{flag-status}~~{\bf ;} \\
\> ~~$|$ \> {\bf TRIM}~~\grammar{flag-status}~~{\bf ;} \\
\> ~~$|$ \> {\bf TRIM}~~\grammar{flag-status}~~
   {\bf WITH}~~\grammar{expr}~~{\bf ;} \\
\\
\> \grammar{INPUT-stat} \' ::= \>
   {\bf INPUT}~~\grammar{input-list}~~{\bf ;} \\
\\
\> \grammar{SPACE-stat} \' ::= \>
   {\bf SPACE =}~~\grammar{space-picture}~~{\bf ;} \\
\\
\\
\\
\implicittabbing
\> \grammar{function-def} \' ::= \>
   {\bf FUNCTION}~~\grammar{function-name}~~
   {\bf (}~~\grammar{function-parm-list}~~{ \bf )}~~
   {\bf =}~~\grammar{expr}~~{\bf ;} \\
\\
\> \grammar{impl-function-def} \' ::= \>
   {\bf IMPLICIT}~~\grammar{impl-func-name}~~
   {\bf [}~~\grammar{impl-parm-list}~~{ \bf ]}~~
   {\bf =}~~\grammar{expr}~~{\bf ;} \\
\> ~~$|$ \>
   {\bf IMPLICIT}~~\grammar{impl-func-name} \\
\> \> \> {\bf [}~~\grammar{impl-parm-list}~~{ \bf ]}~~
   {\bf (}~~\grammar{index-parm-list}~~{ \bf )}~~
   {\bf =}~~\grammar{expr}~~{\bf ;} \\
\\
\> \grammar{dep-variable-def} \' ::= \>
   {\bf VARIABLE}~~\grammar{impl-func-name}~~
   {\bf [}~~\grammar{impl-parm-list}~~{ \bf ]}~~
   {\bf =}~~\grammar{const-var-def-clause}~~{\bf ;} \\
\\
\\
\\
\> \grammar{START-stat} \' ::= \>
   {\bf START =}~~\grammar{space-expression}~~
   {\bf ;} \\
\\
\\
\\
\> \grammar{ASSERT-stat} \' ::= \>
   {\bf ASSERT}~~\grammar{boolean-expression}~~{\bf ;} \\
\\
\> \grammar{DEATHIF-stat} \' ::= \>
   {\bf DEATHIF}~~\grammar{boolean-expression}~~{\bf ;} \\
\\
\> \grammar{PRUNEIF-stat} \' ::= \>
   {\bf PRUNEIF}~~\grammar{boolean-expression}~~{\bf ;} \\
\> ~~$|$ \> {\bf PRUNIF}~~\grammar{boolean-expression}~~
   {\bf ;} \\
\\
\> \grammar{TRANTO-stat} \' ::= \>
   {\bf IF} \> \grammar{boolean-expression}~~
   \grammar{TRANTO-clause}~~{\bf ;} \\
\> ~~$|$ \> \grammar{TRANTO-clause}~~{\bf ;} \\
\\
\> \grammar{IF-stat} \' ::= \>
   {\bf IF} \= \grammar{boolean-expression}~~{\bf THEN} \\
\> \> \> \grammar{rule-stat-seq} \\
\> \> {\bf ENDIF}~~{\bf ;} \\
\> ~~$|$ \>  {\bf IF} \> 
   \grammar{boolean-expression}~~{\bf THEN} \\
\> \> \> \grammar{rule-stat-seq} \\
\> \> {\bf ELSE} \\
\> \> \> \grammar{rule-stat-seq} \\
\> \> {\bf ENDIF}~~{\bf ;} \\
\\
\forlooptabbing
\> \grammar{FOR-stat} \' ::= \>
   {\bf FOR} \> \grammar{for-range} \\
\> \> \> \grammar{rule-stat-seq} \\
\> \> {\bf ENDFOR}~~{\bf ;} \\
\end{bnf_tabbing}\newpage\begin{bnf_tabbing}
\blockiftabbing
\> \grammar{quoted-SURE-stat} \' ::= \> 
   {\bf ''}~~\grammar{quot-text}~~{\bf ''} \\
\\
\\
\\
\> \grammar{command-option-stat} \' ::= \>
   {\bf C\_OPTION}~~\grammar{identifier}~~{\bf ;} \\
\> ~~$|$ \>
   {\bf C\_OPTION}~~\grammar{identifier}~~{\bf\boldmath $=$}~~
   \grammar{value}~~{\bf ;} \\
\\
\\
\\
\> \grammar{debug-stat} \' ::= \>
   {\bf DEBUG\$}~~{\bf ;} \\
\> ~~$|$ \>
   {\bf DEBUG\$}~~\grammar{identifier}~~{\bf ;} \\
\> \grammar{empty-stat} \' ::= \>
   {\bf ;} \\
\\
\\
\\
\> \grammar{TRANTO-clause} \' ::= \>
   {\bf TRANTO}~~\grammar{space-destination-list}~~
   {\bf BY}~~\grammar{rate-expression} \\
\> ~~$|$ \> 
   {\bf TRANTO}~~\grammar{space-expression}~~
   {\bf BY}~~\grammar{rate-expression} \\
\\
\\
\\
\> \grammar{flag-status} \' ::= \> ${\sf \varepsilon}$ \\
\> ~~$|$ \> {\bf OFF} \\
\> ~~$|$ \> {\bf ON} \\
\> ~~$|$ \> {\bf FULL} \\
\> ~~$|$ \> {\bf = 0} \\
\> ~~$|$ \> {\bf = 1} \\
\> ~~$|$ \> {\bf = 2} \\
\\
\\
\\
\> \grammar{input-list} \' ::= \>
   \grammar{input-item} \\
\> ~~$|$ \> \grammar{input-item}~~{\bf ,}~~
   \grammar{input-list} \\
\\
\> \grammar{input-item} \' ::= \>
   \grammar{named-constant} \\
\> ~~$|$ \>
   \grammar{prompt-message}~~{\bf :}~~
   \grammar{named-constant} \\
\> ~~$|$ \>
   {\bf BOOLEAN}~~
   \grammar{named-constant} \\
\> ~~$|$ \> 
   {\bf BOOLEAN}~~
   \grammar{prompt-message}~~{\bf :}~~
   \grammar{named-constant} \\
\\
\> \grammar{prompt-message} \' ::= \>
   {\bf ''}~~\grammar{quot-text}~~{\bf ''} \\
\\
\\
\\
\> \grammar{function-parm-list} \' ::= \>
   ${\sf \varepsilon}$ \\
\> ~~$|$ \>
   \grammar{identifier} \\
\> ~~$|$ \>
   \grammar{identifier}~~{\bf ,}~~
   \grammar{function-parm-list} \\
\\
\> \grammar{index-parm-list} \' ::= \>
   \grammar{identifier} \\
\> ~~$|$ \>
   \grammar{identifier}~~{\bf ,}~~
   \grammar{index-parm-list} \\
\end{bnf_tabbing}\newpage\begin{bnf_tabbing}
\> \grammar{impl-parm-list} \' ::= \>
   \grammar{state-space-var} \\
\> ~~$|$ \>
   \grammar{state-space-var}~~{\bf ,}~~
   \grammar{impl-parm-list} \\
\\
\\
\\
\> \grammar{quot-text} \' ::= \> ${\sf \varepsilon}$ \\
\> ~~$|$ \> \grammar{quot-text-char}
   ~~\grammar{quot-text} \\
\\
\> \grammar{quot-text-char} \' ::= \>
   \grammar{non-quote-ascii-char} \\
\\
\\
\\
\> \grammar{space-expression} \' ::= \>
   {\bf (}~~\grammar{space-expr-list}~~{\bf )} \\
\\
\> \grammar{space-expr-list} \' ::= \>
   \grammar{space-expr-item} \\
\> ~~$|$ \> \grammar{space-expr-item}~~{\bf ,}~~
   \grammar{space-expr-list} \\
\\
\> \grammar{space-expr-item} \' ::= \>
   \grammar{whole-or-boolean-expression} \\
\> ~~$|$ \> \grammar{whole-expression}~~{\bf OF}~~
   \grammar{whole-or-boolean-expression} \\
\> ~~$|$ \> \grammar{space-expression} \\
\\
\\
\\
\> \grammar{space-picture} \' ::= \>
   {\bf (}~~\grammar{space-item-list}~~{\bf )} \\
\\
\> \grammar{space-item-list} \' ::= \>
   \grammar{space-item} \\
\> ~~$|$ \> \grammar{space-item}~~{\bf ,}~~
   \grammar{space-item-list} \\
\\
\> \grammar{space-item} \' ::= \>
   \grammar{state-space-var} \\
\> ~~$|$ \> \grammar{state-space-var}~~{\bf :}~~
   \grammar{i-range} \\
\> ~~$|$ \> \grammar{state-space-var}~~{\bf :~~BOOLEAN} \\
\> ~~$|$ \> \grammar{space-picture} \\
\> ~~$|$ \> 
   \grammar{state-space-var}~~{\bf : ARRAY~~[}~~
   \grammar{array-range}~~{\bf ]} \\
\> ~~$|$ \> 
   \grammar{state-space-var}~~{\bf : ARRAY~~[}~~
   \grammar{array-range}~~{\bf ]~~OF}~~
   \grammar{i-range} \\
\> ~~$|$ \> 
   \grammar{state-space-var}~~{\bf : ARRAY~~[}~~
   \grammar{array-range}~~{\bf ]~~OF~~BOOLEAN} \\
\\
\> \grammar{array-range} \' ::= \>
   \grammar{i-range} \\
\> ~~$|$ \> 
   \grammar{i-range}~~{\bf ,}~~
   \grammar{i-range} \\
\end{bnf_tabbing}\newpage\begin{bnf_tabbing}
\> \grammar{space-destination-list} \' ::= \>
   \grammar{space-destination} \\
\> ~~$|$ \> 
   \grammar{space-destination}~~{\bf ,}~~
   \grammar{space-destination-list} \\
\\
\> \grammar{space-destination} \' ::= \>
   \grammar{dest-adj-clause} \\
\\
\> \grammar{dest-adj-clause} \' ::= \>
   \grammar{state-space-var}~~
   {\bf =}~~
   \grammar{whole-or-boolean-expression} \\
\> ~~$|$ \> \grammar{state-space-var}~~
   \grammar{inc-op} \\
\\
\\
\\
\\
\\
\> \grammar{for-range} \' ::= \>
   \grammar{index-variable}~~{\bf =}~~
   \grammar{whole-expression}~~{\bf ,}~~
   \grammar{whole-expression}~{\sf $^{\dagger}$} \\
\> ~~$|$ \>
   \grammar{index-variable}~~{\bf IN}~~
   \grammar{set} \\
\\
\> \grammar{set} \' ::= \>
   {\bf\boldmath $[$}~~
   \grammar{set-range-list}~~
   {\bf\boldmath $]$} \\
\\
\> \grammar{set-range-list} \' ::= \>
   \grammar{i-range} \\
\> ~~$|$ \> \grammar{whole-expression} \\
\> ~~$|$ \>
   \grammar{i-range}~~{\bf ,}~~
   \grammar{i-range-list} \\
\\
\> \grammar{i-range} \' ::= \>
   \grammar{lower-bound}~~{\bf ..}~~
   \grammar{upper-bound} \\
\\
\> \grammar{lower-bound} \' ::= \>
   \grammar{range-bound} \\
\\
\> \grammar{upper-bound} \' ::= \>
   \grammar{range-bound} \\
\\
\> \grammar{range-bound} \' ::= \>
   \grammar{whole-expression}~{\sf $^{\S}$} \\
\\
\\
\\
\> \grammar{rate-expression} \' ::= \>
   \grammar{real-expression} \\
\> ~~$|$ \> {\bf\boldmath $<$}~~\grammar{real-expression}~~
   {\bf ,}~~\grammar{real-expression}~~{\bf\boldmath $>$} \\
\> ~~$|$ \> {\bf\boldmath $<$}~~\grammar{real-expression}~~
   {\bf ,}~~\grammar{real-expression}~~{\bf ,}~~
   \grammar{real-expression}~~{\bf\boldmath $>$} \\
\> ~~$|$ \> {\bf FAST}~~\grammar{real-expression} \\
\\
\\
\\
\\
\\
\> \grammar{expr-list-with-of} \' ::= \>
   \grammar{expr} \\
\> ~~$|$ \>
   \grammar{whole-expression}~~{\bf OF}~~
   \grammar{expr} \\
\> ~~$|$ \>
   \grammar{expr}~~{\bf ,}~~
   \grammar{expr-list-with-of} \\
\> ~~$|$ \>
   \grammar{whole-expression}~~{\bf OF}~~
   \grammar{expr}~~{\bf ,}~~
   \grammar{expr-list-with-of} \\
\\
\> \grammar{expression-list} \' ::= \>
   \grammar{expr} \\
\> ~~$|$ \>
   \grammar{expr}~~{\bf ,}~~
   \grammar{expression-list} \\
\\
\> \grammar{built-in-expr-list} \' ::= \>
   \grammar{expr} \\
\> ~~$|$ \>
   \grammar{expr}~~{\bf ,}~~
   \grammar{built-in-expr-list} \\
\> ~~$|$ \>
   \grammar{wild-sub-array} \\
\> ~~$|$ \>
   \grammar{wild-sub-array}~~{\bf ,}~~
   \grammar{built-in-expr-list} \\
\\
\> \grammar{wild-sub-array} \' ::= \>
   \grammar{named-constant}~~
   {\bf [~~*~~,}~~\grammar{whole-expression}~~{\bf ]} \\
\> ~~$|$ \>
   \grammar{named-constant}~~
   {\bf [}~~\grammar{whole-expression}~~{\bf ,~~*~~]} \\
\> ~~$|$ \>
   \grammar{state-space-var}~~
   {\bf [~~*~~,}~~\grammar{whole-expression}~~{\bf ]} \\
\> ~~$|$ \>
   \grammar{state-space-var}~~
   {\bf [}~~\grammar{whole-expression}~~{\bf ,~~*~~]} \\
\\
\> \grammar{expr} \' ::= \>
   \grammar{real-expression} \\
\> ~~$|$ \> \grammar{whole-expression} \\
\> ~~$|$ \> \grammar{boolean-expression} \\
\\
\> \grammar{whole-or-boolean-expression} \' ::= \>
   \grammar{whole-expression} \\
\> ~~$|$ \> \grammar{boolean-expression} \\
\\
\> \grammar{whole-expression} \' ::= \>
   \grammar{integer-expression} \\
\\
\> \grammar{real-expression} \' ::= \>
   \grammar{numeric-expression} \\
\\
\> \grammar{integer-expression} \' ::= \>
   \grammar{numeric-expression} \\
\end{bnf_tabbing}\newpage\begin{bnf_tabbing}
\> \grammar{boolean-expression} \' ::= \>
   \grammar{bool-term-expr} \\
\\
\> \grammar{bool-term-expr} \' ::= \>
   \grammar{bool-term} \\
\> ~~$|$ \>
   \grammar{bool-term-expr}~~
   \grammar{or-op}~~
   \grammar{bool-term} \\
\\
\> \grammar{bool-term} \' ::= \>
   \grammar{bool-factor} \\
\> ~~$|$ \>
   \grammar{bool-term}~~
   \grammar{and-op}~~
   \grammar{bool-factor} \\
\\
\> \grammar{bool-factor} \' ::= \>
   \grammar{bool-item} \\
\> ~~$|$ \>
   \grammar{bool-item}~~
   {\bf\boldmath $==$}~~
   \grammar{bool-item} \\
\\
\> \grammar{bool-item} \' ::= \>
   \grammar{numeric-comparison} \\
\> ~~$|$ \>
   \grammar{simple-bool-item} \\
\\
\> \grammar{numeric-comparison} \' ::= \>
   \grammar{whole-expression}~~
   \grammar{relation}~~
   \grammar{whole-expression} \\
\\
\> \grammar{simple-bool-item} \' ::= \>
   \grammar{non-index-single-item} \\
\> ~~$|$ \> \grammar{truth-value} \\
\> ~~$|$ \>
   \grammar{boolean-function-invocation} \\
\> ~~$|$ \> {\bf (}~~\grammar{boolean-expression}~~{\bf )} \\
\> ~~$|$ \>
   {\bf NOT}~~
   \grammar{simple-bool-item} \\
\\
\\
\\
\\
\\
\> \grammar{or-op} \' ::= \> 
   {\bf OR} \\
\> ~~$|$ \> {\bf\boldmath $|$} \\
\> ~~$|$ \> {\bf XOR} \\
\\
\> \grammar{and-op} \' ::= \> {\bf AND} \\
\> ~~$|$ \> {\bf \&} \\
\\
\> \grammar{relation} \' ::= \> 
   \grammar{inequality-relation} \\
\> ~~$|$ \>
   \grammar{equality-relation} \\
\\
\> \grammar{inequality-relation} \' ::= \> 
   {\bf\boldmath $>$} \\
\> ~~$|$ \> {\bf\boldmath $<$} \\
\> ~~$|$ \> {\bf\boldmath $>=$} \\
\> ~~$|$ \> {\bf\boldmath $<=$} \\
\\
\> \grammar{equality-relation} \' ::= \> 
   {\bf\boldmath $<>$} \\
\> ~~$|$ \> {\bf =} \\
\end{bnf_tabbing}\newpage\begin{bnf_tabbing}
\> \grammar{numeric-expression} \' ::= \>
   \grammar{term-expr} \\
\\
\> \grammar{term-expr} \' ::= \>
   \grammar{term} \\
\> ~~$|$ \> \grammar{term-expr}~~
   \grammar{add-op}~~
   \grammar{term} \\
\\
\> \grammar{term} \' ::= \> \grammar{factor} \\
\> ~~$|$ \> \grammar{term}~~
   \grammar{mpy-op}~~
   \grammar{factor} \\
\\
\> \grammar{factor} \' ::= \>
   \grammar{numeric-item} \\
\> ~~$|$ \> \grammar{numeric-item}~~
   \grammar{pow-op}~~\grammar{factor} \\
\\
\> \grammar{numeric-item} \' ::= \>
   \grammar{bin-numeric-item} \\
\> ~~$|$ \> \grammar{sign-op}~~
   \grammar{numeric-item} \\
\\
\> \grammar{bin-numeric-item} \' ::= \>
   \grammar{non-index-single-item} \\
\> ~~$|$ \> \grammar{index-variable} \\
\> ~~$|$ \> \grammar{unsigned-value} \\
\> ~~$|$ \> \grammar{named-constant}~~
   \grammar{cat-op}~~
   \grammar{bin-numeric-item} \\
\> ~~$|$ \> \grammar{numeric-function-invocation} \\
\> ~~$|$ \> {\bf (}~~\grammar{numeric-expression}~~{\bf )} \\
\\
\\
\\
\> \grammar{add-op} \' ::= \> {\bf +} \\
\> ~~$|$ \> {\bf\boldmath $-$} \\
\\
\> \grammar{mpy-op} \' ::= \> {\bf *} \\
\> ~~$|$ \> {\bf /} \\
\> ~~$|$ \> {\bf MOD} \\
\> ~~$|$ \> {\bf CYC} \\
\> ~~$|$ \> {\bf DIV} \\
\\
\> \grammar{pow-op} \' ::= \> {\bf **} \\
\\
\> \grammar{sign-op} \' ::= \> {\bf\boldmath $-$} \\
\\
\> \grammar{inc-op} \' ::= \> {\bf ++} \\
\> ~~$|$ \> {\bf\boldmath $--$} \\
\\
\> \grammar{cat-op} \' ::= \> {\bf\boldmath $\wedge$} \\
\end{bnf_tabbing}\newpage\begin{bnf_tabbing}
\\
\> \grammar{boolean-function-invocation} \' ::= \>
   \grammar{function-invocation} \\
\\
\> \grammar{numeric-function-invocation} \' ::= \>
   \grammar{function-invocation} \\
\\
\> \grammar{function-invocation} \' ::= \>
   \grammar{impl-func-name} \\
\> ~~$|$ \>
   \grammar{function-name}~~{\bf (}~~
   \grammar{expression-list}~~{\bf )} \\
\> ~~$|$ \>
   \grammar{built-in-name}~~{\bf (}~~
   \grammar{built-in-expr-list}~~{\bf )} \\
\\
\\
\\
\> \grammar{non-index-single-item} \' ::= \>
   \grammar{named-constant} \\
\> ~~$|$ \> \grammar{named-constant}~~
   {\bf ~~[}~~\grammar{whole-expression}~~{\bf ]} \\
\> ~~$|$ \> \grammar{named-constant}~~
   {\bf ~~[}~~\grammar{whole-expression}~~{\bf ,}~~
              \grammar{whole-expression}~~{\bf ]} \\
\> ~~$|$ \> \grammar{state-space-var} \\
\> ~~$|$ \> \grammar{state-space-var}~~
   {\bf ~~[}~~\grammar{whole-expression}~~{\bf ]} \\
\> ~~$|$ \> \grammar{state-space-var}~~
   {\bf ~~[}~~\grammar{whole-expression}~~{\bf ,}~~
              \grammar{whole-expression}~~{\bf ]} \\
\\
\\
\\
\> \grammar{function-name} \' ::= \>
   \grammar{identifier} \\
\\
\> \grammar{impl-func-name} \' ::= \>
   \grammar{identifier} \\
\\
\builtintabbing
\> \grammar{built-in-name} \' ::= \>
            {\bf SQRT}   \> ~~$|$ \> {\bf EXP}    \> ~~$|$ \> {\bf LN} \\
\> ~~$|$ \> {\bf SIN}    \> ~~$|$ \> {\bf COS}    \> ~~$|$ \> {\bf TAN} \\
\> ~~$|$ \> {\bf ARCSIN} \> ~~$|$ \> {\bf ARCCOS} \> ~~$|$ \> {\bf ARCTAN} \\
\> ~~$|$ \> {\bf FACT}   \> ~~$|$ \> {\bf SUM}    \> ~~$|$ \> {\bf COUNT} \\
\> ~~$|$ \> {\bf COMB}   \> ~~$|$ \> {\bf PERM}   \> ~~$|$ \> {\bf ABS} \\
\> ~~$|$ \> {\bf ANY}    \> ~~$|$ \> {\bf ALL}    \> ~~$|$ \> {\bf SIZE} \\
\> ~~$|$ \> {\bf MIN}    \> ~~$|$ \> {\bf MAX} \\
%  \> ~~$|$ \> {\bf ROWSUM} \> ~~$|$ \> {\bf COLSUM} \> ~~$|$ \> {\bf ANY\_POSITIVE} \\
% \> ~~$|$ \> {\bf ROWCOUNT} \> ~~$|$ \> {\bf COLCOUNT} \> ~~$|$ \> {\bf ALL\_POSITIVE} \\
\bnftabbing
\\
\\
\\
\> \grammar{truth-value} \' ::= \> {\bf FALSE} \\
\> ~~$|$ \> {\bf TRUE} \\
\\
\> \grammar{comment} \' ::= \>
   {\bf (*}~~\grammar{text}~~{\bf *)} \\
\> ~~$|$ \>
   {\bf \{}~~\grammar{text}~~{\bf \}} \\
\\
\> \grammar{under} \' ::= \> {\bf\boldmath $-$} \\
\\
\> \grammar{dollar} \' ::= \> {\bf \$} \\
\\
\digittabbing
\> \grammar{E-char} \' ::= \> 
   {\bf E}\>$|$~{\bf e}\>$|$~{\bf D}\>$|$~{\bf d} \\
\end{bnf_tabbing}\newpage\begin{bnf_tabbing}
\digittabbing
\> \grammar{letter} \' ::= \> 
   {\bf A}\>$|$~{\bf B}\>$|$~{\bf C}\>$|$~{\bf D}\>$|$~{\bf E} \\
\>~~$|$\>{\bf F}\>$|$~{\bf G}\>$|$~{\bf H}\>$|$~{\bf I}\>$|$~{\bf J} \\
\>~~$|$\>{\bf K}\>$|$~{\bf L}\>$|$~{\bf M}\>$|$~{\bf N}\>$|$~{\bf O} \\
\>~~$|$\>{\bf P}\>$|$~{\bf Q}\>$|$~{\bf R}\>$|$~{\bf S}\>$|$~{\bf T} \\
\>~~$|$\>{\bf U}\>$|$~{\bf V}\>$|$~{\bf W}\>$|$~{\bf X}\>$|$~{\bf Y} \\
\>~~$|$\>{\bf Z} \\
\>~~$|$\>{\bf a}\>$|$~{\bf b}\>$|$~{\bf c}\>$|$~{\bf d}\>$|$~{\bf e} \\
\>~~$|$\>{\bf f}\>$|$~{\bf g}\>$|$~{\bf h}\>$|$~{\bf i}\>$|$~{\bf j} \\
\>~~$|$\>{\bf k}\>$|$~{\bf l}\>$|$~{\bf m}\>$|$~{\bf n}\>$|$~{\bf o} \\
\>~~$|$\>{\bf p}\>$|$~{\bf q}\>$|$~{\bf r}\>$|$~{\bf s}\>$|$~{\bf t} \\
\>~~$|$\>{\bf u}\>$|$~{\bf v}\>$|$~{\bf w}\>$|$~{\bf x}\>$|$~{\bf y} \\
\>~~$|$\>{\bf z} \\
\\
\> \grammar{digit} \' ::= \>
   {\bf 0}\>$|$~{\bf 1}\>$|$~{\bf 2}\>$|$~{\bf 3}\>$|$~{\bf 4} \\
\>~~$|$\>{\bf 5}\>$|$~{\bf 6}\>$|$~{\bf 7}\>$|$~{\bf 8}\>$|$~{\bf 9} \\
\bnftabbing
\\
\> \grammar{ident-char} \' ::= \>
   \grammar{letter} \\
\> ~~$|$ \> \grammar{digit} \\
\> ~~$|$ \> \grammar{under} \\
\> ~~$|$ \> \grammar{dollar}~{\sf $^{\ddagger}$} \\
\\
\> \grammar{identifier} \' ::= \>
   \grammar{letter} \\
\> ~~$|$ \> \grammar{letter}~~
   \grammar{ident-rest} \\
\\
\> \grammar{ident-rest} \' ::= \>
   \grammar{ident-char} \\
\> ~~$|$ \> \grammar{ident-char}~~
   \grammar{ident-rest} \\
\\
\> \grammar{unsigned-integer-value} \' ::= \>
   \grammar{digit} \\
\> ~~$|$ \> \grammar{digit}~~
   \grammar{unsigned-integer-value} \\
\\
\> \grammar{unsigned-real-value} \' ::= \>
   \grammar{unsigned-integer-value}~~{\bf .}~~
   \grammar{unsigned-integer-value} \\
\> ~~$|$ \> \grammar{unsigned-integer-value}~~{\bf .}~~
   \grammar{unsigned-integer-value}~~
   \grammar{exponent-value} \\
\\
\> \grammar{exponent-value} \' ::= \>
   \grammar{E-char}~~\grammar{sign-op}~~
   \grammar{unsigned-integer-value} \\
\> ~~$|$ \> \grammar{E-char}~~
   \grammar{unsigned-integer-value} \\
\\
\> \grammar{named-constant} \' ::= \> 
   \grammar{identifier} \\
\\
\> \grammar{state-space-var} \' ::= \> 
   \grammar{identifier} \\
\\
\> \grammar{index-variable} \' ::= \> 
   \grammar{identifier} \\
\\
\foottabbing
{\sf $\sharp$} \> {\apdixfootsiz\sf 
     The C\_OPTION statement can be repeated but will usually precede
     any other statements.}
{\sf $\dagger$} \> {\apdixfootsiz\sf This syntax is obsolete at revision 7.0 --
 its use will result in warning message.} \\
{\sf $\ddagger$} \> {\apdixfootsiz\sf 
     All identifiers with dollar signs are reserved by the ASSIST language.} \\
{\sf $\S$} \>
    {\apdixfootsiz\sf Although lower and upper bounds can take on
     values between 0 and 32767, their difference must be no more
     than 255.} \\
\end{bnf_tabbing}

\section{Command Line Options}
\label{apix:CommandLineOpts}

The \acronym{ASSIST} command line allows the user to specify options.   These options
control a number of parameters and allow the user more control over how
the \acronym{ASSIST} program executes.

Options must be preceded by a slash under \acronym{VMS} as in:
\begin{codeexample}
\vmsoption{map}
\end{codeexample}
and must be preceded by a dash under {\acronym UNIX} as in:
\begin{codeexample}
\unixoption{map}
\end{codeexample}

Options may be specified either in upper or lower case.   The normal UNIX
case sensitivity does not apply to the \acronym{ASSIST} command line options.

Options may also be typed into the input file via \rw{C\_OPTION} commands.
These commands must precede all other commands including any other debug
commands.   For example, the statement:
\begin{codeexample}
C_OPTION LEL\(=\)10;
\end{codeexample}
in the input file is the same as the following command-line options:
\begin{codeexample}
\vmsliteral{\(/\)lel\(=\)10}
\orline{3mm}
\unixliteral{\(-\)lel\(=\)10}
\end{codeexample}

The following options are available:

\begin{footnotesize}\begin{itemize}
\item \option{c} $\longrightarrow$ Specifies identifier case sensitivity.
      Use of ``\option{c}'' is not recommended since SURE is never case 
      sensitive.  Case sensitive state-space variables are okay because they
      are never passed
      to SURE.  Case sensitive constant names will cause problems because they
      are passed to SURE.
      The default is no case sensitive identifier names.
\item \option{pipe} $\longrightarrow$ This option causes the model output to be
      written to standard output instead of to a model file.   It is useful
      if one wishes to pipe the model directly to SURE.   This option is valid
      only under UNIX.   An attempt to use it under \acronym{VMS}
      will cause \acronym{ASSIST} to print an error message.
      The default is no rerouting of the model file to the standard output file.
\item \option{map} $\longrightarrow$
      This option causes \acronym{ASSIST} to produce a cross
      reference map of all of the
      definitions of and references to identifiers and literal values in the
      program.   The map also tells which \rw{ENDFOR} matches which \rw{FOR}
      and which \rw{ENDIF} matches which \rw{IF}.   It also indicates to which
      \rw{IF} an \rw{ELSE} belongs.
      Although the map is several pages long, it may help the user to find
      misspelled identifiers.   Its use is recommended during the first few
      executions
      against a new input file.
      The default is no cross reference map.
\item \option{xref} $\longrightarrow$
      This option is the same as the \option{map} option.
\item \option{loadmap} $\longrightarrow$
      This option is used to request a load map of the internal data structures
      and memory allocation generated during the parsing of the input file.
      The information produced is extremely technical.   The option remains in
      the language for verification purposes and because it is useful under
      some rare instances.   Its use is {\bf not}
      recommended.   Use of \option{xref} is recommended instead.
      The default is no load map.
\item \option{ss} $\longrightarrow$
      This option forces \acronym{ASSIST} to print the level of each
      warning as part of
      the warning message.   For example, instead of \word{[WARNING]}, the
      message will read \word{[WARNING SEVERITY 3]}
      The default is no display of warning severity.
\item \option{we3} $\longrightarrow$
      This option forces \acronym{ASSIST} to abbreviate to three letters
      in warning and error messages as in \word{[ERR]} and \word{[WRN]}
      The default is no abbreviation of the words ``ERROR'' and ``WARNING''.
\item \option{bat} $\longrightarrow$
      This option causes \acronym{ASSIST} to execute in batch mode.   In batch
      mode, the command line is echoed to standard error (usually
      the user's monitor screen).
      The default is no echoing of the command line used to
      invoke \acronym{ASSIST}.
\item \option{wid}$=${\em nnn} $\longrightarrow$
      This option specifies the overflow length of a line.  The default is 80
      characters, which results in an effective input line length
      of 79 characters.
\item \option{tab}$=${\em nnn} $\longrightarrow$
      This option specifies how many spaces are equivalent to a tab character.
      The default is four spaces per tab.
\item \option{nest}$=${\em nnn} $\longrightarrow$
      Specifies how deeply a space statement
      can be recursively nested.   The default is 16 on most systems
      (8 on the \acronym{IBM PC}).
\item \option{rule}$=${\em nnn} $\longrightarrow$
      Specifies the maximum number of rules that can be nested inside a
      single block
      \rw{IF} or \rw{FOR} construct.   The default is 4096 for
      most systems (1024 on the \acronym{IBM PC}).
\item \option{pic}$=${\em nnn} $\longrightarrow$
      Specifies the maximum number of nodes that can be on the stack when
      parsing a state-space picture.   The number of state-space variables
      may exceed this
      number only if the state-space picture is recursively defined.
      The default is 100.
\item \option{lel}$=${\em nnn} $\longrightarrow$
      Specifies the ``line error limit''.   If the number of errors per line
      ever exceeds this value, then \acronym{ASSIST} will quit
      processing the input file
      immediately after printing one additional and appropriate error message.
      The default is a maximum of 5 errors allowed per line.
\item \option{lwl}$=${\em nnn} $\longrightarrow$
      Specifies the ``line warning limit''.   If the number of warnings per line
      ever exceeds this value, then \acronym{ASSIST} will quit processing
       the input file
      immediately after printing an appropriate error message.
      The default is a maximum of 5 warnings allowed per line.
\item \option{el}$=${\em nnn} $\longrightarrow$
      Specifies the ``error limit''.   If the cumulative number of errors
      ever exceeds this value, then \acronym{ASSIST} will quit processing
      the input file
      immediately after printing one additional and appropriate error message.
      The default is a maximum of 40 errors allowed per input file.
\item \option{wl}$=${\em nnn} $\longrightarrow$
      Specifies the ``warning limit''.   If the cumulative number of warnings
      ever exceeds this value, then \acronym{ASSIST} will quit processing
      the input file
      immediately after printing an appropriate error message.
      The default is a maximum of 40 warnings allowed per input file.
\item \option{bc}$=${\em nnn} $\longrightarrow$
      Specifies the ``bucket count'' for the rule generation state hashing
      algorithm.   If rule generation is taking a long time
      because of identifier hash clashes, then this value can be adjusted.
      The default bucket count is 1009.
\item \option{bi}$=${\em nnn} $\longrightarrow$
      Specifies the ``bucket increment''.   This controls how many additional
      state buckets will be allocated at a time when the system runs out
      of buckets.
\item \option{bw}$=${\em nnn} $\longrightarrow$
      Specifies the ``bucket width'' (i.e., the number of states that will
      fit in a single link of the linked list for each bucket) 
      for the rule generation state hashing
      algorithm.   If rule generation is taking a long time
      because of identifier hash clashes, then this value can be adjusted.
      The default bucket width is 5.
\item \option{lp}$=${\em nnn} $\longrightarrow$
      Specifies the number of lines per page on the log file.
      The default is 58 lines maximum per page on the log file.
\item \option{i}$=${\em nnn} $\longrightarrow$
      Specifies the maximum number of identifiers
      that can be held in the 
      identifier table.
      The default is a maximum of 400 unique identifier names in the table
      for most systems (200 on the \acronym{IBM PC}).
\item \option{n}$=${\em nnn} $\longrightarrow$
      Specifies the maximum number of literal values
      that can be held in the
      identifier table.   Note that ``6.0'' and ``6.00'' are considered
      as two different entries in the table so that they can be written
      to the model file the same way they were typed into the input file.
      The default is a maximum of 200 unique numerical values in the table
      for most systems (50 on the \acronym{IBM PC}).
\item \option{o}$=${\em nnn} $\longrightarrow$
      Specifies the maximum number of operands
      that can be held in the
      expression operand list while parsing a single statement.
      The default is 300 on most systems (50 on the \acronym{IBM PC}).  The
      maximum number of infix/postfix operations
      is a function of this number and is always significantly greater.
\item \option{e}$=${\em nnn} $\longrightarrow$
      Specifies the maximum number of expressions
      that can be held while parsing a single statement.
      The default is 300 on most systems (50 on the \acronym{IBM PC}).
\item \option{p}$=${\em nnn} $\longrightarrow$
      Specifies the maximum number of identifiers for
      a FUNCTION or IMPLICIT or VARIABLE parameter list.
      The default is 64 on most systems (32 on the \acronym{IBM PC}).
\item \option{b}$=${\em nnn} $\longrightarrow$
      Specifies the maximum number of tokens in the body per FUNCTION or
      IMPLICIT or VARIABLE definition.
      The default is 1024 on most systems (256 on the \acronym{IBM PC}).
\item \option{w}$=${\em nnn} $\longrightarrow$
      Specifies the levels of warnings that will be issued.   The higher
      the number, the more warnings.   Levels available are 0 for no warnings
      through 99 for all warnings.   There are currently only three levels
      defined as detailed in Appendix \ref{apix:errors}.
      The default is two levels of warning reporting.
      The \word{W$=$FEWER} form decreases the level to one less level of
      warnings.
%      The \word{W$=$MORE} form increases the level to one greater than
%          the default.
%      The \word{W$=$DEFAULT} form resets the level to the default value.
      The \word{W$=$NONE} form suppresses all warnings.
      The \word{W$=$ALL} form enables all warnings.
\end{itemize}\end{footnotesize}

\section{Notes for VMS Users}
\label{apix:vms}

Certain peculiarities of the \acronym{VMS} operating system are described
in this appendix.
The VMS user should keep these in mind when
running \acronym{ASSIST} in order to
save time and frustration.

\subsection{Special VMS Errors}

The first thing to note about \acronym{VMS} is that there are
certain errors\index{VMS}\label{lab:vmserror} that
the \acronym{VMS} operating system detects that standard \acronym{UNIX} ``C''
does not.  These
errors do not have ``C'' error numbers.   Since {\acronym ASSIST} is written in
``C'' it was decided that, when one of these special error numbers would
arise, the number of the error would be printed.
For example, consider the following
screen session from {\bf ASSIST}:
\begin{codeexample}
[ERROR] SPECIAL VMS ERROR NUMBER: 100052
[ERROR] QUITTING COMPILATION !!!

0002 ERRORS.
$
\end{codeexample}

To check the meaning of the error number 100052, use the VMS \word{exit}
command as illustrated in the following example:
\begin{codeexample}
$ exit 100052
%RMS-F-SYN, file specification syntax error
$
\end{codeexample}

\subsection{Model Cannot be Piped}

The next thing to note about \acronym{VMS} is that, unlike \acronym{UNIX}
and \acronym{IBM MS DOS}, the standard output from one command cannot be
piped to the standard input of another command.   The \vmsoption{pipe} option
is therefore disallowed on \acronym{VMS} systems.



\end{document}
